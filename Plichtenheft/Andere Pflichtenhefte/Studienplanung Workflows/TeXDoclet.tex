\documentclass[11pt,a4paper]{report}
\usepackage{color}
\usepackage{ifthen}
\usepackage{ifpdf}
\usepackage[headings]{fullpage}
\usepackage{listings}

\usepackage{multido}
\usepackage{makeidx} %not generated
\usepackage[toc,page]{appendix}
\usepackage[nottoc]{tocbibind} %not generated
%\usepackage[totoc]{idxlayout} %not generated

\lstset{language=Java,breaklines=true}
\ifpdf \usepackage[pdftex, pdfpagemode={UseOutlines},bookmarks,colorlinks,linkcolor={blue},plainpages=false,pdfpagelabels,citecolor={red},breaklinks=true]{hyperref}
  \usepackage[pdftex]{graphicx}
  \pdfcompresslevel=9
  \DeclareGraphicsRule{*}{mps}{*}{}
\else
  \usepackage[dvips]{graphicx}
\fi

\newcommand{\entityintro}[3]{%
  \hbox to \hsize{%
    \vbox{%
      \hbox to .2in{}%
    }%
    {\bf  #1}%
    \dotfill\pageref{#2}%
  }
  \makebox[\hsize]{%
    \parbox{.4in}{}%
    \parbox[l]{5in}{%
      \vspace{1mm}%
      #3%
      \vspace{1mm}%
    }%
  }%
}
\newcommand{\refdefined}[1]{
\expandafter\ifx\csname r@#1\endcsname\relax
\relax\else
{$($in \ref{#1}, page \pageref{#1}$)$}\fi}
\renewcommand\thepart{\arabic{part}}
\date{\today}
\chardef\textbackslash=`\\

\makeindex

\title{Design Document Of An Online Study Planning Service}

\date{January 11, 2017}
\begin{document}
	\sloppy
	\addtocontents{toc}{\protect\markboth{Contents}{Contents}}
	
	\begin{titlepage}
		\centering
		{\scshape\LARGE KIT - Karlsruhe Institute of Technology \par}
		\vspace{1cm}
		{\scshape\Large Design and architecture report\par}
		\vspace{1.5cm}
		{\huge\bfseries %Studienplanung als Generierung von Workflows mit Compliance-Anforderungen: Planerstellung und Visualisierung\par 
		Study Planning by Generating Workflows with Compliance Requirements: Plan Creation and Visualisation\par
		}
		\vspace{2cm}
		{\large January 11, 2017\par}
		\vspace{1cm}
		\textsc{IPD B{\"o}hm}
		\vfill
		\begin{tabular}{| l | l |}
			\hline
			\textbf{Phase} & \textbf{Supervisor} \\ \hline
			Functional specification & Ali Bejhad \\
			Design and architecture & Janek Westfechtel \\
			Implementation & Clemens Naseband \\
			Qualitycontrol & Robin Berger \\
			Presentation  & Jacques Huss \\ \hline
		\end{tabular}
	\end{titlepage}

	\tableofcontents

\part{General}{
	
	\chapter{Introduction}{
		This document will guide through our software design, using UML-diagrams. The main topic of this document will be the class diagram, inspired by the MVC-principle. The interaction of different objects will be presented in several sequence diagrams. \\
		All desired criteria will be met, but not all interfaces have been added, to avoid empty methods in the implementation, but it is ensured that every missing part can be added easily.
	}
	
	\chapter{Design Overview}{
		
		\section{Architecture}{
			\begin{figure}[ht]
				\centering
				\includegraphics[width=\textwidth]{res/Systemmodell-englisch}
				\caption{MVC-Architecture}
			\end{figure}
			
			The program is designed using the principles of the MVC-architecture. This divides it into three different components Model, View, and Controller, making it easier to modify certain aspects, change strategies, and porting it to other platforms. 
			\\ \\
			{\bf Model:}
			\\
			The Model contains the user, system data and the workflow algorithms. A database is offering the module manual with all of it's constraints. The algorithms can verify and generate workflows.
			\\ \\
			{\bf View}
			\\
			The View has 2 tasks: On the one hand it provides the user with a GUI to create a workflow with the given preferences. On the other hand it presents the generated or verified workflow.\\
			\\ \\
			{\bf Controller}
			\\
			The Controller is the central unit of the system. It coordinates the generation and verification of the workflows by translating the user input, received from the View, into the needed parameters from the Model. After that the new or verified workflow, received from the Model, is transferred to the View, so that it can display it.
		}
		
		\clearpage
		
		\section{Class diagram}{
			\begin{figure}[ht]
				\centering
				%\includegraphics[width=\textwidth, height=.5\textheight]{res/ClassDiagram/allMarked}
				\includegraphics[width=\textwidth, height= .7\textheight, angle=90]{res/ClassDiagram/allMarked}
				\caption{Class diagram}
			\end{figure}
			\begin{center}
				Model (Green), View (Yellow), Controller (Blue), Java libraries (White)
			\end{center}
		}
	}
}

\part{Controller}{
	
	\begin{figure}[ht]
		\centering
		\includegraphics[width=\textwidth]{res/ClassDiagram/controllerMarkedCompressed}
		\caption{Controller}
	\end{figure}
	
	\chapter{Package studyplanning.controller.commands}{
		\label{studyplanning.controller.commands}\hypertarget{studyplanning.controller.commands}{}
		\hskip -.05in
		\hbox to \hsize{\textit{ Package Contents\hfil Page}}
		\vskip .13in
		\hbox{{\bf  Classes}}
		\entityintro{AddModuleCommand}{studyplanning.controller.commands.AddModuleCommand}{Adds a Module to an existing Workflow}
		\entityintro{Command}{studyplanning.controller.commands.Command}{The abstract Command class exists so that new Commands can be created and added with the help of inheritance.}
		\entityintro{GenerateCommand}{studyplanning.controller.commands.GenerateCommand}{Generates a completely new Workflow.}
		\entityintro{IOCommand}{studyplanning.controller.commands.IOCommand}{The IOCommand manages user data, and is used to save and load Workflows.}
		\entityintro{RemoveModuleCommand}{studyplanning.controller.commands.RemoveModuleCommand}{Removes Modules from a Workflow}
		\entityintro{VerifyCommand}{studyplanning.controller.commands.VerifyCommand}{Verifies whether the Workflow is correct and complies with the Module manuals and the users standards}
		\vskip .1in
		\vskip .1in
		\section{\label{studyplanning.controller.commands.AddModuleCommand}Class \index{AddModuleCommand} AddModuleCommand}{
			\hypertarget{studyplanning.controller.commands.AddModuleCommand}{}\vskip .1in 
			Adds a Module to an existing Workflow\vskip .1in 
		
			\subsection{Declaration}{
				\begin{lstlisting}[frame=none]
public class AddModuleCommand
  extends studyplanning.controller.commands.Command
				\end{lstlisting}
			}
			
			\subsection{Constructor summary}{
				\begin{verse}
					\hyperlink{studyplanning.controller.commands.AddModuleCommand()}{{\bf AddModuleCommand()}} Creates a new AddModuleCommand instance\\
				\end{verse}
			}
			
			\subsection{Method summary}{
				\begin{verse}
					\hyperlink{studyplanning.controller.commands.AddModuleCommand.execute(java.lang.String)}{{\bf execute(String)}} \\
				\end{verse}
			}
			
			\subsection{Constructors}{
				\vskip -2em
				\begin{itemize}
					\item{ 
						\index{AddModuleCommand!AddModuleCommand()}
						\hypertarget{studyplanning.controller.commands.AddModuleCommand()}{{\bf  AddModuleCommand}\\}
						\begin{lstlisting}[frame=none]
public AddModuleCommand()
						\end{lstlisting} %end signature
						\begin{itemize}
							\item{
								{\bf  Description}
								Creates a new AddModuleCommand instance
							}
						\end{itemize}
					}%end item
				\end{itemize}
			}
		
			\subsection{Methods}{
				\vskip -2em
				\begin{itemize}
					\item{ 
						\index{AddModuleCommand!execute(String)}
						\hypertarget{studyplanning.controller.commands.AddModuleCommand.execute(java.lang.String)}{{\bf  execute}\\}
						\begin{lstlisting}[frame=none]
public abstract Response execute(String input)
						\end{lstlisting} %end signature
						\begin{itemize}
							\item{
								{\bf Description}
								Adds a Module to an existing Workflow
							}
							\item{
								{\bf  Parameters}
								\begin{itemize}
									\item{\texttt{input} -- A message generated by the View.}
								\end{itemize}
							}%end item
							\item{
								{\bf  Returns} -- A Response containing the changed Workflow. If the addition violates any Contraints, some error messages will be contained in the Response.
							}%end item
						\end{itemize}
					}%end item
				\end{itemize}
			}
		
			\subsection{Members inherited from class Command }{
				\texttt{studyplanning.controller.commands.Command} 
				{\small \refdefined{studyplanning.controller.commands.Command}}
				{
					\small 
					\vskip -2em
					\begin{itemize}
						\item{\vskip -1.5ex 
							\texttt{public abstract Response {\bf  execute}(\texttt{java.lang.String} {\bf  input})}%end signature
						}%end item
					\end{itemize}
				}
			}
			
		\section{\label{studyplanning.controller.commands.Command}Class \index{Command} Command}{
			\hypertarget{studyplanning.controller.commands.Command}{}\vskip .1in 
			With the abstract Command class commands can be created and added with the help of inheritance.\vskip .1in 
			
			\subsection{Declaration}{
				\begin{lstlisting}[frame=none]
public abstract class Command
  extends java.lang.Object
				\end{lstlisting}
			}	
			
			\subsection{All known subclasses}{
					VerifyCommand\small{\refdefined{studyplanning.controller.commands.VerifyCommand}}, RemoveModuleCommand\small{\refdefined{studyplanning.controller.commands.RemoveModuleCommand}}, IOCommand\small{\refdefined{studyplanning.controller.commands.IOCommand}}, GenerateCommand\small{\refdefined{studyplanning.controller.commands.GenerateCommand}}, AddModuleCommand\small{\refdefined{studyplanning.controller.commands.AddModuleCommand}}
			}
				
			\subsection{Constructor summary}{
				\begin{verse}
					\hyperlink{studyplanning.controller.commands.Command()}{{\bf Command()}} \\
				\end{verse}
			}
				
			\subsection{Method summary}{
				\begin{verse}
					\hyperlink{studyplanning.controller.commands.Command.execute(java.lang.String)}{{\bf execute(String)}} This method will take all actions required to process a request. It will return a Response indicating the outcome of the Command.\\
				\end{verse}
			}
				
			\subsection{Constructors}{
				\vskip -2em
				\begin{itemize}
					\item{ 
						\index{Command!Command()}
						\hypertarget{studyplanning.controller.commands.Command()}{{\bf  Command}\\}
						\begin{lstlisting}[frame=none]
public Command()
						\end{lstlisting} %end signature
					}%end item
				\end{itemize}
			}
				
			\subsection{Methods}{
				\vskip -2em
				\begin{itemize}
					\item{ 
						\index{Command!execute(String)}
						\hypertarget{studyplanning.controller.commands.Command.execute(java.lang.String)}{{\bf  execute}\\}
						\begin{lstlisting}[frame=none]
public abstract Response execute(String input)
						\end{lstlisting} %end signature
						\begin{itemize}
							\item{
								{\bf  Description}
								This method will take all actions required to process a request. It will return a Response indicating the outcome of the Command.
							}
							\item{
								{\bf  Parameters}
								\begin{itemize}
									\item{\texttt{input} -- A message generated by the View.}
								\end{itemize}
							}%end item
							\item{
								{\bf  Returns} 
								-- A Response the contents of which will be specified in the subclasses
								}%end item
							\end{itemize}
					}%end item
				\end{itemize}
			}
		}
		\section{\label{studyplanning.controller.commands.GenerateCommand}Class \index{GenerateCommand} GenerateCommand}{
			\hypertarget{studyplanning.controller.commands.GenerateCommand}{}\vskip .1in 
			Generates a completely new Workflow. 
			The new Workflow will be created according to the constraints, and will be completed by the Model\vskip .1in 
			\subsection{Declaration}{
				\begin{lstlisting}[frame=none]
public class GenerateCommand
  extends studyplanning.controller.commands.Command\end{lstlisting}
			}
				
			\subsection{Constructor summary}{
				\begin{verse}
					\hyperlink{studyplanning.controller.commands.GenerateCommand()}{{\bf GenerateCommand()}} Creates a new GenerateCommand instance\\
				\end{verse}
			}
				
			\subsection{Method summary}{
				\begin{verse}
					\hyperlink{studyplanning.controller.commands.GenerateCommand.execute(java.lang.String)}{{\bf execute(String)}} Creates a new Workflow, using the algorithms in the Model package.\\
				\end{verse}
			}
				
			\subsection{Constructors}{
				\vskip -2em
				\begin{itemize}
					\item{ 
						\index{GenerateCommand!GenerateCommand()}
						\hypertarget{studyplanning.controller.commands.GenerateCommand()}{{\bf  GenerateCommand}\\}
						\begin{lstlisting}[frame=none]
public GenerateCommand()
						\end{lstlisting} %end signature
						\begin{itemize}
							\item{
								{\bf  Description}
								Creates a new instance
							}
						\end{itemize}
					}%end item
				\end{itemize}
			}
			
			\subsection{Methods}{
				\vskip -2em
				\begin{itemize}
					\item{ 
						\index{GenerateCommand!execute(String)}
						\hypertarget{studyplanning.controller.commands.GenerateCommand.execute(java.lang.String)}{{\bf  execute}\\}
						\begin{lstlisting}[frame=none]
public Response execute(String input)
						\end{lstlisting} %end signature
						\begin{itemize}
							\item{
								{\bf  Description}
								Creates a new Workflow, using the algorithms in the Model package.
							}
							\item{
								{\bf  Parameters}
								\begin{itemize}
									\item{\texttt{input} -- The message generated by the View}
								\end{itemize}
							}%end item
							\item{
								{\bf  Returns} 
								-- A Response containing the newly generated Workflow.\\
								Errors during the creation of the Workflows or, if the constraints don't allow the creation of a correct Workflow, will cause the Response to contain the error messages
							}%end item
						\end{itemize}
					}%end item
				\end{itemize}
			}
			\subsection{Members inherited from class Command }{
				\texttt{studyplanning.controller.commands.Command} 
				{\small\refdefined{studyplanning.controller.commands.Command}}
				{
					\small 
					\vskip -2em
					\begin{itemize}
						\item{\vskip -1.5ex 
							\texttt{public abstract Response {\bf  execute}(\texttt{java.lang.String} {\bf  input})}%end signature
						}%end item
					\end{itemize}
				}
			}
		\section{\label{studyplanning.controller.commands.IOCommand}Class \index{IOCommand}IOCommand}{
			\hypertarget{studyplanning.controller.commands.IOCommand}{}\vskip .1in 
			The IOCommand manages user data, and is used to save and load Workflows.\vskip .1in 
			
			\subsection{Declaration}{
				\begin{lstlisting}[frame=none]
public class IOCommand
  extends studyplanning.controller.commands.Command
				\end{lstlisting}
			}
			
			\subsection{Constructor summary}{
				\begin{verse}
					\hyperlink{studyplanning.controller.commands.IOCommand()}{{\bf IOCommand()}} Creates a new IOCommand instance\\
				\end{verse}
			}
			
			\subsection{Method summary}{
				\begin{verse}
					\hyperlink{studyplanning.controller.commands.IOCommand.execute(java.lang.String)}{{\bf execute(String)}} \\
				\end{verse}
			}
			
			\subsection{Constructors}{
				\vskip -2em
				\begin{itemize}
					\item{ 
						\index{IOCommand!IOCommand()}
						\hypertarget{studyplanning.controller.commands.IOCommand()}{{\bf  IOCommand}\\}
						\begin{lstlisting}[frame=none]
public IOCommand()
						\end{lstlisting} %end signature
						\begin{itemize}
							\item{
								{\bf  Description}
								Creates a new IOCommand instance
							}
						\end{itemize}
					}%end item
				\end{itemize}
			}
			
			\subsection{Methods}{
				\vskip -2em
				\begin{itemize}
					\item{ 
						\index{IOCommand!execute(String)}
						\hypertarget{studyplanning.controller.commands.IOCommand.execute(java.lang.String)}{{\bf  execute}\\}
						\begin{lstlisting}[frame=none]
public abstract Response execute(String input)
						\end{lstlisting} %end signature
						\begin{itemize}
							\item{
								{\bf  Description }
								Loads a Workflow or creates a new one, if there was no Workflow found with the given UUID.
							}
							\item{
								{\bf  Parameters}
								\begin{itemize}
									\item{\texttt{input} -- A message generated by the View.}
								\end{itemize}
							}%end item
							\item{
								{\bf  Returns} 
								-- A Response containing the Workflow with the given UUID or containing an empty Workflow, if the given UUID is not yet associated with a Workflow. 
							}%end item
						\end{itemize}
					}%end item
				\end{itemize}
			}
							
			\subsection{Members inherited from class Command }{
				\texttt{studyplanning.controller.commands.Command} {\small 
				\refdefined{studyplanning.controller.commands.Command}}
				{
					\small 
					\vskip -2em
					\begin{itemize}
						\item{
							\vskip -1.5ex 
							\texttt{public abstract Response {\bf  execute}(\texttt{java.lang.String} {\bf  input})}%end signature
						}%end item
					\end{itemize}
				}
			}
		
		\section{\label{studyplanning.controller.commands.RemoveModuleCommand}Class \index{RemoveModuleCommand} RemoveModuleCommand}{
			\hypertarget{studyplanning.controller.commands.RemoveModuleCommand}{}\vskip .1in 
			Removes Modules from a Workflow\vskip .1in 
			
			\subsection{Declaration}{
				\begin{lstlisting}[frame=none]
public class RemoveModuleCommand
  extends studyplanning.controller.commands.Command\end{lstlisting}
			}
			
			\subsection{Constructor summary}{
				\begin{verse}
					\hyperlink{studyplanning.controller.commands.RemoveModuleCommand()}{{\bf RemoveModuleCommand()}} Creates a new RemoveModuleCommand instance\\
				\end{verse}
			}
			
			\subsection{Method summary}{
				\begin{verse}
					\hyperlink{studyplanning.controller.commands.RemoveModuleCommand.execute(java.lang.String)}{{\bf execute(String)}} \\
				\end{verse}
			}
			
			\subsection{Constructors}{
				\vskip -2em
				\begin{itemize}
					\item{ 
						\index{RemoveModuleCommand!RemoveModuleCommand()}
						\hypertarget{studyplanning.controller.commands.RemoveModuleCommand()}{{\bf  RemoveModuleCommand}\\}
						\begin{lstlisting}[frame=none]
public RemoveModuleCommand()
						\end{lstlisting} %end signature
						\begin{itemize}
							\item{
								{\bf  Description}
								Creates a new instance of the RemoveModuleCommand.
							}
						\end{itemize}
					}%end item
				\end{itemize}
			}
			
			\subsection{Methods}{
				\vskip -2em
				\begin{itemize}
					\item{ 
						\index{RemoveModuleCommand!execute(String)}
						\hypertarget{studyplanning.controller.commands.RemoveModuleCommand.execute(java.lang.String)}{{\bf  execute}\\}
						\begin{lstlisting}[frame=none]
public abstract Response execute(String input)
						\end{lstlisting} %end signature
						\begin{itemize}
							\item{
								{\bf  Description}
								Removes a Module from a Workflow.
							}
							\item{
								{\bf  Parameters}
								\begin{itemize}
									\item{\texttt{input} -- A message generated by the View.}
								\end{itemize}
							}%end item
							\item{
								{\bf  Returns}
								-- A Response that contains the changed Workflow, and information about the performed action 
							}%end item
						\end{itemize}
					}%end item
				\end{itemize}
			}
			
			\subsection{Members inherited from class Command }{
				\texttt{studyplanning.controller.commands.Command} 
				{\small \refdefined{studyplanning.controller.commands.Command}}
				{
					\small 
					\vskip -2em
					\begin{itemize}
						\item{
							\vskip -1.5ex 
							\texttt{public abstract Response {\bf  execute}(\texttt{java.lang.String} {\bf  input})}%end signature
						}%end item
					\end{itemize}
				}
			}
		
		\section{\label{studyplanning.controller.commands.VerifyCommand}Class \index{VerifyCommand} VerifyCommand}{
			\hypertarget{studyplanning.controller.commands.VerifyCommand}{}\vskip .1in 
			Checks, whether the Workflow complies with the module manual.\vskip .1in 
			
			\subsection{Declaration}{
			
				\begin{lstlisting}[frame=none]
public class VerifyCommand
  extends studyplanning.controller.commands.Command
				\end{lstlisting}
			}	
			
			\subsection{Constructor summary}{
				\begin{verse}
					\hyperlink{studyplanning.controller.commands.VerifyCommand()}{{\bf VerifyCommand()}} Creates a new VerifyCommand instance\\
				\end{verse}
			}
			
			\subsection{Method summary}{
				\begin{verse}
					\hyperlink{studyplanning.controller.commands.VerifyCommand.execute(java.lang.String)}{{\bf execute(String)}} Verifies whether the Workflow is correct.\\
				\end{verse}
			}
			
			\subsection{Constructors}{
				\vskip -2em
				\begin{itemize}
					\item{ 
						\index{VerifyCommand!VerifyCommand()}
						\hypertarget{studyplanning.controller.commands.VerifyCommand()}{{\bf  VerifyCommand}\\}
						\begin{lstlisting}[frame=none]
public VerifyCommand()
						\end{lstlisting} %end signature
						\begin{itemize}
							\item{
								{\bf  Description}
								Creates a new instance of the VerifyCommand.
							}
						\end{itemize}
					}%end item
				\end{itemize}
			}
			
			\subsection{Methods}{
				\vskip -2em
				\begin{itemize}
					\item{ 
						\index{VerifyCommand!execute(String)}
						\hypertarget{studyplanning.controller.commands.VerifyCommand.execute(java.lang.String)}{{\bf  execute}\\}
						\begin{lstlisting}[frame=none]
public Response execute(String input)
						\end{lstlisting} %end signature
						\begin{itemize}
							\item{
								{\bf  Description}
								Verifies the Workflow.
							}
							\item{
								{\bf  Parameters}
								\begin{itemize}
									\item{\texttt{input} -- A message generated by the View.}
								\end{itemize}
							}%end item
							\item{
								{\bf  Returns}
								-- A Response that shows if the Workflow is correct. \\
								The Response will contain error messages if the Workflow is not correct 
							}%end item
						\end{itemize}
					}%end item
				\end{itemize}
			}
			
			\subsection{Members inherited from class Command }{
				\texttt{studyplanning.controller.commands.Command} {\small 
				\refdefined{studyplanning.controller.commands.Command}}
				{
					\small 
					\vskip -2em
					\begin{itemize}
						\item{
							\vskip -1.5ex 
							\texttt{public abstract Response {\bf  execute}(\texttt{java.lang.String} {\bf  input})}%end signature
						}%end item
					\end{itemize}
				}
			}
		}
	}
	
	\chapter{Package studyplanning.controller}{
		\label{studyplanning.controller}\hypertarget{studyplanning.controller}{}
		\hskip -.05in
		\hbox to \hsize{\textit{ Package Contents\hfil Page}}
		\vskip .13in
		\hbox{{\bf  Classes}}
		\entityintro{Controller}{studyplanning.controller.Controller}{The Controller class exists to coordinate between Model and View.}
		\entityintro{InputParser}{studyplanning.controller.InputParser}{The InputParser class is used by the Controller to parse the messages sent by the View}
		\entityintro{Response}{studyplanning.controller.Response}{Exists to pass generated and changed Workflows to the View.}
		\vskip .1in
		\vskip .1in
		
		\section{\label{studyplanning.controller.Controller}Class \index{Controller} Controller}{
			\hypertarget{studyplanning.controller.Controller}{}\vskip .1in 
			The Controller class exists to coordinate between Model and View.\vskip .1in
			\subsection{Declaration}{
				\begin{lstlisting}[frame=none]
public class Controller
				\end{lstlisting}
			}
			
			\subsection{Constructor summary}{
				\begin{verse}
					\hyperlink{studyplanning.controller.Controller(studyplanning.view.ViewBuilder)}{{\bf Controller(ViewBuilder)}} Creates a new Controller instance\\
				\end{verse}
			}
			
			\subsection{Method summary}{
				\begin{verse}
					\hyperlink{studyplanning.controller.Controller.getDataSet()}{{\bf getDataSet()}} Retrieves a DataSet containing all the Modules\\
					\hyperlink{studyplanning.controller.Controller.init()}{{\bf init()}} Initiates the server program and loads the data base\\
					\hyperlink{studyplanning.controller.Controller.processRequest(java.lang.String)}{{\bf processRequest(String)}} Processes a View request and returns a Response\\
					\hyperlink{studyplanning.controller.Controller.reload()}{{\bf reload()}} Reloads the database\\
				\end{verse}
			}

			\subsection{Constructors}{
				\vskip -2em
				\begin{itemize}
					\item{ 
						\index{Controller!Controller(ViewBuilder)}
						\hypertarget{studyplanning.controller.Controller(studyplanning.view.ViewBuilder)}{{\bf  Controller}\\}
						\begin{lstlisting}[frame=none]
public Controller(ViewBuilder newView)
						\end{lstlisting} %end signature
						\begin{itemize}
							\item{
								{\bf  Description}
								Creates a new Controller instance with a specified ViewBuilder.
							}
							\item{
								{\bf  Parameters}
								\begin{itemize}
								   \item{\texttt{newView} -- } The Servlet to build an application runnable on a server
								\end{itemize}
							}%end item
						\end{itemize}
					}%end item
				\end{itemize}
			}

			\subsection{Methods}{
				\vskip -2em
				\begin{itemize}
					\item{ 
						\index{Controller!getDataSet()}
						\hypertarget{studyplanning.controller.Controller.getDataSet()}{{\bf  getDataSet}\\}
						\begin{lstlisting}[frame=none]
public DataSet getDataSet()
						\end{lstlisting} %end signature
						\begin{itemize}
							\item{
								{\bf  Description}
								Retrieves a DataSet containing all Modules
							}
							\item{
								{\bf  Returns} 
								-- A DataSet containing every Module
							}%end item
						\end{itemize}
					}
					\item{ 
						\index{Controller!init()}
						\hypertarget{studyplanning.controller.Controller.init()}{{\bf  init}\\}
						\begin{lstlisting}[frame=none]
public void init()
						\end{lstlisting} %end signature
						\begin{itemize}
							\item{
								{\bf  Description}
								Initiates the server program and loads the database
							}
						\end{itemize}
					}%end item
					\item{ 
						\index{Controller!processRequest(String)}
						\hypertarget{studyplanning.controller.Controller.processRequest(java.lang.String)}{{\bf  processRequest}\\}
						\begin{lstlisting}[frame=none]
public Response processRequest(String request)
						\end{lstlisting} %end signature
						\begin{itemize}
							\item{
								{\bf  Description}
								Processes a View request and returns a Response
							}
							\item{
								{\bf  Parameters}
								\begin{itemize}
								   \item{\texttt{request} -- The String generated by the View}
								\end{itemize}
							}%end item
							\item{
								{\bf  Returns}
								-- A Response giving the View its demanded information 
							}%end item
						\end{itemize}
					}%end item
					\item{ 
						\index{Controller!reload()}
						\hypertarget{studyplanning.controller.Controller.reload()}{{\bf  reload}\\}
						\begin{lstlisting}[frame=none]
public void reload()
						\end{lstlisting} %end signature
						\begin{itemize}
							\item{
								{\bf  Description}
								Reloads the database
							}
						\end{itemize}
					}%end item
				\end{itemize}
			}
		}
		
		\section{\label{studyplanning.controller.InputParser}Class \index{InputParser} InputParser}{
			\hypertarget{studyplanning.controller.InputParser}{}\vskip .1in 
			The InputParser class is used to parse the messages generated by the View, and get the specified objects. \\
			All of the View's messages are coded in a string, this class offers methods to decode these messages.\vskip .1in 

			\subsection{Declaration}{
				\begin{lstlisting}[frame=none]
public class InputParser
				\end{lstlisting}
			}
			
			\subsection{Method summary}{
				\begin{verse}
					\hyperlink{studyplanning.controller.InputParser.parseModuleName(java.lang.String)}{{\bf parseModuleName(String)}} Parses the View's input to get the name of a Module \\
					\hyperlink{studyplanning.controller.InputParser.parsePreferences(java.lang.String)}{{\bf parsePreferences(String)}} Parses the Views input and retrieves the Preferences\\
					\hyperlink{studyplanning.controller.InputParser.parseSemester(java.lang.String)}{{\bf parseSemester(String)}} Parses the View's input for the given Semester\\
					\hyperlink{studyplanning.controller.InputParser.parseStudySubject(java.lang.String)}{{\bf parseStudySubject(String)}} Parses the View's input and the StudySubject from the Model\\
					\hyperlink{studyplanning.controller.InputParser.parseUserID(java.lang.String)}{{\bf parseUserID(String)}} Parses a user UUID \\
					\hyperlink{studyplanning.controller.InputParser.parseWorkflow(java.lang.String)}{{\bf parseWorkflow(String)}} Parses the View's input and parses a Workflow from the Model\\
					\hyperlink{studyplanning.controller.InputParser.parseWorkflowID(java.lang.String)}{{\bf parseWorkflowID(String)}} Parses a Workflow's UUID\\
				\end{verse}
			}
			
			\subsection{Methods}{
				\vskip -2em
				\begin{itemize}
					\item{ 
						\index{InputParser!parseModuleName(String)}
						\hypertarget{studyplanning.controller.InputParser.parseModuleName(java.lang.String)}{{\bf  parseModuleName}\\}
						\begin{lstlisting}[frame=none]
public static String parseModuleName(String input)
						\end{lstlisting} %end signature
						\begin{itemize}
							\item{
								{\bf  Description}
								Parses the View's input to get the name of a Module
							}
							\item{
								{\bf  Parameters}
								\begin{itemize}
									\item{\texttt{input} -- The string message generated by the View}
								\end{itemize}
							}%end item
							\item{{\bf  Returns}
								 -- The Modules name as a string 
							}%end item
						\end{itemize}
					}%end item
					\item{ 
						\index{InputParser!parsePreferences(String)}
						\hypertarget{studyplanning.controller.InputParser.parsePreferences(java.lang.String)}{{\bf  parsePreferences}\\}
						\begin{lstlisting}[frame=none]
public static Preferences parsePreferences(jString input)
						\end{lstlisting} %end signature
						\begin{itemize}
							\item{
								{\bf  Description}
								Parses the views input and retrieves the Preferences
							}
							\item{
								{\bf  Parameters}
								\begin{itemize}
									\item{\texttt{input} -- The message generated by the View}
								\end{itemize}
							}%end item
							\item{{\bf  Returns} 
								-- 	The parsed users Preferences 
							}%end item
						\end{itemize}
					}%end item
					\item{ 
						\index{InputParser!parseSemester(String)}
						\hypertarget{studyplanning.controller.InputParser.parseSemester(java.lang.String)}{{\bf  parseSemester}\\}
						\begin{lstlisting}[frame=none]
public static int parseSemester(String input)
						\end{lstlisting} %end signature
						\begin{itemize}
							\item{
								{\bf  Description}
								Parses the View's input for the given Semester
							}
							\item{
								{\bf  Parameters}
								\begin{itemize}
									\item{\texttt{input} -- The message generated by the View}
								\end{itemize}
							}%end item
							\item{{\bf  Returns} 
								-- The parsed semester number 
							}%end item
						\end{itemize}
					}%end item
					\item{ 
						\index{InputParser!parseStudySubject(String)}
						\hypertarget{studyplanning.controller.InputParser.parseStudySubject(java.lang.String)}{{\bf  parseStudySubject}\\}
						\begin{lstlisting}[frame=none]
public static StudySubject parseStudySubject(String input)
						\end{lstlisting} %end signature
						\begin{itemize}
							\item{
								{\bf  Description}
								Parses the View's input and the StudySubject from the Model
							}
							\item{
								{\bf  Parameters}
								\begin{itemize}
									\item{\texttt{input} -- The message generated by the View}
								\end{itemize}
							}%end item
							\item{{\bf  Returns}
								-- The parsed StudySubject
							}%end item
						\end{itemize}
					}%end item
					\item{ 
						\index{InputParser!parseUserID(String)}
						\hypertarget{studyplanning.controller.InputParser.parseUserID(java.lang.String)}{{\bf  parseUserID}\\}
						\begin{lstlisting}[frame=none]
public static UUID parseUserID(String input)
						\end{lstlisting} %end signature
						\begin{itemize}
							\item{
								{\bf Description}
								Parses a user id
							}
							\item{
								{\bfseries Parameters}
								\begin{itemize}
									\item{\texttt{input} The message generated by the View}
								\end{itemize}
							}
							\item{
								{\bf Returns}
								-- The parsed id
								}
						\end{itemize}
					}%end item
					\item{ 
						\index{InputParser!parseWorkflow(String)}
						\hypertarget{studyplanning.controller.InputParser.parseWorkflow(java.lang.String)}{{\bf  parseWorkflow}\\}
						\begin{lstlisting}[frame=none]
public static Workflow parseWorkflow(String input)
						\end{lstlisting} %end signature
						\begin{itemize}
							\item{
								{\bf  Description}		
								Parses the View's input and parses a Workflow from the Model
							}
							\item{
								{\bf  Parameters}
								\begin{itemize}
									\item{
										\texttt{input} -- The message generated by the View}
								\end{itemize}
							}%end item
							\item{{\bf  Returns} -- 
								A parsed Workflow	
							}%end item
						\end{itemize}
					}%end item
					\item{ 
						\index{InputParser!parseWorkflowID(String)}
						\hypertarget{studyplanning.controller.InputParser.parseWorkflowID(java.lang.String)}{{\bf  parseWorkflowID}\\}
						\begin{lstlisting}[frame=none]
public static UUID parseWorkflowID(String input)
						\end{lstlisting} %end signature
						\begin{itemize}
							\item{
								{\bf  Description}
								Parses a Workflow's id
							}
							\item{
								{\bf  Parameters}
								\begin{itemize}
									\item{
										\texttt{input} -- The message generated by the View}
								\end{itemize}
							}%end item
							\item{{\bf  Returns} -- 
								The parsed Workflow's id
							}%end item
						\end{itemize}
					}%end item
				\end{itemize}
			}
		}
		
		\section{\label{studyplanning.controller.Response}Class \index{Response} Response}{
			\hypertarget{studyplanning.controller.Response}{}\vskip .1in 
			Passes generated and changed Workflows to the View. \\
			Allows the Controller to communicate with the View. \\
			Each Response consits of a Workflow, a Collection of Mistakes, and an HTML-statuscode\\
			\vskip .1in 
			
			\subsection{Declaration}{
				\begin{lstlisting}[frame=none]
public class Response
				\end{lstlisting}
			}
			
			\subsection{Constructor summary}{
				\begin{verse}
					\hyperlink{studyplanning.controller.Response(studyplanning.model.workflow.Workflow, java.util.Collection)}{{\bf Response(Workflow, Collection)}} Creates a new Response instance with the default "correct" an HTML-Code 200\\
					\hyperlink{studyplanning.controller.Response(studyplanning.model.workflow.Workflow, java.util.Collection, int)}{{\bf Response(Workflow, Collection, int)}} Creates a new Response instance\\
				\end{verse}
			}
			
			\subsection{Method summary}{
				\begin{verse}
					\hyperlink{studyplanning.controller.Response.getHtmlCode()}{{\bf getHtmlCode()}} Getter method for the an HTML-Code.\\
					\hyperlink{studyplanning.controller.Response.getMistakes()}{{\bf getMistakes()}} Returns the Collection of Mistakes contained, if this is a Response for a failed generation or verification task, an empty Collection otherwise.\\
					\hyperlink{studyplanning.controller.Response.getWorkflow()}{{\bf getWorkflow()}} Returns the current Workflow.\\
					\hyperlink{studyplanning.controller.Response.isCorrect()}{{\bf isCorrect()}} Checks whether there are any Mistakes contained in the Response.\\
				\end{verse}
			}

			\subsection{Constructors}{
				\vskip -2em
				\begin{itemize}
					\item{ 
						\index{Response!Response(Workflow, Collection)}
						\hypertarget{studyplanning.controller.Response(studyplanning.model.workflow.Workflow, java.util.Collection)}{{\bf  Response}\\}
						\begin{lstlisting}[frame=none]
public Response(Workflow workflow, Collection mistakes)
						\end{lstlisting} %end signature
						\begin{itemize}
							\item{
								{\bf  Description}
								Creates a new Response Instance with the default "correct" an HTML-Code 200
							}
							\item{
								{\bf  Parameters}
								\begin{itemize}
									\item{\texttt{workflow} -- A Workflow that needs to be passed on to the View}
									\item{\texttt{mistakes} -- A Collection containing information about relevant errors}
								\end{itemize}
							}%end item
						\end{itemize}
					}%end item
					\item{ 
						\index{Response!Response(Workflow, Collection, int)}
						\hypertarget{studyplanning.controller.Response(studyplanning.model.workflow.Workflow, java.util.Collection, int)}{{\bf  Response}\\}
						\begin{lstlisting}[frame=none]
public Response(Workflow workflow, Collection<Mistakes> mistakes, int htmlCode)
						\end{lstlisting} %end signature
						\begin{itemize}
							\item{
								{\bf  Description}
								Creates a new Response instance
							}
							\item{
								{\bf  Parameters}
								\begin{itemize}
									\item{\texttt{workflow} -- A Workflow, that needs to be passed on to the View}
									\item{\texttt{mistakes} -- A Collection containing information about relevant errors}
									\item{\texttt{htmlCode} -- An HTML-Code that can be used to pass on different information to the View}
								\end{itemize}
							}%end item
						\end{itemize}
					}%end item
				\end{itemize}
			}
		
			\subsection{Methods}{
				\vskip -2em
				\begin{itemize}
					\item{ 
						\index{Response!getHtmlCode()}
						\hypertarget{studyplanning.controller.Response.getHtmlCode()}{{\bf  getHtmlCode}\\}
						\begin{lstlisting}[frame=none]
public int getHtmlCode()
						\end{lstlisting} %end signature
						\begin{itemize}
							\item{
								{\bf  Description}
								Getter method for the HTML-Code
							}
							\item{
								{\bf  Returns} 
								-- The Response HTML-Code 
							}%end item
						\end{itemize}
					}
					\item{ 
						\index{Response!getMistakes()}
						\hypertarget{studyplanning.controller.Response.getMistakes()}{{\bf  getMistakes}\\}
						\begin{lstlisting}[frame=none]
public Collection<Mistake> getMistakes()
						\end{lstlisting} %end signature
						\begin{itemize}
							\item{
								{\bf Description}
								Returns the Collection of Mistakes contained in the Workflow, if this is a Response for a failed generation or verification task, an empty Collection otherwise. Empty if Workflow is correct.
							}
							\item{
								{\bf  Returns} 
								-- A Collection of Mistakes.
							}%end item
						\end{itemize}
					}%end item
					\item{ 
						\index{Response!getWorkflow()}
						\hypertarget{studyplanning.controller.Response.getWorkflow()}{{\bf  getWorkflow}\\}
						\begin{lstlisting}[frame=none]
public Workflow getWorkflow()
						\end{lstlisting} %end signature
						\begin{itemize}
							\item {
								{\bf Description}
								Returns the current Workflow.
							}
							\item {
								{\bf Returns}
								-- The current Workflow.
							}
						\end{itemize}
					}%end item
					\item{ 
						\index{Response!isCorrect()}
						\hypertarget{studyplanning.controller.Response.isCorrect()}{{\bf  isCorrect}\\}
						\begin{lstlisting}[frame=none]
public boolean isCorrect()
						\end{lstlisting} %end signature
						\begin{itemize}
							\item{
								{\bf  Description}
								Checks if there are any Mistakes contained in the Response.
							}
							\item{
								{\bf  Returns} 
								-- True, if there are not any Mistakes 
							}%end item
						\end{itemize}
					}%end item
				\end{itemize}
			}
		}
	
\part{Model}{
	\begin{figure}[ht]
		\centering
		\includegraphics[width=\textwidth, angle=90]{res/ClassDiagram/modelMarked}
		\caption{Model}
	\end{figure}
	
	\chapter{Package studyplanning.model}{
		\label{studyplanning.model}\hypertarget{studyplanning.model}{}
		\hskip -.05in
		\hbox to \hsize{\textit{ Package Contents\hfil Page}}
		\vskip .13in
		\hbox{{\bf  Classes}}
		\entityintro{DataIO}{studyplanning.model.DataIO}{Saves and load user data.}
		\entityintro{WorkflowOperations}{studyplanning.model.WorkflowOperations}{Facade for all workflow operations.}
		\vskip .1in
		\vskip .1in
		
		\section{\label{studyplanning.model.DataIO}Class\index{DataIO} DataIO}{
			\hypertarget{studyplanning.model.DataIO}{}\vskip .1in 
			Saves and loads user data. Loads the database.\vskip .1in 
			
			\subsection{Declaration}{
			\begin{lstlisting}[frame=none]
public final class DataIOt
			\end{lstlisting}
			}
			
			\subsection{Constructor summary}{
				\begin{verse}
					\hyperlink{studyplanning.model.DataIO()}{{\bf DataIO()}} \\
				\end{verse}
			}

			\subsection{Method summary}{
				\begin{verse}
					\hyperlink{studyplanning.model.DataIO.addWorkflowToUser(java.util.UUID, java.util.UUID)}{{\bf addWorkflowToUser(UUID, UUID)}} Associates another Workflow ID with the given user.\\
					\hyperlink{studyplanning.model.DataIO.getDataForStudySubject(java.lang.String)}{{\bf getDataForStudySubject(String)}} Returns the StudySubject for the given name.\\
					\hyperlink{studyplanning.model.DataIO.getNextFreeUserID()}{{\bf getNextFreeUserID()}} Generates and returns a new UUID for use with a user.\\
					\hyperlink{studyplanning.model.DataIO.getNextFreeWorkflowID()}{{\bf getNextFreeWorkflowID()}} Generates and returns a new UUID for use with a Workflow.\\
					\hyperlink{studyplanning.model.DataIO.getWorkflowsForUser(java.util.UUID)}{{\bf getWorkflowsForUser(UUID)}} Returns all Workflow IDs associated with the given user.\\
					\hyperlink{studyplanning.model.DataIO.loadDatabase()}{{\bf loadDatabase()}} Loads the database.\\
					\hyperlink{studyplanning.model.DataIO.loadWorkflow(java.util.UUID)}{{\bf loadWorkflow(UUID)}} Loads the Workflow for the given UUID.\\
					\hyperlink{studyplanning.model.DataIO.saveWorkflow(studyplanning.model.workflow.Workflow, java.util.UUID)}{{\bf saveWorkflow(Workflow, UUID)}} Saves the Workflow with the given id, so it can be fetched with \texttt{\small \hyperlink{studyplanning.model.DataIO.loadWorkflow(java.util.UUID)}{loadWorkflow(UUID)}}{\small 
					\refdefined{studyplanning.model.DataIO.loadWorkflow(java.util.UUID)}}\\
				\end{verse}
			}
			
			\subsection{Constructors}{
				\vskip -2em
				\begin{itemize}
					\item{ 
						\index{DataIO!DataIO()}
						\hypertarget{studyplanning.model.DataIO()}{{\bf  DataIO}\\}
						\begin{lstlisting}[frame=none]
public DataIO()
						\end{lstlisting} %end signature
						\begin{itemize}
							\item{
								{\bf Description}
								Creates a new instance and loads the database with all StudySubjects.
							}
						\end{itemize}
					}%end item
				\end{itemize}
			}
			
			\subsection{Methods}{
				\vskip -2em
				\begin{itemize}
					\item{ 
						\index{DataIO!addWorkflowToUser(UUID, UUID)}
						\hypertarget{studyplanning.model.DataIO.addWorkflowToUser(java.util.UUID, java.util.UUID)}{{\bf  addWorkflowToUser}\\}
						\begin{lstlisting}[frame=none]
public static void addWorkflowToUser(UUID userID, UUID workflowID)
						\end{lstlisting} %end signature
						\begin{itemize}
							\item{
								{\bf  Description}
								Associates another Workflow ID with the given user.
							}
							\item{
								{\bf  Parameters}
								\begin{itemize}
									\item{\texttt{userID} -- The user the Workflow ID should be added to.}
									\item{\texttt{workflowID} -- The Workflow ID to be added to the user.}
								\end{itemize}
							}%end item
						\end{itemize}
					}%end item
					\item{ 
						\index{DataIO!getDataForStudySubject(String)}
						\hypertarget{studyplanning.model.DataIO.getDataForStudySubject(java.lang.String)}{{\bf  getDataForStudySubject}\\}
						\begin{lstlisting}[frame=none]
public StudySubject getDataForStudySubject(String subject)
						\end{lstlisting} %end signature
						\begin{itemize}
							\item{
								{\bf  Description}
								Returns the StudySubject for the given name.
							}
							\item{
								{\bf  Parameters}
								\begin{itemize}
									\item{\texttt{subject} -- The name of the study subject.}
								\end{itemize}
							}%end item
							\item{
								{\bf  Returns}
								 -- The requested StudySubject. 
							}%end item
						\end{itemize}
					}%end item
					\item{ 
						\index{DataIO!getNextFreeUserID()}
						\hypertarget{studyplanning.model.DataIO.getNextFreeUserID()}{{\bf  getNextFreeUserID}\\}
						\begin{lstlisting}[frame=none]
public UUID getNextFreeUserID()
						\end{lstlisting} %end signature
						\begin{itemize}
							\item{
								{\bf  Description}
								Generates and returns a new UUID for use with a user.
							}
							\item{
								{\bf  Returns} 
								-- A new user UUID. 
							}%end item
						\end{itemize}
					}%end item
					\item{ 
						\index{DataIO!getNextFreeWorkflowID()}
						\hypertarget{studyplanning.model.DataIO.getNextFreeWorkflowID()}{{\bf  getNextFreeWorkflowID}\\}
						\begin{lstlisting}[frame=none]
public UUID getNextFreeWorkflowID()
						\end{lstlisting} %end signature
						\begin{itemize}
							\item{
								{\bf  Description}
								Generates and returns a new UUID for use with a Workflow.
							}
							\item{
								{\bf  Returns} 
								-- A new Workflow UUID. 
							}%end item
						\end{itemize}
					}%end item
					\item{ 
						\index{DataIO!getWorkflowsForUser(UUID)}
						\hypertarget{studyplanning.model.DataIO.getWorkflowsForUser(java.util.UUID)}{{\bf  getWorkflowsForUser}\\}
						\begin{lstlisting}[frame=none]
public Collection<UUID> getWorkflowsForUser(UUID userID)
						\end{lstlisting} %end signature
						\begin{itemize}
							\item{
								{\bf  Description}
								Returns all Workflow IDs associated with the given user.
							}
							\item{
								{\bf  Parameters}
								\begin{itemize}
									\item{\texttt{userID} -- The user ID to get the Workflow ID's from.}
								\end{itemize}
							}%end item
							\item{
								{\bf  Returns} 
								-- A Collection of all Workflow IDs associated with the given user. 
							}%end item
						\end{itemize}
					}%end item
					\item{ 
						\index{DataIO!loadDatabase()}
						\hypertarget{studyplanning.model.DataIO.loadDatabase()}{{\bf  loadDatabase}\\}
						\begin{lstlisting}[frame=none]
public synchronized void loadDatabase()
						\end{lstlisting} %end signature
						\begin{itemize}
							\item{
								{\bf  Description}
								Loads the database.
							}
						\end{itemize}
					}%end item
					\item{ 
						\index{DataIO!loadWorkflow(UUID)}
						\hypertarget{studyplanning.model.DataIO.loadWorkflow(java.util.UUID)}{{\bf  loadWorkflow}\\}
						\begin{lstlisting}[frame=none]
public Workflow loadWorkflow(UUID id)
						\end{lstlisting} %end signature
						\begin{itemize}
							\item{
								{\bf  Description}
								Loads the Workflow for the given UUID.
							}
							\item{
								{\bf  Parameters}
								\begin{itemize}
									\item{\texttt{id} -- The UUID to load the Workflow from.}
								\end{itemize}
							}%end item
							\item{
								{\bf  Returns} 
								-- The Workflow previously saved for the given UUID. 
							}%end item
						\end{itemize}
					}%end item
					\item{ 
						\index{DataIO!saveWorkflow(Workflow, UUID)}
						\hypertarget{studyplanning.model.DataIO.saveWorkflow(studyplanning.model.workflow.Workflow, java.util.UUID)}{{\bf  saveWorkflow}\\}
						\begin{lstlisting}[frame=none]
public boolean saveWorkflow(Workflow workflow, UUID id)
						\end{lstlisting} %end signature
						\begin{itemize}
							\item{
								{\bf  Description}
									Saves the Workflow with the given id, so it can be fetched with \texttt{\small \hyperlink{studyplanning.model.DataIO.loadWorkflow(java.util.UUID)}{loadWorkflow(UUID)}}{\small 
									\refdefined{studyplanning.model.DataIO.loadWorkflow(java.util.UUID)}}
							}
							\item{
								{\bf  Parameters}
								\begin{itemize}
									\item{\texttt{workflow} -- The Workflow to save.}
									\item{\texttt{id} -- The Workflow's identifier.}
								\end{itemize}
							}%end item
							\item{
								{\bf  Returns} 
								-- True, if saving succeeded 
							}%end item
						\end{itemize}
					}%end item
				\end{itemize}
			}
		}
		
		\section{\label{studyplanning.model.WorkflowOperations}Class \index{WorkflowOperations} WorkflowOperations}{
			\hypertarget{studyplanning.model.WorkflowOperations}{}\vskip .1in 
			Facade for all workflow operations.\vskip .1in 

			\subsection{Declaration}{
				\begin{lstlisting}[frame=none]
public class WorkflowOperations
				\end{lstlisting}
			
			\subsection{Constructor summary}{
				\begin{verse}
					\hyperlink{studyplanning.model.WorkflowOperations()}{{\bf WorkflowOperations()}} \\
				\end{verse}
			}

			\subsection{Method summary}{
				\begin{verse}
					\hyperlink{studyplanning.model.WorkflowOperations.addModule(studyplanning.model.workflow.Workflow, java.lang.String, int)}{{\bf addModule(Workflow, String, int)}} Adds a Module to the Workflow.\\
					\hyperlink{studyplanning.model.WorkflowOperations.generateWorkflow(studyplanning.model.workflow.Workflow, studyplanning.model.workflow.StudySubject, studyplanning.model.workflow.generation.Preferences)}{{\bf generateWorkflow(Workflow, StudySubject, Preferences)}} Use this method to generate a Workflow of the given preferences.\\
					\hyperlink{studyplanning.model.WorkflowOperations.removeModule(studyplanning.model.workflow.Workflow, java.lang.String)}{{\bf removeModule(Workflow, String)}} Removes a Module from the workflow.\\
					\hyperlink{studyplanning.model.WorkflowOperations.removeModule(studyplanning.model.workflow.Workflow, java.lang.String, int)}{{\bf removeModule(Workflow, String, int)}} Removes a Module from the given Semester of the Workflow.\\
					\hyperlink{studyplanning.model.WorkflowOperations.verifyWorkflow(studyplanning.model.workflow.Workflow, studyplanning.model.workflow.StudySubject)}{{\bf verifyWorkflow(Workflow, StudySubject)}} Use this method to verify if the Workflow is correct. \\
				\end{verse}
			}
			
			\subsection{Constructors}{
				\vskip -2em
				\begin{itemize}
					\item{ 
						\index{WorkflowOperations!WorkflowOperations()}
						\hypertarget{studyplanning.model.WorkflowOperations()}{{\bf  WorkflowOperations}\\}
						\begin{lstlisting}[frame=none]
public WorkflowOperations()
						\end{lstlisting} %end signature
						\begin{itemize}
							\item{
								{\bf Description}
							This class works as a proxy object, making it easier to control access.
							}
						\end{itemize}
					}%end item
				\end{itemize}
			}

			\subsection{Methods}{
				\vskip -2em
				\begin{itemize}
					\item{ 
						\index{WorkflowOperations!addModule(Workflow, String, int)}
						\hypertarget{studyplanning.model.WorkflowOperations.addModule(studyplanning.model.workflow.Workflow, java.lang.String, int)}{{\bf  addModule}\\}
						\begin{lstlisting}[frame=none]
public boolean addModule(Workflow workflow, String moduleName, int semester)
						\end{lstlisting} %end signature
						\begin{itemize}
							\item{
								{\bf  Description}
								Adds a Module to the Workflow.
							}
							\item{
								{\bf  Parameters}
								\begin{itemize}
									\item{\texttt{workflow} -- Workflow to be edited}
									\item{\texttt{moduleName} -- Name and identifier of the Module}
									\item{\texttt{semester} -- The Semester the Module should be added to}
								\end{itemize}
							}%end item
							\item{
								{\bf  Returns} 
								-- True, if adding the Module succeeded 
							}%end item
						\end{itemize}
					}%end item
					\item{ 
						\index{WorkflowOperations!generateWorkflow(Workflow, StudySubject, Preferences)}
						\hypertarget{studyplanning.model.WorkflowOperations.generateWorkflow(studyplanning.model.workflow.Workflow, studyplanning.model.workflow.StudySubject, studyplanning.model.worklfow.generation.Preferences)}{{\bf  generateWorkflow}\\}
						\begin{lstlisting}[frame=none]
public Collection<Mistake> generateWorkflow(Workflow workflow, StudySubject studySubject, Preferences preferences)
						\end{lstlisting} %end signature
						\begin{itemize}
							\item{
								{\bf  Description}
								Use this method to generate a Workflow of the given Preferences. This changes the Workflow.
							}
							\item{
								{\bf  Parameters}
								\begin{itemize}
									\item{\texttt{workflow} -- The Workflow generated by the user or the system.}
									\item{\texttt{studySubject} -- The StudySubject related to the Workflow}
									\item{\texttt{preferences} -- All user preferences. Module related preferences have to be added in this object}
								\end{itemize}
							}%end item
							\item{
								{\bf  Returns} 
								-- A Collection of Mistakes occurred during generation. 
							}%end item
						\end{itemize}
					}%end item
					\item{ 
						\index{WorkflowOperations!removeModule(Workflow, String)}
						\hypertarget{studyplanning.model.WorkflowOperations.removeModule(studyplanning.model.workflow.Workflow, java.lang.String)}{{\bf  removeModule}\\}
						\begin{lstlisting}[frame=none]
public Module removeModule(Workflow workflow, String moduleName)
						\end{lstlisting} %end signature
						\begin{itemize}
							\item{
								{\bf  Description}
								Removes a Module from the Workflow.
							}
							\item{
								{\bf  Parameters}
								\begin{itemize}
									\item{\texttt{workflow} -- Workflow to be edited}
									\item{\texttt{moduleName} -- Name and identifier of the Module}
								\end{itemize}
							}%end item
							\item{
								{\bf  Returns} 
								-- The Module which has been removed. 
							}%end item
						\end{itemize}
					}%end item
					\item{ 
						\index{WorkflowOperations!removeModule(Workflow, String, int)}
						\hypertarget{studyplanning.model.WorkflowOperations.removeModule(studyplanning.model.workflow.Workflow, java.lang.String, int)}{{\bf  removeModule}\\}
						\begin{lstlisting}[frame=none]
public Module removeModule(Workflow workflow, String moduleName, int semester)
						\end{lstlisting} %end signature
						\begin{itemize}
							\item{
								{\bf  Description}
								Removes a Module from the given Semester of the Workflow.
							}
							\item{
								{\bf  Parameters}
								\begin{itemize}
									\item{\texttt{workflow} -- Workflow to be edited}
									\item{\texttt{moduleName} --Identifier of the Module}
									\item{\texttt{semester} -- The Module's Semester}
								\end{itemize}
							}%end item
							\item{
								{\bf  Returns}
								 -- The removed Module, null if not found 
							}%end item
						\end{itemize}
					}%end item
					\item{ 
						\index{WorkflowOperations!verifyWorkflow(Workflow, StudySubject)}
						\hypertarget{studyplanning.model.WorkflowOperations.verifyWorkflow(studyplanning.model.workflow.Workflow, studyplanning.model.workflow.StudySubject)}{{\bf  verifyWorkflow}\\}
						\begin{lstlisting}[frame=none]
public Collection<Mistake> verifyWorkflow(Workflow workflow, StudySubject studySubject)
						\end{lstlisting} %end signature
						\begin{itemize}
							\item{
								{\bf  Description}
								Use this method to verify if the Workflow is correct.
							}
							\item{
								{\bf  Parameters}
								\begin{itemize}
									\item{\texttt{workflow} -- The Workflow generated by the user or the system.}
									\item{\texttt{studySubject} -- The related StudySubject the user studies}
								\end{itemize}
							}%end item
							\item{
								{\bf  Returns} 
								-- A Collection containing all module manual violations. 
							}%end item
						\end{itemize}
					}%end item
				\end{itemize}
			}
		}
	}

	\chapter{Package studyplanning.model.mistake}{
		\label{studyplanning.model.mistake}\hypertarget{studyplanning.model.mistake}{}
		\hskip -.05in
		\hbox to \hsize{\textit{ Package Contents\hfil Page}}
		\vskip .13in
		\hbox{{\bf  Classes}}
		\entityintro{Mistake}{studyplanning.model.mistake.Mistake}{Errors made by the user are listed here.}
		\entityintro{MistakeConstraint}{studyplanning.model.mistake.MistakeConstraint}{A subclass of the \texttt{\small \hyperlink{studyplanning.model.mistake.MistakeModule}{MistakeModule}}{\small 
		\refdefined{studyplanning.model.mistake.MistakeModule}} class.}
		\entityintro{MistakeModule}{studyplanning.model.mistake.MistakeModule}{A subclass of the \texttt{\small \hyperlink{studyplanning.model.mistake.Mistake}{Mistake}}{\small 
		\refdefined{studyplanning.model.mistake.Mistake}} class which is used for mistakes connected to a certain Module.}
		\entityintro{Mistakes}{studyplanning.model.mistake.Mistakes}{Class with factory methods for quick access to new \texttt{\small \hyperlink{studyplanning.model.mistake.Mistake}{Mistake}}{\small 
		\refdefined{studyplanning.model.mistake.Mistake}}s.}
		\vskip .1in
		\vskip .1in
		
		\section{\label{studyplanning.model.mistake.Mistake}Class \index{Mistake} Mistake}{
			\hypertarget{studyplanning.model.mistake.Mistake}{}\vskip .1in 
			Errors made by the user are listed here. Objects of this class will be generated by the verify method of the Model class.\vskip .1in 
			
			\subsection{Declaration}{
				\begin{lstlisting}[frame=none]
public class Mistake
				\end{lstlisting}
			}
			
			\subsection{All known subclasses}{
				{MistakeModule\small{\refdefined{studyplanning.model.mistake.MistakeModule}}, MistakeConstraint\small{\refdefined{studyplanning.model.mistake.MistakeConstraint}}}
			}
			
			\subsection{Constructor summary}{
				\begin{verse}
					\hyperlink{studyplanning.model.mistake.Mistake(java.lang.String)}{{\bf Mistake(String)}} Creates a new Mistake object with a given message.\\
				\end{verse}
			}

			\subsection{Method summary}{
				\begin{verse}
					\hyperlink{studyplanning.model.mistake.Mistake.getLocalizationKey()}{{\bf getLocalizationKey()}} Returns the localization key for this Mistake.\\
					\hyperlink{studyplanning.model.mistake.Mistake.getLocalizedMessage(studyplanning.view.Locale)}{{\bf getLocalizedMessage(Locale)}} Returns a readable String for a Mistake.\\
				\end{verse}
			}

			\subsection{Constructors}{
				\vskip -2em
				\begin{itemize}
					\item{ 
						\index{Mistake!Mistake(String)}
						\hypertarget{studyplanning.model.mistake.Mistake(java.lang.String)}{{\bf  Mistake}\\}
						\begin{lstlisting}[frame=none]
public Mistake(String localizationKey)
						\end{lstlisting} %end signature
						\begin{itemize}
							\item{
								{\bf  Description}
								Creates a new Mistake object with a given message.
							}
							\item{
								{\bf  Parameters}
								\begin{itemize}
									\item{\texttt{localizationKey} -- A string containing they key word for a Mistake}
								\end{itemize}
							}%end item
						\end{itemize}
					}%end item
				\end{itemize}
			}
			
			\subsection{Methods}{
				\vskip -2em
				\begin{itemize}
					\item{ 
						\index{Mistake!getLocalizationKey()}
						\hypertarget{studyplanning.model.mistake.Mistake.getLocalizationKey()}{{\bf  getLocalizationKey}\\}
						\begin{lstlisting}[frame=none]
public String getLocalizationKey()
						\end{lstlisting} %end signature
						\begin{itemize}
							\item{
								{\bf  Description}
								Returns the localization key for this Mistake. May not be null.
								Returns the localization key for this mistake.
							}
							\item{
								{\bf  Returns} 
								-- The localization key for this Mistake. 
							}%end item
						\end{itemize}
					}%end item
					\item{ 
						\index{Mistake!getLocalizedMessage(Locale)}
						\hypertarget{studyplanning.model.mistake.Mistake.getLocalizedMessage(studyplanning.view.Locale)}{{\bf  getLocalizedMessage}\\}
						\begin{lstlisting}[frame=none]
public String getLocalizedMessage(Locale language)
						\end{lstlisting} %end signature
						\begin{itemize}
							\item{
								{\bf  Description}
								Returns a readable String for a Mistake.
							}
							\item{
								{\bf  Parameters}
								\begin{itemize}
									\item{\texttt{language} -- The target Language}
								\end{itemize}
							}%end item
							\item{
								{\bf  Returns} 
								-- The translated String 
							}%end item
						\end{itemize}
					}%end item
				\end{itemize}
			}
		}
		
		\section{\label{studyplanning.model.mistake.MistakeConstraint}Class \index{MistakeConstraint} MistakeConstraint}{
			\hypertarget{studyplanning.model.mistake.MistakeConstraint}{}\vskip .1in 
			A subclass of the \texttt{\small \hyperlink{studyplanning.model.mistake.MistakeModule}{MistakeModule}}{\small 
			\refdefined{studyplanning.model.mistake.MistakeModule}} class. This class contains the affected Constraint and its Modules.\vskip .1in 
			\refdefined{studyplanning.model.mistake.MistakeModule}} class. This class contains the affected Constraint and its Modules. \vskip .1in 
			
			\subsection{Declaration}{
				\begin{lstlisting}[frame=none]
public class MistakeConstraint
 extends studyplanning.model.mistake.MistakeModule
				\end{lstlisting}
			}
			
			\subsection{Constructor summary}{
				\begin{verse}
					\hyperlink{studyplanning.model.mistake.MistakeConstraint(java.lang.String, studyplanning.model.workflow.constraint.Constraint)}{{\bf MistakeConstraint(String, Constraint)}} Creates a new Mistake with the given Mistake key word and the affected Constraint.\\
				\end{verse}
			}
			
			\subsection{Method summary}{
				\begin{verse}
					\hyperlink{studyplanning.model.mistake.MistakeConstraint.getLocalizedMessage(studyplanning.view.Locale)}{{\bf getLocalizedMessage(Locale)}} \\
					\hyperlink{studyplanning.model.mistake.MistakeConstraint.getViolatedConstraint()}{{\bf getViolatedConstraint()}} Returns the violated Constraint, this Mistake represents.\\
				\end{verse}
			}
			
			\subsection{Constructors}{
				\vskip -2em
				\begin{itemize}
					\item{ 
						\index{MistakeConstraint!MistakeConstraint(String, Constraint)}
						\hypertarget{studyplanning.model.mistake.MistakeConstraint(java.lang.String, studyplanning.model.workflow.constraint.Constraint)}{{\bf  MistakeConstraint}\\}
						\begin{lstlisting}[frame=none]
public MistakeConstraint(String localizationKey, Constraint constraint)
						\end{lstlisting} %end signature
						\begin{itemize}
							\item{
								{\bf  Description}
								Creates a new Mistake with the given Mistake key-word and the affected Constraint.
							}
							\item{
								{\bf  Parameters}
								\begin{itemize}
								   \item{\texttt{localizationKey} -- The key-word for the Mistake}
								   \item{\texttt{constraint} -- The affected Constraint}
								\end{itemize}
							}%end item
						\end{itemize}
					}%end item
				\end{itemize}
			}
			
			\subsection{Methods}{
				\vskip -2em
				\begin{itemize}
					\item{ 
						\index{MistakeConstraint!getLocalizedMessage(Locale)}
						\hypertarget{studyplanning.model.mistake.MistakeConstraint.getLocalizedMessage(studyplanning.view.Locale)}{{\bf  getLocalizedMessage}\\}
						\begin{lstlisting}[frame=none]
public String getLocalizedMessage(Locale language)
						\end{lstlisting} %end signature
						\begin{itemize}
							\item{
								{\bf  Description copied from \hyperlink{studyplanning.model.mistake.Mistake}{Mistake}{\small \refdefined{studyplanning.model.mistake.Mistake}} }
								Returns a readable String for a mistake.
							}
							\item{
								{\bf  Parameters}
								\begin{itemize}
								   \item{\texttt{language} -- The target Language}
								\end{itemize}
							}%end item
							\item{
								{\bf  Returns} 
								-- The translated String 
								}%end item
						\end{itemize}
					}%end item
					\item{ 
						\index{MistakeConstraint!getViolatedConstraint()}
						\hypertarget{studyplanning.model.mistake.MistakeConstraint.getViolatedConstraint()}{{\bf  getViolatedConstraint}\\}
						\begin{lstlisting}[frame=none]
public Constraint getViolatedConstraint()
						\end{lstlisting} %end signature
						\begin{itemize}
							\item{
								{\bf  Description}
								Returns the violated Constraint, this Mistake represents.
							}
							\item{
								{\bf  Returns} 
								-- The violated Constraint. 
							}%end item
						\end{itemize}
					}%end item
				\end{itemize}
			}

			\subsection{Members inherited from class MistakeModule }{
				\texttt{studyplanning.model.mistake.MistakeModule} {\small 
				\refdefined{studyplanning.model.mistake.MistakeModule}}{
					\small 
					\vskip -2em
					\begin{itemize}
						\item{\vskip -1.5ex 
							\texttt{public String {\bf  getLocalizedMessage}(\texttt{Locale} {\bf language})}%end signature
						}%end item
						\item{\vskip -1.5ex 
							\texttt{protected Module {\bf  getModule}()}%end signature
						}%end item
					\end{itemize}
				}
			}
			
			\subsection{Members inherited from class Mistake }{
				\texttt{studyplanning.model.mistake.Mistake} {\small 
				\refdefined{studyplanning.model.mistake.Mistake}}{
					\small 
					\vskip -2em
					\begin{itemize}
						\item{
							\vskip -1.5ex 
							\texttt{public String {\bf  getLocalizationKey}()}%end signature
						}%end item
						\item{
							\vskip -1.5ex 
							\texttt{public String {\bf  getLocalizedMessage}(\texttt{studyplanning.view.Locale} {\bf  language})}%end signature
						}%end item
					\end{itemize}
				}
			}	
		}
	
		\section{\label{studyplanning.model.mistake.MistakeModule}Class \index{MistakeModule} MistakeModule}{
			\hypertarget{studyplanning.model.mistake.MistakeModule}{}\vskip .1in 
			A subclass of the \texttt{\small \hyperlink{studyplanning.model.mistake.Mistake}{Mistake}}{\small 
			\refdefined{studyplanning.model.mistake.Mistake}} class, which is used for Mistakes connected to a certain Module. This class is used if a Mistake is strictly coherent with a Module.\vskip .1in 
			
			\subsection{Declaration}{
				\begin{lstlisting}[frame=none]
public class MistakeModule
 extends studyplanning.model.mistake.Mistake
				\end{lstlisting}
			}
			
			\subsection{All known subclasses}{
				MistakeConstraint\small{\refdefined{studyplanning.model.mistake.MistakeConstraint}}
			}
			
			\subsection{Constructor summary}{
				\begin{verse}
					\hyperlink{studyplanning.model.mistake.MistakeModule(java.lang.String, studyplanning.model.workflow.Module)}{{\bf MistakeModule(String, Module)}} Creates a new object with localization key-word and a Module.\\
				\end{verse}
			}
			
			\subsection{Method summary}{
				\begin{verse}
					\hyperlink{studyplanning.model.mistake.MistakeModule.getLocalizedMessage(studyplanning.view.Locale)}{{\bf getLocalizedMessage(Locale)}} \\
					\hyperlink{studyplanning.model.mistake.MistakeModule.getModule()}{{\bf getModule()}}Returns the Module causing the Mistake. \\
				\end{verse}
			}
			
			\subsection{Constructors}{
				\vskip -2em
				\begin{itemize}
					\item{
						\index{MistakeModule!MistakeModule(String, Module)}
						\hypertarget{studyplanning.model.mistake.MistakeModule(java.lang.String, studyplanning.model.workflow.Module)}{{\bf  MistakeModule}\\}
						\begin{lstlisting}[frame=none]
public MistakeModule(String localizationKey, Module module)
						\end{lstlisting} %end signature
						\begin{itemize}
							\item{
								{\bf  Description}
								Creates a new object with localization key-word and a Module.
							}
							\item{
								{\bf  Parameters}
								\begin{itemize}
									\item{\texttt{localizationKey} -- The Mistake key word}
									\item{\texttt{module} -- The coherent Module}
								\end{itemize}
							}%end item
						\end{itemize}
					}%end item
				\end{itemize}
			}
			
			\subsection{Methods}{
				\vskip -2em
				\begin{itemize}
					\item{ 
						\index{MistakeModule!getLocalizedMessage(Locale)}
						\hypertarget{studyplanning.model.mistake.MistakeModule.getLocalizedMessage(studyplanning.view.Locale)}{{\bf  getLocalizedMessage}\\}
						\begin{lstlisting}[frame=none]
public String getLocalizedMessage(Locale language)
						\end{lstlisting} %end signature
						\begin{itemize}
							\item{
								{\bf  Description copied from \hyperlink{studyplanning.model.mistake.Mistake}{Mistake}{\small \refdefined{studyplanning.model.mistake.Mistake}} }
								Returns a String, which can be understood by a human, for a Mistake.
							}
							\item{
								{\bf  Parameters}
								\begin{itemize}
								   \item{\texttt{language} -- The target Language}
								\end{itemize}
							}%end item
							\item{
								{\bf  Returns} 
								-- The translated String 
							}%end item
						\end{itemize}
					}%end item
					\item{ 
						\index{MistakeModule!getModule()}
						\hypertarget{studyplanning.model.mistake.MistakeModule.getModule()}{{\bf  getModule}\\}
						\begin{lstlisting}[frame=none]
protected Module getModule()
						\end{lstlisting} %end signature
						\begin{itemize}
							\item{
								{\bf  Description}
								Returns the Module causing the Mistake.
							}
							\item{
								{\bf  Returns} -- The Module affected by the Mistake 
							}%end item
						\end{itemize}
					}%end item
				\end{itemize}
			}

			\subsection{Members inherited from class Mistake }{
				\texttt{studyplanning.model.mistake.Mistake} {\small 
				\refdefined{studyplanning.model.mistake.Mistake}}{
					\small 
					\vskip -2em
					\begin{itemize}
						\item{
							\vskip -1.5ex 
							\texttt{public String {\bf  getLocalizationKey}()}%end signature
						}%end item
						\item{
							\vskip -1.5ex 
							\texttt{public String {\bf  getLocalizedMessage}(\texttt{studyplanning.view.Locale} {\bf  language})}%end signature
						}%end item
					\end{itemize}
				}
			}
		
		\section{\label{studyplanning.model.mistake.Mistakes}Class \index{Mistakes} Mistakes}{
			\hypertarget{studyplanning.model.mistake.Mistakes}{}\vskip .1in 
			Class with factory methods for quick access to new \texttt{\small \hyperlink{studyplanning.model.mistake.Mistake}{Mistake}}{\small 
			\refdefined{studyplanning.model.mistake.Mistake}}s.\vskip .1in 
			
			\subsection{Declaration}{
				\begin{lstlisting}[frame=none]
public class Mistakes
				\end{lstlisting}

			\subsection{Method summary}{
				\begin{verse}
					\hyperlink{studyplanning.model.mistake.Mistakes.getDuplicateModule(studyplanning.model.workflow.Module)}{{\bf getDuplicateModule(Module)}} Returns a new Mistake representing a Module being duplicated in a Workflow.\\
					\hyperlink{studyplanning.model.mistake.Mistakes.getMissingCompulsaryModule(studyplanning.model.workflow.Module)}{{\bf getMissingCompulsaryModule(Module)}} Returns a new Mistake representing a missing compulsory Module.\\
					\hyperlink{studyplanning.model.mistake.Mistakes.getMissingECTS()}{{\bf getMissingECTS()}} Returns a new Mistake, in case of a Workflow with not enough ECTS points.\\
					\hyperlink{studyplanning.model.mistake.Mistakes.getTooManyECTS()}{{\bf getTooManyECTS()}} Returns a new Mistake, in case of a Workflow with too many ECTS points.\\
					\hyperlink{studyplanning.model.mistake.Mistakes.getViolatedConstraint(studyplanning.model.workflow.constraint.Constraint)}{{\bf getViolatedConstraint(Constraint)}} Returns a new Mistake representing a violated Constraint in a Workflow.\\
				\end{verse}
			}

			\subsection{Methods}{
				\vskip -2em
				\begin{itemize}
					\item{ 
						\index{Mistakes!getDuplicateModule(Module)}
						\hypertarget{studyplanning.model.mistake.Mistakes.getDuplicateModule(studyplanning.model.workflow.Module)}{{\bf  getDuplicateModule}\\}
						\begin{lstlisting}[frame=none]
public static Mistake getDuplicateModule(Module module)
						\end{lstlisting} %end signature
						\begin{itemize}
							\item{
								{\bf  Description}
								Returns a new Mistake representing a Module being duplicated in a Workflow.
							}
							\item{
								{\bf  Parameters}
								\begin{itemize}
								   \item{
										\texttt{module} -- The duplicated Module}
								\end{itemize}
							}%end item
							\item{
								{\bf  Returns} -- The constructed Mistake. 
							}%end item
						\end{itemize}
					}%end item
					\item{ 
						\index{Mistakes!getMissingCompulsaryModule(Module)}
						\hypertarget{studyplanning.model.mistake.Mistakes.getMissingCompulsaryModule(studyplanning.model.workflow.Module)}{{\bf  getMissingCompulsaryModule}\\}
						\begin{lstlisting}[frame=none]
public static Mistake getMissingCompulsaryModule(Module module)
						\end{lstlisting} %end signature
						\begin{itemize}
							\item{
								{\bf  Description}
								Returns a new Mistake representing a missing compulsory Module.
							}
							\item{
								{\bf  Parameters}
								\begin{itemize}
									\item{
										\texttt{module} -- The Module, that is missing.}
								\end{itemize}
							}%end item
							\item{
								{\bf  Returns} -- The constructed Mistake. 
							}%end item
						\end{itemize}
					}%end item
					\item{ 
						\index{Mistakes!getMissingECTS()}
						\hypertarget{studyplanning.model.mistake.Mistakes.getMissingECTS()}{{\bf  getMissingECTS}\\}
						\begin{lstlisting}[frame=none]
public static Mistake getMissingECTS()
						\end{lstlisting} %end signature
						\begin{itemize}
							\item{
								{\bf  Description}
								Returns a new Mistake, in case of a Workflow with not enough ECTS points.
							}
							\item{
								{\bf  Returns} 
								-- The constructed Mistake. 
							}%end item
						\end{itemize}
					}%end item
					\item{ 
						\index{Mistakes!getTooManyECTS()}
						\hypertarget{studyplanning.model.mistake.Mistakes.getTooManyECTS()}{{\bf  getTooManyECTS}\\}
						\begin{lstlisting}[frame=none]
public static Mistake getTooManyECTS()
						\end{lstlisting} %end signature
						\begin{itemize}
							\item{
								{\bf  Description}
								Returns a new Mistake, in case of a Workflow with too many ECTS points.
							}
							\item{
								{\bf  Returns} -- The constructed Mistake. 
							}%end item
						\end{itemize}
					}%end item
					\item{ 
						\index{Mistakes!getViolatedConstraint(Constraint)}
						\hypertarget{studyplanning.model.mistake.Mistakes.getViolatedConstraint(studyplanning.model.workflow.constraint.Constraint)}{{\bf  getViolatedConstraint}\\}
						\begin{lstlisting}[frame=none]
public static Mistake getViolatedConstraint(Constraint constraint)
						\end{lstlisting} %end signature
						\begin{itemize}
							\item{
								{\bf  Description}
								Returns a new Mistake representing a violated Constraint in a Workflow.
							}
							\item{
								{\bf  Parameters}
								\begin{itemize}
								\item{
									\texttt{constraint} -- The Constraint that was violated.
								}
								\end{itemize}
							}%end item
							\item{
								{\bf  Returns}
								 -- The constructed Mistake. 
							}%end item
						\end{itemize}
					}%end item
				\end{itemize}
			}
		}
	}

	\chapter{Package studyplanning.model.workflow.constraint}{
		\label{studyplanning.model.workflow.constraint}\hypertarget{studyplanning.model.workflow.constraint}{}
		\hskip -.05in
		\hbox to \hsize{\textit{ Package Contents\hfil Page}}
		\vskip .13in
		\hbox{{\bf  Classes}}
		\entityintro{Constraint}{studyplanning.model.workflow.constraint.Constraint}{This represents a constraint with a way to verify it.}
		\entityintro{ConstraintIntersection}{studyplanning.model.workflow.constraint.ConstraintIntersection}{Objects of this class are checking if an other from is in the same Semester as this.}
		\entityintro{ConstraintRequirement}{studyplanning.model.workflow.constraint.ConstraintRequirement}{Objects of this class are checking if a required Module has been completed before.}
		\entityintro{ConstraintRequirementUnordered}{studyplanning.model.workflow.constraint.ConstraintRequirementUnordered}{A Constraint for Modules which need other Modules in the same workflow, but not in a specified order.}
		\entityintro{ConstraintSameSemester}{studyplanning.model.workflow.constraint.ConstraintSameSemester}{This represents a constraint between two Modules, which need to be in the same Semester.}
		\entityintro{ConstraintType}{studyplanning.model.workflow.constraint.ConstraintType}{Different types of constraints.}
		\vskip .1in
		\vskip .1in
		
		\section{\label{studyplanning.model.workflow.constraint.Constraint}Class \index{Constraint} Constraint}{
			\hypertarget{studyplanning.model.workflow.constraint.Constraint}{}\vskip .1in 
			This represents a constraint with a way to verify it.\vskip .1in 
			
			\subsection{Declaration}{
				\begin{lstlisting}[frame=none]
public abstract class Constraint
				\end{lstlisting}
			}
			
			\subsection{All known subclasses}{
			ConstraintSameSemester\small{\refdefined{studyplanning.model.workflow.constraint.ConstraintSameSemester}}, ConstraintRequirementUnordered\small{\refdefined{studyplanning.model.workflow.constraint.ConstraintRequirementUnordered}}, ConstraintRequirement\small{\refdefined{studyplanning.model.workflow.constraint.ConstraintRequirement}}, ConstraintIntersection\small{\refdefined{studyplanning.model.workflow.constraint.ConstraintIntersection}}
			}
			
			\subsection{Constructor summary}{
				\begin{verse}
					\hyperlink{studyplanning.model.workflow.constraint.Constraint(studyplanning.model.workflow.Module, studyplanning.model.workflow.Module)}{{\bf Constraint(Module, Module)}} \\
				\end{verse}	
			}			
			\subsection{Method summary}{
				\begin{verse}
					\hyperlink{studyplanning.model.workflow.constraint.Constraint.getSourceModule()}{{\bf getSourceModule()}} Returns the secondary from this constraint applies to.\\
					\hyperlink{studyplanning.model.workflow.constraint.Constraint.getTargetModule()}{{\bf getTargetModule()}} Returns the from this constraint applies to.\\
					\hyperlink{studyplanning.model.workflow.constraint.Constraint.getType()}{{\bf getType()}} Returns the constraint type\\
					\hyperlink{studyplanning.model.workflow.constraint.Constraint.isSatisfied(studyplanning.model.workflow.Workflow, studyplanning.model.workflow.Semester)}{{\bf isSatisfied(Workflow, Semester)}} Returns whether the Workflow satisfies this constraint.\\
				\end{verse}
			}
			
			\subsection{Constructors}{
				\vskip 2em
				\begin{itemize}
					\item{
						\index{Constraint!Constraint(Module, Module)}
						\hypertarget{studyplanning.model.workflow.constraint.Constraint(studyplanning.model.workflow.Module, studyplanning.model.workflow.Module)}{{\bf  Constraint}\\}
						\begin{lstlisting}[frame=none]
public Constraint(Module targetModule, Module sourceModule)
						\end{lstlisting} %end signature
						\begin{itemize}
							\item{
								{\bf Description}
								Creates a new constraint between two modules. The specification of each specific Constraint are considered in sub classes.
							}
							\item{
								{\bf Parameters}
								\begin{itemize}
									\item{\texttt{targetModule} -- The Module acting as a dependency.}
									\item{\texttt{sourceModule} -- The Module, this dependency applies to.}
								\end{itemize}
							}
						\end{itemize}
					}
				\end{itemize}	
			}
			
			\subsection{Methods}{
				\vskip -2em
				\begin{itemize}
					\item{ 
						\index{Constraint!getSourceModule()}
						\hypertarget{studyplanning.model.workflow.constraint.Constraint.getSourceModule()}{{\bf  getSourceModule}\\}
						\begin{lstlisting}[frame=none]
public Module getSourceModule()
						\end{lstlisting} %end signature
						\begin{itemize}
							\item{
								{\bf  Description}
								Returns the secondary Module this constraint applies to. If Module A requires Module B, then this will return Module A.
							}
							\item{
								{\bf  Returns}
								 -- The source Module
							}%end item
						\end{itemize}
					}%end item
					\item{ 
						\index{Constraint!getTargetModule()}
						\hypertarget{studyplanning.model.workflow.constraint.Constraint.getTargetModule()}{{\bf  getTargetModule}\\}
						\begin{lstlisting}[frame=none]
public Module getTargetModule()
						\end{lstlisting} %end signature
						\begin{itemize}
							\item{
								{\bf  Description}
								Returns the Module this constraint applies to. If Module A requires Module B, then this will return Module B.
							}
							\item{
								{\bf  Returns} 
								-- The target Module
							}%end item
						\end{itemize}
					}%end item
					\item{ 
						\index{Constraint!getType()}
						\hypertarget{studyplanning.model.workflow.constraint.Constraint.getType()}{{\bf  getType}\\}
						\begin{lstlisting}[frame=none]
public ConstraintType getType()
						\end{lstlisting} %end signature
						\begin{itemize}
							\item{
								{\bf  Description}
								Returns the constraint type. See \texttt{\small \hyperlink{studyplanning.model.workflow.constraint.ConstraintType}{ConstraintType}}{\small 
								\refdefined{studyplanning.model.workflow.constraint.ConstraintType}}
							}
							\item{
								{\bf  Returns} 
								-- The type of Constraint. 
							}%end item
						\end{itemize}
					}%end item
					\item{ 
						\index{Constraint!isSatisfied(Workflow, Semester)}
						\hypertarget{studyplanning.model.workflow.constraint.Constraint.isSatisfied(studyplanning.model.workflow.Workflow, studyplanning.model.workflow.Semester)}{{\bf  isSatisfied}\\}
						\begin{lstlisting}[frame=none]
public abstract boolean isSatisfied(Workflow workflow, Semester semester)
						\end{lstlisting} %end signature
						\begin{itemize}
							\item{
								{\bf  Description}
								Returns whether the Workflow satisfies this constraint..
							}
							\item{
								{\bf  Parameters}
								\begin{itemize}
									\item{\texttt{workflow} -- The Workflow to verify with the current Constraint.}
									\item{\texttt{semester} -- The Semester, the current Module is in.}
								\end{itemize}
							}%end item
							\item{
								{\bf  Returns} -- True, if the constraint is satisfied 
							}%end item
						\end{itemize}
					}%end item
				\end{itemize}
			}
		}
		
		\section{\label{studyplanning.model.workflow.constraint.ConstraintIntersection}Class \index{ConstraintIntersection} ConstraintIntersection}{
			\hypertarget{studyplanning.model.workflow.constraint.ConstraintIntersection}{}\vskip .1in 
			Objects of this class are checking if another Module is in the same Semester as this. They cannot be in the same Semester.\vskip .1in 
			
			\subsection{Declaration}{
				\begin{lstlisting}[frame=none]
public class ConstraintIntersection
 extends studyplanning.model.workflow.constraint.Constraint
				\end{lstlisting}
			}
			\subsection{Constructor summary}{
				\begin{verse}
					\hyperlink{studyplanning.model.workflow.constraint.ConstraintIntersection(studyplanning.model.workflow.Module, studyplanning.model.workflow.Module)}{{\bf ConstraintIntersection(Module, Module)}} \\
				\end{verse}
			}

			\subsection{Method summary}{
				\begin{verse}
					\hyperlink{studyplanning.model.workflow.constraint.ConstraintIntersection.isSatisfied(studyplanning.model.workflow.Workflow, studyplanning.model.workflow.Semester)}{{\bf isSatisfied(Workflow, Semester)}} \\
				\end{verse}
			}

			\subsection{Constructors}{
				\vskip -2em
				\begin{itemize}
					\item{ 
						\index{ConstraintIntersection!ConstraintIntersection(Module, Module)}
						\hypertarget{studyplanning.model.workflow.constraint.ConstraintIntersection(studyplanning.model.workflow.Module, studyplanning.model.workflow.Module)}{{\bf  ConstraintIntersection}\\}
						\begin{lstlisting}[frame=none]
public ConstraintIntersection(Module targetModule, Module otherModule)
						\end{lstlisting} %end signature
						\begin{itemize}
							\item{
								{\bf Description}
								Instantiates a new Constraint between two Modules.
							}
							\item{
								{\bf Parameters}
								\begin{itemize}
									\item{\texttt{targetModule} -- The target Module}
									\item{\texttt{sourceModule} -- The source Module}
								\end{itemize}
							}
						\end{itemize}
					}%end item
				\end{itemize}
			}
			
			\subsection{Methods}{
				\vskip -2em
				\begin{itemize}
					\item{ 
						\index{ConstraintIntersection!isSatisfied(Workflow, Semester)}
						\hypertarget{studyplanning.model.workflow.constraint.ConstraintIntersection.isSatisfied(studyplanning.model.workflow.Workflow, studyplanning.model.workflow.Semester)}{{\bf  isSatisfied}\\}
						\begin{lstlisting}[frame=none]
public abstract boolean isSatisfied(Workflow workflow, Semester semester)
						\end{lstlisting} %end signature
						\begin{itemize}
							\item{
								{\bf  Description copied from \hyperlink{studyplanning.model.workflow.constraint.Constraint}{Constraint}{\small \refdefined{studyplanning.model.workflow.constraint.Constraint}} }
								Returns whether the Workflow satisfies this constraint.
							}
							\item{
								{\bf  Parameters}
								\begin{itemize}
									\item{\texttt{workflow} -- The Workflow to verify with the current constraint.}
									\item{\texttt{semester} -- The Semester, the current Module is in.}
								\end{itemize}
							}%end item
							\item{
								{\bf  Returns} 
								-- true if the Constraint is satisfied 
							}%end item
						\end{itemize}
					}%end item
				\end{itemize}
			}
			
			\subsection{Members inherited from class Constraint }{
				\texttt{studyplanning.model.workflow.constraint.Constraint} {\small 
				\refdefined{studyplanning.model.workflow.constraint.Constraint}}{
					\small 
					\vskip -2em
					\begin{itemize}
						\item{
							\vskip -1.5ex 
							\texttt{public Module {\bf  getSourceModule}()}%end signature
						}%end item
						\item{
							\vskip -1.5ex 
							\texttt{public Module {\bf  getTargetModule}()}%end signature
						}%end item
						\item{
							\vskip -1.5ex 
							\texttt{public ConstraintType {\bf  getType}()}%end signature
						}%end item
						\item{
							\vskip -1.5ex 
							\texttt{public abstract boolean {\bf  isSatisfied}(\texttt{studyplanning.model.workflow.Workflow} {\bf  workflow},
							\texttt{studyplanning.model.workflow.Semester} {\bf  semester})}%end signature
						}%end item
					\end{itemize}
				}
			}
		}
		
		\section{\label{studyplanning.model.workflow.constraint.ConstraintRequirement}Class \index{ConstraintRequirement} ConstraintRequirement}{
			\hypertarget{studyplanning.model.workflow.constraint.ConstraintRequirement}{}\vskip .1in 
			Objects of this class are checking if a required Module has been completed before.\vskip .1in 
			
			\subsection{Declaration}{
				\begin{lstlisting}[frame=none]
public class ConstraintRequirement
 extends studyplanning.model.workflow.constraint.Constraint
				\end{lstlisting}
			}
			
			\subsection{Constructor summary}{
				\begin{verse}
					\hyperlink{studyplanning.model.workflow.constraint.ConstraintRequirement(studyplanning.model.workflow.Module, studyplanning.model.workflow.Module)}{{\bf ConstraintRequirement(Module, Module)}} \\
				\end{verse}
			}
			
			\subsection{Method summary}{
				\begin{verse}
					\hyperlink{studyplanning.model.workflow.constraint.ConstraintRequirement.isSatisfied(studyplanning.model.workflow.Workflow, studyplanning.model.workflow.Semester)}{{\bf isSatisfied(Workflow, Semester)}} \\
				\end{verse}
			}

			\subsection{Constructors}{
				\vskip -2em
				\begin{itemize}
					\item{ 
						\index{ConstraintRequirement!ConstraintRequirement(Module, Module)}
						\hypertarget{studyplanning.model.workflow.constraint.ConstraintRequirement(studyplanning.model.workflow.Module, studyplanning.model.workflow.Module)}{{\bf  ConstraintRequirement}\\}
						\begin{lstlisting}[frame=none]
public ConstraintRequirement(Module targetModule, Module otherModule)
						\end{lstlisting} %end signature
						\begin{itemize}
							\item{
								{\bf Description}
								Instantiates a new Constraint between two Modules.
							}
							\item{
								{\bf Parameters}
								\begin{itemize}
									\item{\texttt{targetModule} -- The target Module}
									\item{\texttt{sourceModule} -- The source Module}
								\end{itemize}
							}
						\end{itemize}
					}%end item
				\end{itemize}
			}
			
			\subsection{Methods}{
				\vskip -2em
				\begin{itemize}
					\item{ 
						\index{ConstraintRequirement!isSatisfied(Workflow, Semester)}
						\hypertarget{studyplanning.model.workflow.constraint.ConstraintRequirement.isSatisfied(studyplanning.model.workflow.Workflow, studyplanning.model.workflow.Semester)}{{\bf  isSatisfied}\\}
						\begin{lstlisting}[frame=none]
public abstract boolean isSatisfied(Workflow workflow, Semester semester)
						\end{lstlisting} %end signature
						\begin{itemize}
							\item{
								{\bf  Description copied from \hyperlink{studyplanning.model.workflow.constraint.Constraint}{Constraint}{\small \refdefined{studyplanning.model.workflow.constraint.Constraint}} }
								Returns whether the workflow satisfies this constraint.
							}
							\item{
								{\bf  Parameters}
								\begin{itemize}
									\item{
										\texttt{workflow} -- The Workflow to verify with the current Constraint.
									}
									\item{
										\texttt{semester} -- The semester, the current Module is in.}
								\end{itemize}
							}%end item
							\item{
								{\bf  Returns} 
								-- true if the constraint is satisfied 
							}%end item
						\end{itemize}
					}%end item
				\end{itemize}
			}
			
			\subsection{Members inherited from class Constraint }{
				\texttt{studyplanning.model.workflow.constraint.Constraint} {\small 
				\refdefined{studyplanning.model.workflow.constraint.Constraint}}{
					\small 
					\vskip -2em
					\begin{itemize}
						\item{
							\vskip -1.5ex 
							\texttt{public Module {\bf  getSourceModule}()}%end signature
						}%end item
						\item{
							\vskip -1.5ex 
							\texttt{public Module {\bf  getTargetModule}()}%end signature
						}%end item
						\item{
							\vskip -1.5ex 
							\texttt{public ConstraintType {\bf  getType}()}%end signature
						}%end item
						\item{
							\vskip -1.5ex 
							\texttt{public abstract boolean {\bf  isSatisfied}(\texttt{studyplanning.model.workflow.Workflow} {\bf  workflow},
							\texttt{studyplanning.model.workflow.Semester} {\bf  semester})}%end signature
						}%end item
					\end{itemize}
				}
			}
		}
		
		\section{\label{studyplanning.model.workflow.constraint.ConstraintRequirementUnordered}Class \index{ConstraintRequirementUnordered} ConstraintRequirementUnordered}{
			\hypertarget{studyplanning.model.workflow.constraint.ConstraintRequirementUnordered}{}\vskip .1in 
			A Constraint for Modules which need other Modules in the same Workflow, but not in a specified order.\vskip .1in 
			
			\subsection{Declaration}{
				\begin{lstlisting}[frame=none]
public class ConstraintRequirementUnordered
 extends studyplanning.model.workflow.constraint.Constraint
				\end{lstlisting}

			\subsection{Constructor summary}{
				\begin{verse}
					\hyperlink{studyplanning.model.workflow.constraint.ConstraintRequirementUnordered(studyplanning.model.workflow.Module, studyplanning.model.workflow.Module)}{{\bf ConstraintRequirementUnordered(Module, Module)}} \\
				\end{verse}
			}
			
			\subsection{Method summary}{
				\begin{verse}
					\hyperlink{studyplanning.model.workflow.constraint.ConstraintRequirementUnordered.isSatisfied(studyplanning.model.workflow.Workflow, studyplanning.model.workflow.Semester)}{{\bf isSatisfied(Workflow, Semester)}} \\
				\end{verse}
			}
			
			\subsection{Constructors}{
				\vskip -2em
				\begin{itemize}
					\item{ 
						\index{ConstraintRequirementUnordered!ConstraintRequirementUnordered(Module, Module)}
						\hypertarget{studyplanning.model.workflow.constraint.ConstraintRequirementUnordered(studyplanning.model.workflow.Module, studyplanning.model.workflow.Module)}{{\bf  ConstraintRequirementUnordered}\\}
						\begin{lstlisting}[frame=none]
public ConstraintRequirementUnordered(Module targetModule, Module otherModule)
						\end{lstlisting} %end signature
						\begin{itemize}
							\item{
								{\bf Description}
								Instantiates a new Constraint between two Modules.
							}
							\item{
								{\bf Parameters}
								\begin{itemize}
									\item{\texttt{targetModule} -- The target Module}
									\item{\texttt{sourceModule} -- The source Module}
								\end{itemize}
							}
						\end{itemize}
					}%end item
				\end{itemize}
			}
			
			\subsection{Methods}{
				\vskip -2em
				\begin{itemize}
					\item{ 
						\index{ConstraintRequirementUnordered!isSatisfied(Workflow, Semester)}
						\hypertarget{studyplanning.model.workflow.constraint.ConstraintRequirementUnordered.isSatisfied(studyplanning.model.workflow.Workflow, studyplanning.model.workflow.Semester)}{{\bf  isSatisfied}\\}
						\begin{lstlisting}[frame=none]
public abstract boolean isSatisfied(Workflow workflow, Semester semester)
						\end{lstlisting} %end signature
						\begin{itemize}
							\item{
								{\bf  Description copied from \hyperlink{studyplanning.model.workflow.constraint.Constraint}{Constraint}{\small \refdefined{studyplanning.model.workflow.constraint.Constraint}} }
								Returns whether the Workflow satisfies this constraint or not.
							}
							\item{
								{\bf  Parameters}
								\begin{itemize}
									\item{\texttt{workflow} -- The Workflow to verify with the current Constraint.}
									\item{\texttt{semester} -- The semester, the current Module is in.}
								\end{itemize}
							}%end item
							\item{
								{\bf  Returns} -- True, if the Constraint is satisfied 
							}%end item
						\end{itemize}
					}%end item
				\end{itemize}
			}
			
			\subsection{Members inherited from class Constraint }{
				\texttt{studyplanning.model.workflow.constraint.Constraint} {\small 
				\refdefined{studyplanning.model.workflow.constraint.Constraint}}{
					\small 
					\vskip -2em
					\begin{itemize}
						\item{\vskip -1.5ex 
							\texttt{public Module {\bf  getSourceModule}()}%end signature
						}%end item
						\item{
							\vskip -1.5ex 
							\texttt{public Module {\bf  getTargetModule}()}%end signature
						}%end item
						\item{\vskip -1.5ex 
							\texttt{public ConstraintType {\bf  getType}()}%end signature
						}%end item
						\item{\vskip -1.5ex 
							\texttt{public abstract boolean {\bf  isSatisfied}(\texttt{studyplanning.model.workflow.Workflow} {\bf  workflow},
							\texttt{studyplanning.model.workflow.Semester} {\bf  semester})}%end signature
						}%end item
					\end{itemize}
				}
			}
		}
		
		\section{\label{studyplanning.model.workflow.constraint.ConstraintSameSemester}Class \index{ConstraintSameSemester} ConstraintSameSemester}{
			\hypertarget{studyplanning.model.workflow.constraint.ConstraintSameSemester}{}\vskip .1in 
			This represents a constraint between two Modules, which need to be in the same Semester.\vskip .1in 
			\subsection{Declaration}{
				\begin{lstlisting}[frame=none]
public class ConstraintSameSemester
 extends studyplanning.model.workflow.constraint.Constraint
				 \end{lstlisting}
			\subsection{Constructor summary}{
				\begin{verse}
					\hyperlink{studyplanning.model.workflow.constraint.ConstraintSameSemester(studyplanning.model.workflow.Module, studyplanning.model.workflow.Module)}{{\bf ConstraintSameSemester(Module, Module)}} \\
				\end{verse}
			}

			\subsection{Method summary}{
				\begin{verse}
					\hyperlink{studyplanning.model.workflow.constraint.ConstraintSameSemester.isSatisfied(studyplanning.model.workflow.Workflow, studyplanning.model.workflow.Semester)}{{\bf isSatisfied(Workflow, Semester)}} \\
				\end{verse}
			}

			\subsection{Constructors}{
				\vskip -2em
				\begin{itemize}
					\item{ 
						\index{ConstraintSameSemester!ConstraintSameSemester(Module, Module)}
						\hypertarget{studyplanning.model.workflow.constraint.ConstraintSameSemester(studyplanning.model.workflow.Module, studyplanning.model.workflow.Module)}{{\bf  ConstraintSameSemester}\\}
						\begin{lstlisting}[frame=none]
public ConstraintSameSemester(Module targetModule, Module otherModule)
						\end{lstlisting} %end signature
						\begin{itemize}
							\item{
								{\bf Description}
								Instantiates a new Constraint between two Modules.
							}
							\item{
								{\bf Parameters}
								\begin{itemize}
									\item{\texttt{targetModule} -- The target Module}
									\item{\texttt{sourceModule} -- The source Module}
								\end{itemize}
							}
						\end{itemize}
					}%end item
				\end{itemize}
			}
			
			\subsection{Methods}{
				\vskip -2em
				\begin{itemize}
					\item{ 
						\index{ConstraintSameSemester!isSatisfied(Workflow, Semester)}
						\hypertarget{studyplanning.model.workflow.constraint.ConstraintSameSemester.isSatisfied(studyplanning.model.workflow.Workflow, studyplanning.model.workflow.Semester)}{{\bf  isSatisfied}\\}
						\begin{lstlisting}[frame=none]
public abstract boolean isSatisfied(Workflow workflow, Semester semester)		
						\end{lstlisting} %end signature
						\begin{itemize}
							\item{
								{\bf  Description copied from \hyperlink{studyplanning.model.workflow.constraint.Constraint}{Constraint}{\small \refdefined{studyplanning.model.workflow.constraint.Constraint}} }
								Returns whether the Workflow satisfies this constraint.
							}
							\item{
								{\bf  Parameters}
								\begin{itemize}
									\item{\texttt{workflow} -- The Workflow to verify with the current Constraint.}
									\item{\texttt{semester} -- The semester, the current Module is in.}
								\end{itemize}
							}%end item
							\item{
								{\bf  Returns} -- True if the constraint is satisfied 
							}%end item
						\end{itemize}
					}%end item
				\end{itemize}
			}
			
			\subsection{Members inherited from class Constraint }{
				\texttt{studyplanning.model.workflow.constraint.Constraint} {\small 
				\refdefined{studyplanning.model.workflow.constraint.Constraint}}{
					\small 
					\vskip -2em
					\begin{itemize}
						\item{\vskip -1.5ex 
							\texttt{public Module {\bf  getSourceModule}()}%end signature
						}%end item
						\item{
							\vskip -1.5ex 
							\texttt{public Module {\bf  getTargetModule}()}%end signature
						}%end item
						\item{
							\vskip -1.5ex 
							\texttt{public ConstraintType {\bf  getType}()}%end signature
						}%end item
						\item{
							\vskip -1.5ex 
							\texttt{public abstract boolean {\bf  isSatisfied}(\texttt{studyplanning.model.workflow.Workflow} {\bf  workflow},
							\texttt{studyplanning.model.workflow.Semester} {\bf  semester})}%end signature
						}%end item
					\end{itemize}
				}
			}
		}
		
		\section{\label{studyplanning.model.workflow.constraint.ConstraintType}Class \index{ConstraintType} ConstraintType}{
			\hypertarget{studyplanning.model.workflow.constraint.ConstraintType}{}\vskip .1in 
			Different types of Constraints. Can be accessed via \texttt{\small \hyperlink{studyplanning.model.workflow.constraint.ConstraintType.VALUES}{VALUES}}{\small 
			\refdefined{studyplanning.model.workflow.constraint.ConstraintType.VALUES}}\lbrack id\rbrack\ \texttt{\small \hyperlink{studyplanning.model.workflow.constraint.ConstraintType.create(studyplanning.model.workflow.Module, studyplanning.model.workflow.Module)}{create(Module, Module)}}{\small 
			\refdefined{studyplanning.model.workflow.constraint.ConstraintType.create(studyplanning.model.workflow.Module, studyplanning.model.workflow.Module)}}\vskip .1in 
			
			\subsection{Declaration}{
				\begin{lstlisting}[frame=none]
public final class ConstraintType
				\end{lstlisting}
			}
			
			\subsection{Field summary}{
				\begin{verse}
					\hyperlink{studyplanning.model.workflow.constraint.ConstraintType.INTERSECTING}{{\bf INTERSECTING}} \\
					\hyperlink{studyplanning.model.workflow.constraint.ConstraintType.REQUIRED_ANY_ORDER}{{\bf REQUIRED\_ANY\_ORDER}} \\
					\hyperlink{studyplanning.model.workflow.constraint.ConstraintType.REQUIRED_BEFORE}{{\bf REQUIRED\_BEFORE}} \\
					\hyperlink{studyplanning.model.workflow.constraint.ConstraintType.REQUIRED_SAME_SEMESTER}{{\bf REQUIRED\_SAME\_SEMESTER}} \\
					\hyperlink{studyplanning.model.workflow.constraint.ConstraintType.VALUES}{{\bf VALUES}} Array containing all ConstraintTypes.\\
				\end{verse}
			}
			
			\subsection{Method summary}{
				\begin{verse}
					\hyperlink{studyplanning.model.workflow.constraint.ConstraintType.create(studyplanning.model.workflow.Module, studyplanning.model.workflow.Module)}{{\bf create(Module, Module)}} Creates a new Constraint with the given target and source Module.\\
					\hyperlink{studyplanning.model.workflow.constraint.ConstraintType.valueOf(java.lang.String)}{{\bf valueOf(String)}} \\
					\hyperlink{studyplanning.model.workflow.constraint.ConstraintType.values()}{{\bf values()}} \\
				\end{verse}
			}

			\subsection{Fields}{
				\begin{itemize}
					\item{
						\index{ConstraintType!REQUIRED\_BEFORE}
						\label{studyplanning.model.workflow.constraint.ConstraintType.REQUIRED_BEFORE}\hypertarget{studyplanning.model.workflow.constraint.ConstraintType.REQUIRED_BEFORE}{\texttt{public static final ConstraintType\ {\bf  REQUIRED\_BEFORE}}}
					}
					\item{
						\index{ConstraintType!REQUIRED\_SAME\_SEMESTER}
						\label{studyplanning.model.workflow.constraint.ConstraintType.REQUIRED_SAME_SEMESTER}\hypertarget{studyplanning.model.workflow.constraint.ConstraintType.REQUIRED_SAME_SEMESTER}{\texttt{public static final ConstraintType\ {\bf  REQUIRED\_SAME\_SEMESTER}}}
					}
					\item{
						\index{ConstraintType!REQUIRED\_ANY\_ORDER}
						\label{studyplanning.model.workflow.constraint.ConstraintType.REQUIRED_ANY_ORDER}\hypertarget{studyplanning.model.workflow.constraint.ConstraintType.REQUIRED_ANY_ORDER}{\texttt{public static final ConstraintType\ {\bf  REQUIRED\_ANY\_ORDER}}}
					}
					\item{
						\index{ConstraintType!INTERSECTING}
						\label{studyplanning.model.workflow.constraint.ConstraintType.INTERSECTING}\hypertarget{studyplanning.model.workflow.constraint.ConstraintType.INTERSECTING}{\texttt{public static final ConstraintType\ {\bf  INTERSECTING}}}
					}
					\item{
						\index{ConstraintType!VALUES}
						\label{studyplanning.model.workflow.constraint.ConstraintType.VALUES}\hypertarget{studyplanning.model.workflow.constraint.ConstraintType.VALUES}{\texttt{public static final ConstraintType\lbrack \rbrack \ {\bf  VALUES}}}
						\begin{itemize}
							\item{
								\vskip -.9ex 
								Array containing all ConstraintTypes. Same as \texttt{\small \hyperlink{studyplanning.model.workflow.constraint.ConstraintType.values()}{values()}}{\small 
								\refdefined{studyplanning.model.workflow.constraint.ConstraintType.values()}}, but with greater performance.}
						\end{itemize}
					}
				\end{itemize}
			}
			
			\subsection{Methods}{
				\vskip -2em
				\begin{itemize}
					\item{ 
						\index{ConstraintType!create(Module, Module)}
						\hypertarget{studyplanning.model.workflow.constraint.ConstraintType.create(studyplanning.model.workflow.Module, studyplanning.model.workflow.Module)}{{\bf  create}\\}
						\begin{lstlisting}[frame=none]
public Constraint create(Module targetModule, Module requiredModule)
						\end{lstlisting} %end signature
						\begin{itemize}
							\item{
								{\bf  Description}
								Creates a new Constraint with the given target and source Module.
							}
							\item{
								{\bf  Parameters}
								\begin{itemize}
									\item{
										\texttt{targetModule} -- The target Module for this constraint. This corresponds to \texttt{\small \hyperlink{studyplanning.model.workflow.constraint.Constraint.getTargetModule()}{getTargetModule()}}{\small 
										\refdefined{studyplanning.model.workflow.constraint.Constraint.getTargetModule()}}
									}
								   \item{
										\texttt{sourceModule} -- The source from for this constraint. This corresponds to \texttt{\small \hyperlink{studyplanning.model.workflow.constraint.Constraint.getSourceModule()}{getSourceModule()}}{\small 
										\refdefined{studyplanning.model.workflow.constraint.Constraint.getSourceModule()}}}
								\end{itemize}
							}%end item
							\item{
								{\bf  Returns} 
								-- A new Vonstraint of this type and the two Modules. 
							}%end item
						\end{itemize}
					}%end item
					\item{ 
						\index{ConstraintType!valueOf(String)}
						\hypertarget{studyplanning.model.workflow.constraint.ConstraintType.valueOf(java.lang.String)}{{\bf  valueOf}\\}
						\begin{lstlisting}[frame=none]
public static ConstraintType valueOf(String name)
						\end{lstlisting} %end signature
					}%end item
					\item{ 
						\index{ConstraintType!values()}
						\hypertarget{studyplanning.model.workflow.constraint.ConstraintType.values()}{{\bf  values}\\}
						\begin{lstlisting}[frame=none]
public static ConstraintType[] values()
						\end{lstlisting} %end signature
					}%end item
				\end{itemize}
			}
			
			\subsection{Members inherited from class Enum }{
				\texttt{java.lang.Enum} {\small 
				\refdefined{java.lang.Enum}}{
					\small 
					\vskip -2em
					\begin{itemize}
						\item{
							\vskip -1.5ex 
							\texttt{protected final Object {\bf  clone}() throws CloneNotSupportedException}%end signature
						}%end item
						\item{
							\vskip -1.5ex 
							\texttt{public final int {\bf  compareTo}(\texttt{Enum} {\bf  arg0})}%end signature
						}%end item
						\item{
							\vskip -1.5ex 
							\texttt{public final boolean {\bf  equals}(\texttt{Object} {\bf  arg0})}%end signature
						}%end item
						\item{
							\vskip -1.5ex 
							\texttt{protected final void {\bf  finalize}()}%end signature
						}%end item
						\item{
							\vskip -1.5ex 
							\texttt{public final Class {\bf  getDeclaringClass}()}%end signature
						}%end item
						\item{
							\vskip -1.5ex 
							\texttt{public final int {\bf  hashCode}()}%end signature
						}%end item
						\item{
							\vskip -1.5ex 
							\texttt{public final String {\bf  name}()}%end signature
						}%end item
						\item{
							\vskip -1.5ex 
							\texttt{public final int {\bf  ordinal}()}%end signature
						}%end item
						\item{
							\vskip -1.5ex 
							\texttt{public String {\bf  toString}()}%end signature
						}%end item
						\item{
							\vskip -1.5ex 
							\texttt{public static Enum {\bf  valueOf}(\texttt{Class} {\bf  arg0}, \texttt{String} {\bf  arg1})}%end signature
						}%end item
					\end{itemize}
				}
			}
		}
	}
	
	\chapter{Package studyplanning.model.workflow}{
		\label{studyplanning.model.workflow}\hypertarget{studyplanning.model.workflow}{}
		\hskip -.05in
		\hbox to \hsize{\textit{ Package Contents\hfil Page}}
		\vskip .13in
		\hbox{{\bf  Classes}}
		\entityintro{DataSet}{studyplanning.model.workflow.DataSet}{This represents the set of data we get from the database.}
		\entityintro{Module}{studyplanning.model.workflow.Module}{All modules are objects of this class.}
		\entityintro{ModuleWrapper}{studyplanning.model.workflow.ModuleWrapper}{A class for wrapping around a Module to store workflow-specific info like constraint violations.}
		\entityintro{Semester}{studyplanning.model.workflow.Semester}{This class is representing a semester of the user.}
		\entityintro{SemesterType}{studyplanning.model.workflow.SemesterType}{Enumeration for the semester type of Modules.}
		\entityintro{StudySubject}{studyplanning.model.workflow.StudySubject}{The subject of study.}
		\entityintro{Workflow}{studyplanning.model.workflow.Workflow}{Representation of the study plan.}
		\entityintro{WorkflowTasks}{studyplanning.model.workflow.WorkflowTasks}{This class is dedicated to verifying and generating workflows.}
		\vskip .1in
		\vskip .1in
		
		\section{\label{studyplanning.model.workflow.DataSet}Class \index{DataSet} DataSet}{
			\hypertarget{studyplanning.model.workflow.DataSet}{}\vskip .1in 
			This represents the set of data we get from the database.\vskip .1in 
			
			\subsection{Declaration}{
				\begin{lstlisting}[frame=none]
public class DataSet
				\end{lstlisting}
			}
			
			\subsection{Constructor summary}{
				\begin{verse}
					\hyperlink{studyplanning.model.workflow.DataSet()}{{\bf DataSet()}} Creates a new DataSet and adds all Modules to the specific category.\\
				\end{verse}
			}
			
			\subsection{Method summary}{
				\begin{verse}
					\hyperlink{studyplanning.model.workflow.DataSet.getCompulsaryModules()}{{\bf getCompulsaryModules()}} Returns all Modules a student of this study course needs to finish before graduating. \\ %study subject umschreiben
					\hyperlink{studyplanning.model.workflow.DataSet.getModules()}{{\bf getModules()}} Returns all Modules contained in this DataSet.\\
					\hyperlink{studyplanning.model.workflow.DataSet.getModulesInCategory(java.lang.String)}{{\bf getModulesInCategory(String)}} Returns all the Modules to a belonging category.\\
				\end{verse}
			}
			
			\subsection{Constructors}{
				\vskip -2em
				\begin{itemize}
					\item{ 
						\index{DataSet!DataSet()}
						\hypertarget{studyplanning.model.workflow.DataSet()}{{\bf  DataSet}\\}
						\begin{lstlisting}[frame=none]
public DataSet()
						\end{lstlisting} %end signature
						\begin{itemize}
							\item{
								{\bf  Description}
								Creates a new DataSet and adds all Modules to the specific category.
							}
						\end{itemize}
					}%end item
				\end{itemize}
			}
			
			\subsection{Methods}{
				\vskip -2em
				\begin{itemize}
					\item{ 
						\index{DataSet!getCompulsaryModules()}
						\hypertarget{studyplanning.model.workflow.DataSet.getCompulsaryModules()}{{\bf  getCompulsaryModules}\\}
						\begin{lstlisting}[frame=none]
public Collection<Module> getCompulsaryModules()
						\end{lstlisting} %end signature
						\begin{itemize}
							\item{
								{\bf Description}
								Returns all Modules a student of this StudySubject needs to finish before graduating.
							}
							\item{
								{\bf  Returns} 
								-- Returns all compulsory Modules. 
							}%end item
						\end{itemize}
					}%end item
					\item{ 
						\index{DataSet!getModules()}
						\hypertarget{studyplanning.model.workflow.DataSet.getModules()}{{\bf  getModules}\\}
						\begin{lstlisting}[frame=none]
public Collection<Module> getModules()
						\end{lstlisting} %end signature
						\begin{itemize}
							\item{
								{\bf  Description}
								Returns all Modules contained in this DataSet.
							}
							\item{
								{\bf  Returns} -- Returns all Modules in this instance 
							}%end item
						\end{itemize}
					}%end item
					\item{ 
						\index{DataSet!getModulesInCategory(String)}
						\hypertarget{studyplanning.model.workflow.DataSet.getModulesInCategory(java.lang.String)}{{\bf  getModulesInCategory}\\}
						\begin{lstlisting}[frame=none]
public Collection<Module> getModulesInCategory(String category)
						\end{lstlisting} %end signature
						\begin{itemize}
							\item{
								{\bf  Description}
								Returns all the Modules that belong to the given category.
							}
							\item{
								{\bf  Parameters}
								\begin{itemize}
									\item{\texttt{category} -- The key-word for a categeroy}
								\end{itemize}
							}%end item
							\item{
								{\bf  Returns} -- All Modules to the given key-word 
							}%end item
						\end{itemize}
					}%end item
				\end{itemize}
			}
		}
		
		\section{\label{studyplanning.model.workflow.Module}Class \index{Module} Module}{
			\hypertarget{studyplanning.model.workflow.Module}{}\vskip .1in 
			All Modules are objects of this class. These objects will be generated at the start of the system.\vskip .1in 
			
			\subsection{Declaration}{
				\begin{lstlisting}[frame=none]
public class Module
				\end{lstlisting}
			}
			
			\subsection{Constructor summary}{
				\begin{verse}
					\hyperlink{studyplanning.model.workflow.Module(java.lang.String, studyplanning.model.workflow.SemesterType, int, boolean)}{{\bf Module(String, SemesterType, int, boolean)}} Creates a new Module with the given arguments.\\
				\end{verse}
			}
			
			\subsection{Method summary}{
				\begin{verse}
					\hyperlink{studyplanning.model.workflow.Module.getConstraints()}{{\bf getConstraints()}} Returns a Collection of Constraints, all of which need to be met, when a workflow is considered as valid.\\
					\hyperlink{studyplanning.model.workflow.Module.getEctsPoints()}{{\bf getEctsPoints()}} Returns the amount of ECTS points this Module rewards.\\
					\hyperlink{studyplanning.model.workflow.Module.getSemester()}{{\bf getSemester()}} Returns SemesterType.SUMMER, if this Module happens in the summer semester and SemesterType.WINTER, if this from happens in the winter semester.\\
					\hyperlink{studyplanning.model.workflow.Module.getUnlocalizedName()}{{\bf getUnlocalizedName()}} Returns the unlocalized name of this Module.\\
					\hyperlink{studyplanning.model.workflow.Module.isCompulsary()}{{\bf isCompulsory()}} Whether this Module is compulsory or not.\\
				\end{verse}
			}
			
			\subsection{Constructors}{
				\vskip -2em
				\begin{itemize}
					\item{ 
						\index{Module!Module(String, SemesterType, int, boolean)}
						\hypertarget{studyplanning.model.workflow.Module(java.lang.String, studyplanning.model.workflow.SemesterType, int, boolean)}{{\bf  Module}\\}
						\begin{lstlisting}[frame=none]
public Module(String name, SemesterType semester, int ectsPoints, boolean compulsory)
						\end{lstlisting} %end signature
						\begin{itemize}
							\item{
								{\bf  Description}
								Creates a new Module with the given arguments.
							}
							\item{
								{\bf  Parameters}
								\begin{itemize}
									\item{\texttt{name} -- The name of the Module.}
									\item{\texttt{semester} -- The semester this event is offered.}
									\item{\texttt{ectsPoints} -- The amount of ECTS points granted, by completing this Module.}
									\item{\texttt{compulsory} -- Is the Module compulsory or not.}
								\end{itemize}
							}%end item
						\end{itemize}
					}%end item
				\end{itemize}
			}
		
			\subsection{Methods}{
				\vskip -2em
				\begin{itemize}
					\item{ 
						\index{Module!getConstraints()}
						\hypertarget{studyplanning.model.workflow.Module.getConstraints()}{{\bf  getConstraints}\\}
						\begin{lstlisting}[frame=none]
public Collection<Constraint> getConstraints()
						\end{lstlisting} %end signature
						\begin{itemize}
							\item{
								{\bf  Description}
								Returns a Collection of Constraints, all of which need to be met, when a Workflow is considered as valid.
							}
							\item{
								{\bf  Returns} -- A Collection of Constraints. 
							}%end item
						\end{itemize}
					}%end item
					\item{ 
						\index{Module!getEctsPoints()}
						\hypertarget{studyplanning.model.workflow.Module.getEctsPoints()}{{\bf  getEctsPoints}\\}
						\begin{lstlisting}[frame=none]
public int getEctsPoints()
						\end{lstlisting} %end signature
						\begin{itemize}
							\item{
								{\bf  Description}
								Returns the amount of ECTS this Module rewards.
							}
							\item{
								{\bf  Returns} 
								-- the amount of ECTS points of this Module. 
							}%end item
						\end{itemize}
					}%end item
					\item{ 
						\index{Module!getSemester()}
						\hypertarget{studyplanning.model.workflow.Module.getSemester()}{{\bf  getSemester}\\}
						\begin{lstlisting}[frame=none]
public SemesterType getSemester()
						\end{lstlisting} %end signature
						\begin{itemize}
							\item{
								{\bf  Description}
								Returns SemesterType.SUMMER, if this Module happens in the summer and SemesterType.WINTER, if this Module happens in the winter semester.
							}
							\item{
								{\bf  Returns} 
								-- The SemesterType this Module happens in.
							}%end item
						\end{itemize}
					}%end item
					\item{ 
						\index{Module!getUnlocalizedName()}
						\hypertarget{studyplanning.model.workflow.Module.getUnlocalizedName()}{{\bf  getUnlocalizedName}\\}
						\begin{lstlisting}[frame=none]
public String getUnlocalizedName()
						\end{lstlisting} %end signature
						\begin{itemize}
							\item{
								{\bf  Description}
								Returns the unlocalized name of this Module.
							}
							\item{
								{\bf  Returns} 
								-- The unlocalized name of this Module. 
							}%end item
						\end{itemize}
					}%end item
					\item{ 
						\index{Module!isCompulsory()}
						\hypertarget{studyplanning.model.workflow.Module.isCompulsary()}{{\bf  isCompulsary}\\}
						\begin{lstlisting}[frame=none]
public boolean isCompulsary()
						\end{lstlisting} %end signature
						\begin{itemize}
							\item{
								{\bf  Description}
								Whether this Module is compulsory or not.
							}
							\item{
								{\bf  Returns} 
								-- True, if the Module is compulsory, false otherwise. 
							}%end item
						\end{itemize}
					}%end item
				\end{itemize}
			}
		}
		
		\section{\label{studyplanning.model.workflow.ModuleWrapper}Class \index{ModuleWrapper} ModuleWrapper}{
			\hypertarget{studyplanning.model.workflow.ModuleWrapper}{}\vskip .1in 
			A class for wrapping around a Module to store workflow-specific info, like constraint violations.\vskip .1in 
			
			\subsection{Declaration}{
				\begin{lstlisting}[frame=none]
public final class ModuleWrapper
				\end{lstlisting}
			}
			
			\subsection{Constructor summary}{
				\begin{verse}
					\hyperlink{studyplanning.model.workflow.ModuleWrapper(studyplanning.model.workflow.Module)}{{\bf ModuleWrapper(Module)}} \\
				\end{verse}
			}
			
			\subsection{Method summary}{
				\begin{verse}
					\hyperlink{studyplanning.model.workflow.ModuleWrapper.getModule()}{{\bf getModule()}} Returns the currently wrapped Module. \\
					\hyperlink{studyplanning.model.workflow.ModuleWrapper.hasMistakes()}{{\bf hasMistakes()}} Returns whether the current Module in the current Workflow does not violate any of its Constraints.\\
				\end{verse}
			}
			
			\subsection{Constructors}{
				\vskip -2em
				\begin{itemize}
					\item{ 
						\index{ModuleWrapper!ModuleWrapper(Module)}
						\hypertarget{studyplanning.model.workflow.ModuleWrapper(studyplanning.model.workflow.Module)}{{\bf  ModuleWrapper}\\}
						\begin{lstlisting}[frame=none]
public ModuleWrapper(Module module)
						\end{lstlisting} %end signature
						\begin{itemize}
							\item{
								{\bf Description}
								Creates a new instance with the given Module. Instances are getting edited by the
								verification algorithm, making them visible for the View.
							}
							\item{
								{\bf Parameters}
								\begin{itemize}
									\item{\texttt{module} -- The Module associated to this instance}
								\end{itemize}
							}
						\end{itemize}
					}%end item
				\end{itemize}
			}
			
			\subsection{Methods}{
				\vskip -2em
				\begin{itemize}
					\item{ 
						\index{ModuleWrapper!getModule()}
						\hypertarget{studyplanning.model.workflow.ModuleWrapper.getModule()}{{\bf  getModule}\\}
						\begin{lstlisting}[frame=none]
public Module getModule()
						\end{lstlisting} %end signature
						\begin{itemize}
							\item{
								{\bf  Description}
								Returns the currently wrapped Module.
							}
							\item{
								{\bf  Returns} 
								-- The current Module. 
							}%end item
						\end{itemize}
					}%end item
					\item{ 
						\index{ModuleWrapper!hasMistakes()}
						\hypertarget{studyplanning.model.workflow.ModuleWrapper.hasMistakes()}{{\bf  hasMistakes}\\}
						\begin{lstlisting}[frame=none]
public boolean hasMistakes()
						\end{lstlisting} %end signature
						\begin{itemize}
							\item{
								{\bf  Description}
								Returns whether the current Module in the current Workflow does not violate any of its Constraints. Used for drawing in the GUI.
							}
							\item{
								{\bf  Returns} 
								-- Whether this Module is OK in the current Workflow. 
							}%end item
						\end{itemize}
					}%end item
				\end{itemize}
			}
		}

		\section{\label{studyplanning.model.workflow.Semester}Class \index{Semester} Semester}{
			\hypertarget{studyplanning.model.workflow.Semester}{}\vskip .1in 
			This class is representing a semester of a Workflow.\vskip .1in 
			
			\subsection{Declaration}{
				\begin{lstlisting}[frame=none]
public class Semester
				\end{lstlisting}
			}
			
			\subsection{Constructor summary}{
				\begin{verse}
					\hyperlink{studyplanning.model.workflow.Semester(int)}{{\bf Semester(int)}} Creates a new empty Semester\\
				\end{verse}
			}
			
			\subsection{Method summary}{
				\begin{verse}
					\hyperlink{studyplanning.model.workflow.Semester.addModule(studyplanning.model.workflow.Module)}{{\bf addModule(Module)}} Adds the given Module to the Semester.\\
					\hyperlink{studyplanning.model.workflow.Semester.getEctsPoints()}{{\bf getEctsPoints()}} The sum of the ECTS points rewarded in the Semester.\\
					\hyperlink{studyplanning.model.workflow.Semester.getModules()}{{\bf getModules()}} Returns an immutable set of Modules in this Semester.\\
					\hyperlink{studyplanning.model.workflow.Semester.getSemesterType()}{{\bf getSemesterType()}} Whether this Semester represents a Summer or Winter Semester.\\
					\hyperlink{studyplanning.model.workflow.Semester.removeModule(studyplanning.model.workflow.Module)}{{\bf removeModule(Module)}} Removes the Module of the Semester\\
				\end{verse}
			}
			
			\subsection{Constructors}{
				\vskip -2em
				\begin{itemize}
					\item{ 
						\index{Semester!Semester(int)}
						\hypertarget{studyplanning.model.workflow.Semester(int)}{{\bf  Semester}\\}
						\begin{lstlisting}[frame=none]
public Semester(int id)
						\end{lstlisting} %end signature
						\begin{itemize}
							\item{
								{\bf  Description}
								Creates a new empty Semester
							}
							\item{
								{\bf Parameters}
								\begin{itemize}
									\item{\texttt{id}} -- The semester number.
								\end{itemize}
							}
						\end{itemize}
					}%end item
				\end{itemize}
			}
			
			\subsection{Methods}{
				\vskip -2em
				\begin{itemize}
					\item{ 
						\index{Semester!addModule(Module)}
						\hypertarget{studyplanning.model.workflow.Semester.addModule(studyplanning.model.workflow.Module)}{{\bf  addModule}\\}
						\begin{lstlisting}[frame=none]
public boolean addModule(Module module)
						\end{lstlisting} %end signature
						\begin{itemize}
							\item{
								{\bf  Description}
								Adds the given Module to the Semester.
							}
							\item{
								{\bf  Parameters}
								\begin{itemize}
								   \item{\texttt{module} -- The Module to be added to the Semester}
								\end{itemize}
							}%end item
							\item{
								{\bf  Returns}
								 -- True, if the from was successfully added to the Semester, false otherwise.
							}%end item
						\end{itemize}
					}%end item
					\item{ 
						\index{Semester!getEctsPoints()}
						\hypertarget{studyplanning.model.workflow.Semester.getEctsPoints()}{{\bf  getEctsPoints}\\}
						\begin{lstlisting}[frame=none]
public int getEctsPoints()
						\end{lstlisting} %end signature
						\begin{itemize}
							\item{
								{\bf  Description}
								The sum of the ECTS points rewarded in the Semester.
							}
							\item{
								{\bf  Returns} 
								-- The amount of ECTS points in this Semester. 
							}%end item
						\end{itemize}
					}%end item
					\item{ 
						\index{Semester!getModules()}
						\hypertarget{studyplanning.model.workflow.Semester.getModules()}{{\bf  getModules}\\}
						\begin{lstlisting}[frame=none]
public java.util.Collection getModules()
						\end{lstlisting} %end signature
						\begin{itemize}
							\item{
								{\bf  Description}
								Returns an immutable set of Modules in this Semester.
							}
							\item{
								{\bf  Returns} 
								-- All Modules in this Semester. 
							}%end item
						\end{itemize}
					}%end item
					\item{ 
						\index{Semester!getSemesterType()}
						\hypertarget{studyplanning.model.workflow.Semester.getSemesterType()}{{\bf  getSemesterType}\\}
						\begin{lstlisting}[frame=none]
public SemesterType getSemesterType()
						\end{lstlisting} %end signature
						\begin{itemize}
							\item{
								{\bf  Description}
								Whether this semester represents a Summer or Winter semester. \texttt{\small \hyperlink{studyplanning.model.workflow.SemesterType}{SemesterType}}{\small 
								\refdefined{studyplanning.model.workflow.SemesterType}}.
							}
							\item{
								{\bf  Returns} 
								-- The type of this Semester. 
							}%end item
						\end{itemize}
					}%end item
					\item{ 
						\index{Semester!removeModule(Module)}
						\hypertarget{studyplanning.model.workflow.Semester.removeModule(studyplanning.model.workflow.Module)}{{\bf  removeModule}\\}
						\begin{lstlisting}[frame=none]
public boolean removeModule(Module module)
						\end{lstlisting} %end signature
						\begin{itemize}
							\item{
								{\bf  Description}
								Removes the Module of the Semester
							}
							\item{
								{\bf  Parameters}
								\begin{itemize}
									\item{\texttt{module} -- The Module to remove from the Semester}
								\end{itemize}
							}%end item
							\item{
								{\bf  Returns} 
								-- True, if the Module was successfully removed from the Semester, false otherwise.
							}%end item
						\end{itemize}
					}%end item
				\end{itemize}
			}
		}

		\section{\label{studyplanning.model.workflow.SemesterType}Class \index{SemesterType} SemesterType}{
			\hypertarget{studyplanning.model.workflow.SemesterType}{}\vskip .1in 
			Enumeration for the semester type of modules. (Summer or Winter Semester)\vskip .1in 
			
			\subsection{Declaration}{
				\begin{lstlisting}[frame=none]
public final class SemesterType
				\end{lstlisting}
			}
			
			\subsection{Field summary}{
				\begin{verse}
					\hyperlink{studyplanning.model.workflow.SemesterType.SUMMER}{{\bf SUMMER}} \\
					\hyperlink{studyplanning.model.workflow.SemesterType.WINTER}{{\bf WINTER}} \\
				\end{verse}
			}
			
			\subsection{Method summary}{
				\begin{verse}
					\hyperlink{studyplanning.model.workflow.SemesterType.valueOf(java.lang.String)}{{\bf valueOf(String)}} \\
					\hyperlink{studyplanning.model.workflow.SemesterType.values()}{{\bf values()}} \\
				\end{verse}
			}
			
			\subsection{Fields}{
				\begin{itemize}
					\item{
						\index{SemesterType!SUMMER}
						\label{studyplanning.model.workflow.SemesterType.SUMMER}\hypertarget{studyplanning.model.workflow.SemesterType.SUMMER}{\texttt{public static final SemesterType\ {\bf  SUMMER}}}
					}
					\item{
						\index{SemesterType!WINTER}
						\label{studyplanning.model.workflow.SemesterType.WINTER}\hypertarget{studyplanning.model.workflow.SemesterType.WINTER}{\texttt{public static final SemesterType\ {\bf  WINTER}}}
					}
				\end{itemize}
			}
			
			\subsection{Methods}{
				\vskip -2em
				\begin{itemize}
					\item{ 
						\index{SemesterType!valueOf(String)}
						\hypertarget{studyplanning.model.workflow.SemesterType.valueOf(java.lang.String)}{{\bf  valueOf}\\}
						\begin{lstlisting}[frame=none]
public static SemesterType valueOf(String name)
						\end{lstlisting} %end signature
					}%end item
					\item{ 
						\index{SemesterType!values()}
						\hypertarget{studyplanning.model.workflow.SemesterType.values()}{{\bf  values}\\}
						\begin{lstlisting}[frame=none]
public static SemesterType[] values()
						\end{lstlisting} %end signature
					}%end item
				\end{itemize}
			}
			
			\subsection{Members inherited from class Enum }{
				\texttt{java.lang.Enum} {\small 
				\refdefined{java.lang.Enum}}{
					\small 
					\vskip -2em
					\begin{itemize}
						\item{
							\vskip -1.5ex 
							\texttt{protected final Object {\bf  clone}() throws CloneNotSupportedException}%end signature
						}%end item
						\item{
							\vskip -1.5ex 
							\texttt{public final int {\bf  compareTo}(\texttt{Enum} {\bf  arg0})}%end signature
						}%end item
						\item{
							\vskip -1.5ex 
							\texttt{public final boolean {\bf  equals}(\texttt{Object} {\bf  arg0})}%end signature
						}%end item
						\item{
							\vskip -1.5ex 
							\texttt{protected final void {\bf  finalize}()}%end signature
						}%end item
						\item{
							\vskip -1.5ex 
							\texttt{public final Class {\bf  getDeclaringClass}()}%end signature
						}%end item
						\item{
							\vskip -1.5ex 
							\texttt{public final int {\bf  hashCode}()}%end signature
						}%end item
						\item{
							\vskip -1.5ex 
							\texttt{public final String {\bf  name}()}%end signature
						}%end item
						\item{
							\vskip -1.5ex 
							\texttt{public final int {\bf  ordinal}()}%end signature
						}%end item
						\item{
							\vskip -1.5ex 
							\texttt{public String {\bf  toString}()}%end signature
						}%end item
						\item{
							\vskip -1.5ex 
							\texttt{public static Enum {\bf  valueOf}(\texttt{Class} {\bf  arg0},\texttt{String} {\bf  arg1})}%end signature
						}%end item
					\end{itemize}
				}
			}
		}	
		
		\section{\label{studyplanning.model.workflow.StudySubject}Class \index{StudySubject} StudySubject}{
			\hypertarget{studyplanning.model.workflow.StudySubject}{}\vskip .1in 
			The subject of study. By default, the only existing subject is computer science.\vskip .1in 
			
			\subsection{Declaration}{
				\begin{lstlisting}[frame=none]
public class StudySubject
				\end{lstlisting}
			}
			
			\subsection{Constructor summary}{
				\begin{verse}
					\hyperlink{studyplanning.model.workflow.StudySubject(int, int, int, java.lang.String, studyplanning.model.workflow.DataSet)}{{\bf StudySubject(int, int, int, String, DataSet)}} Creates a new StudySubject object with all needed informations.\\
				\end{verse}
			}
	
			\subsection{Method summary}{
				\begin{verse}
					\hyperlink{studyplanning.model.workflow.StudySubject.getDataSet()}{{\bf getDataSet()}} \\
					\hyperlink{studyplanning.model.workflow.StudySubject.getMaxECTS()}{{\bf getMaxECTS()}} \\
					\hyperlink{studyplanning.model.workflow.StudySubject.getMaxStudyDuration()}{{\bf getMaxStudyDuration()}} \\
					\hyperlink{studyplanning.model.workflow.StudySubject.getName()}{{\bf getName()}} \\
					\hyperlink{studyplanning.model.workflow.StudySubject.getRequiredECTS()}{{\bf getRequiredECTS()}} \\
				\end{verse}
			}
			
			\subsection{Constructors}{
				\vskip -2em
				\begin{itemize}
					\item{ 
						\index{StudySubject!StudySubject(int, int, int, String, DataSet)}
						\hypertarget{studyplanning.model.workflow.StudySubject(int, int, int, java.lang.String, studyplanning.model.workflow.DataSet)}{{\bf  StudySubject}\\}
						\begin{lstlisting}[frame=none]
public StudySubject(int maxStudyDuration, int requiredECTS, int maxECTS, String name, DataSet dataSet)
						\end{lstlisting} %end signature
						\begin{itemize}
							\item{
								{\bf  Description}
								Creates a new StudySubject object with all needed informations.
							}
							\item{
								{\bf  Parameters}
								\begin{itemize}
									\item{\texttt{maxStudyDuration} -- The max amount of semesters allowed to study}
									\item{\texttt{requiredECTS} -- The minimum amount of ECTS points which has to be reached}
									\item{\texttt{maxECTS} -- The maximum allowed amount of ECTS points}
									\item{\texttt{name} -- The name of this subject}
									\item{\texttt{dataSet} -- The belonging \texttt{\small \hyperlink{studyplanning.model.workflow.DataSet}{DataSet}}{\small \refdefined{studyplanning.model.workflow.DataSet}}}
								\end{itemize}
							}%end item
						\end{itemize}
					}%end item
				\end{itemize}
			}
			
			\subsection{Methods}{
				\vskip -2em
				\begin{itemize}
					\item{ 
						\index{StudySubject!getDataSet()}
						\hypertarget{studyplanning.model.workflow.StudySubject.getDataSet()}{{\bf  getDataSet}\\}
						\begin{lstlisting}[frame=none]
public DataSet getDataSet()
						\end{lstlisting} %end signature
						\begin{itemize}
							\item{
								{\bf  Returns} -- The belonging \texttt{\small \hyperlink{studyplanning.model.workflow.DataSet}{DataSet}}{\small \refdefined{studyplanning.model.workflow.DataSet}} 
							}%end item
						\end{itemize}
					}%end item
					\item{ 
						\index{StudySubject!getMaxECTS()}
						\hypertarget{studyplanning.model.workflow.StudySubject.getMaxECTS()}{{\bf  getMaxECTS}\\}
						\begin{lstlisting}[frame=none]
public int getMaxECTS()
						\end{lstlisting} %end signature
						\begin{itemize}
							\item{
								{\bf  Returns} 
								-- The maximum allowed amount of ECTS points
							}%end item
						\end{itemize}
					}%end item
					\item{ 
						\index{StudySubject!getMaxStudyDuration()}
						\hypertarget{studyplanning.model.workflow.StudySubject.getMaxStudyDuration()}{{\bf  getMaxStudyDuration}\\}
						\begin{lstlisting}[frame=none]
public int getMaxStudyDuration()
						\end{lstlisting} %end signature
						\begin{itemize}
							\item{
								{\bf  Returns} 
								-- The maximum amount of semesters allowed to be studied 
							}%end item
						\end{itemize}
					}%end item
					\item{ 
						\index{StudySubject!getName()}
						\hypertarget{studyplanning.model.workflow.StudySubject.getName()}{{\bf  getName}\\}
						\begin{lstlisting}[frame=none]
public String getName()
						\end{lstlisting} %end signature
						\begin{itemize}
							\item{
								{\bf  Returns} 
								-- The name of this subject 
							}%end item
						\end{itemize}
					}%end item
					\item{ 
						\index{StudySubject!getRequiredECTS()}
						\hypertarget{studyplanning.model.workflow.StudySubject.getRequiredECTS()}{{\bf  getRequiredECTS}\\}
						\begin{lstlisting}[frame=none]
public int getRequiredECTS()
						\end{lstlisting} %end signature
						\begin{itemize}
							\item{
								{\bf  Returns} 
								-- The minimum amount of ECTS point to be reached 
							}%end item
						\end{itemize}
					}%end item
				\end{itemize}
			}
		}
		
		\section{\label{studyplanning.model.workflow.Workflow}Class \index{Workflow} Workflow}{
			\hypertarget{studyplanning.model.workflow.Workflow}{}\vskip .1in 
			Representation of the study plan.\vskip .1in 
			
			\subsection{Declaration}{
				\begin{lstlisting}[frame=none]
public final class Workflow
 implements java.lang.Iterable, java.lang.Cloneable
				 \end{lstlisting}
			}
			
			\subsection{Constructor summary}{
				\begin{verse}
					\hyperlink{studyplanning.model.workflow.Workflow(int)}{{\bf Workflow(int)}} Creates a new Workflow\\
				\end{verse}
			}
			
			\subsection{Method summary}{
				\begin{verse}
					\hyperlink{studyplanning.model.workflow.Workflow.clone()}{{\bf clone()}} \\
					\hyperlink{studyplanning.model.workflow.Workflow.getEctsPoints()}{{\bf getEctsPoints()}} Returns the sum of all ECTS points of all Module in this workflow.\\
					\hyperlink{studyplanning.model.workflow.Workflow.getSemester(int)}{{\bf getSemester(int)}} Returns the semester object for the given semester index.\\
					\hyperlink{studyplanning.model.workflow.Workflow.getSemester(studyplanning.model.workflow.Module)}{{\bf getSemester(Module)}} Returns the semester, the Module is in, in this specific Workflow or null, if a Module is not contained in this Workflow.\\
					\hyperlink{studyplanning.model.workflow.Workflow.iterator()}{{\bf iterator()}} Allows to iterate over all semesters in this Workflow.\\
				\end{verse}
			}
			
			\subsection{Constructors}{
				\vskip -2em
				\begin{itemize}
					\item{ 
						\index{Workflow!Workflow(int)}
						\hypertarget{studyplanning.model.workflow.Workflow(int)}{{\bf  Workflow}\\}
						\begin{lstlisting}[frame=none]
public Workflow(int maxSemester)
						\end{lstlisting} %end signature
						\begin{itemize}
							\item{
								{\bf  Description}
								Creates a new Workflow
							}
							\item{
								{\bf  Parameters}
								\begin{itemize}
									\item{
										\texttt{maxSemester} -- The maximal amount of semesters allowed by the StudySubject.}
								\end{itemize}
							}%end item
						\end{itemize}
					}%end item
				\end{itemize}
			}
			
			\subsection{Methods}{
				\vskip -2em
				\begin{itemize}
					\item{ 
						\index{Workflow!clone()}
						\hypertarget{studyplanning.model.workflow.Workflow.clone()}{{\bf  clone}\\}
						\begin{lstlisting}[frame=none]
protected native Workflow clone() throws java.lang.CloneNotSupportedException\end{lstlisting} %end signature
					}%end item
					\item{ 
						\index{Workflow!getEctsPoints()}	
						\hypertarget{studyplanning.model.workflow.Workflow.getEctsPoints()}{{\bf  getEctsPoints}\\}
						\begin{lstlisting}[frame=none]
public int getEctsPoints()
						\end{lstlisting} %end signature
						\begin{itemize}
							\item{
								{\bf  Description}
								Returns the sum of all ECTS points of all Modules in this Workflow.
							}
							\item{
								{\bf  Returns} 
								-- The amount of ECTS points in this Workflow. 
							}%end item
						\end{itemize}
					}%end item
					\item{ 
						\index{Workflow!getSemester(int)}
						\hypertarget{studyplanning.model.workflow.Workflow.getSemester(int)}{{\bf  getSemester}\\}
						\begin{lstlisting}[frame=none]
public Semester getSemester(int id)
						\end{lstlisting} %end signature
						\begin{itemize}
							\item{
								{\bf  Description}
								Returns the semester object for the given semester index.
							}
							\item{
								{\bf  Parameters}
								\begin{itemize}
									\item{\texttt{id} -- The ordinal of the looked Semester}
								\end{itemize}
							}%end item
							\item{
								{\bf  Returns} 
								-- The requested Semester 
							}%end item
						\end{itemize}
					}%end item
					\item{ 
						\index{Workflow!getSemester(Module)}
						\hypertarget{studyplanning.model.workflow.Workflow.getSemester(studyplanning.model.workflow.Module)}{{\bf  getSemester}\\}
						\begin{lstlisting}[frame=none]
public Semester getSemester(Module module)
						\end{lstlisting} %end signature
						\begin{itemize}
							\item{
								{\bf  Description}
								Returns the semester the Module is in, in this specific Workflow, or null if a Module is not contained in this Workflow.
							}
							\item{
								{\bf  Parameters}
								\begin{itemize}
									\item{\texttt{module} -- The Module to be looked for}
								\end{itemize}
							}%end item
							\item{
								{\bf  Returns} 
								-- The Semenster which contains this Module 
							}%end item
						\end{itemize}
					}%end item
					\item{ 
						\index{Workflow!iterator()}
						\hypertarget{studyplanning.model.workflow.Workflow.iterator()}{{\bf  iterator}\\}
						\begin{lstlisting}[frame=none]
public Iterator<Semester> iterator()
						\end{lstlisting} %end signature
						\begin{itemize}
							\item{
								{\bf  Description}
								Allows to iterate over all Semesters in this Workflow.
							}
						\end{itemize}
					}%end item
				\end{itemize}
			}
		}
		
		\section{\label{studyplanning.model.workflow.WorkflowTasks}Class \index{WorkflowTasks} WorkflowTasks}{
			\hypertarget{studyplanning.model.workflow.WorkflowTasks}{}\vskip .1in 
			This class is dedicated to verifying and generating workflows.\vskip .1in 
			
			\subsection{Declaration}{
				\begin{lstlisting}[frame=none]
public class WorkflowTasks
				\end{lstlisting}
			}
			
			\subsection{Method summary}{
				\begin{verse}
					\hyperlink{studyplanning.model.workflow.WorkflowTasks.generate(studyplanning.model.workflow.Workflow, studyplanning.model.workflow.StudySubject, studyplanning.model.workflow.generation.Preferences)}{{\bf generate(Workflow, StudySubject, Preferences)}} Generates a new Workflow based on a given Workflow, filling in missing Modules following all Constraints.\\
					\hyperlink{studyplanning.model.workflow.WorkflowTasks.verify(studyplanning.model.workflow.Workflow, studyplanning.model.workflow.StudySubject)}{{\bf verify(Workflow, StudySubject)}} Verifies a Workflow.\\
				\end{verse}
			}
			
			\subsection{Constructors}{
				\vskip -2em
				\begin{itemize}
					\item{
						\index{WorkflowTasks!WorkflowTasks()}
						\hypertarget{studyplanning.model.workflow.WorkflowTasks()}{{\bf  WorkflowTasks}\\}
						\begin{lstlisting}[frame=none]
public WorkflowTasks()
						\end{lstlisting} %end signature
					}%end item
				\end{itemize}
			}
			
			\subsection{Methods}{
				\vskip -2em
				\begin{itemize}
					\item{ 
						\index{WorkflowTasks!generate(Workflow, StudySubject, Preferences)}
						\hypertarget{studyplanning.model.workflow.WorkflowTasks.generate(studyplanning.model.workflow.Workflow, studyplanning.model.workflow.StudySubject, studyplanning.model.workflow.generation.Preferences)}{{\bf  generate}\\}
						\begin{lstlisting}[frame=none]
public static Collection<Mistake> generate(Workflow workflow, StudySubject subject, Preferences prefs)
						\end{lstlisting} %end signature
						\begin{itemize}
							\item{
								{\bf  Description}
								Generates a new Workflow based on a given Workflow, filling in missing Modules following all Constraints.\mbox{}\newline In case the given Workflow cannot be generated, because it already violates the module manual, all violations will be returned in a Collection.
							}
							\item{
								{\bf  Parameters}
								\begin{itemize}
									\item{\texttt{workflow} -- The Workflow to use as a base to generate.}
									\item{\texttt{subject} -- The StudySubject containing all demanded restrictions.}
									\item{\texttt{prefs} -- User preferences sent by the Controller.}
								\end{itemize}
							}%end item
							\item{
								{\bf  Returns} 
								-- A Collection of module manual violations. 
							}%end item
						\end{itemize}
					}%end item
					\item{ 
						\index{WorkflowTasks!verify(Workflow, StudySubject)}
						\hypertarget{studyplanning.model.workflow.WorkflowTasks.verify(studyplanning.model.workflow.Workflow, studyplanning.model.workflow.StudySubject)}{{\bf  verify}\\}
						\begin{lstlisting}[frame=none]
public static Collection<Mistake> verify(Workflow workflow, StudySubject subject)
						\end{lstlisting} %end signature
						\begin{itemize}
							\item{
								{\bf  Description}
								Verifies a Workflow.
							}
							\item{
								{\bf  Parameters}
								\begin{itemize}
									\item{\texttt{workflow} -- The Workflow to verify.}
									\item{\texttt{subject} -- The StudySubject to get the verification data from.}
								\end{itemize}
							}%end item
							\item{
								{\bf  Returns} 
								-- A Collection of module manual violations. 
							}%end item
						\end{itemize}
					}%end item
				\end{itemize}
			}
		}
	}
	
	\chapter{Package studyplanning.model.workflow.generation}{
		\label{studyplanning.model.workflow.generation}\hypertarget{studyplanning.model.workflow.generation}{}
		\hskip -.05in
		\hbox to \hsize{\textit{ Package Contents\hfil Page}}
		\vskip .13in
		\hbox{{\bf  Classes}}
		\entityintro{ModuleEvaluation}{studyplanning.model.workflow.generation.ModuleEvaluation}{An individual evaluation of all Modules in the workflow.}
		\entityintro{Preferences}{studyplanning.model.workflow.generation.Preferences}{Class representing all user input.}
		\vskip .1in
		\vskip .1in
		\section{\label{studyplanning.model.workflow.generation.ModuleEvaluation}Class \index{ModuleEvaluation} ModuleEvaluation}{
			\hypertarget{studyplanning.model.workflow.generation.ModuleEvaluation}{}\vskip .1in 
			An individual evaluation of all Modules in the Workflow giving the generation algorithm easy access to the values of a Module.\vskip .1in 
			
			\subsection{Declaration}{
				\begin{lstlisting}[frame=none]
public class ModuleEvaluation
				\end{lstlisting}
			}
			
			\subsection{Constructor summary}{
				\begin{verse}
					\hyperlink{studyplanning.model.workflow.generation.ModuleEvaluation()}{{\bf ModuleEvaluation()}} \\
				\end{verse}
			}
			
			\subsection{Method summary}{
				\begin{verse}
					\hyperlink{studyplanning.model.workflow.generation.ModuleEvaluation.addModuleValue(studyplanning.model.workflow.Module, int)}{{\bf addModuleValue(Module, int)}} Increases the value of a specified Module by the given parameter.\\
					\hyperlink{studyplanning.model.workflow.generation.ModuleEvaluation.getModuleValue(studyplanning.model.workflow.Module)}{{\bf getModuleValue(Module)}} Returns the value of the given Module.\\
					\hyperlink{studyplanning.model.workflow.generation.ModuleEvaluation.getValuedModules()}{{\bf getValuedModules()}} Returns an iterator of all Modules evaluated here in their valued order (from highest value to lowest).\\
					\hyperlink{studyplanning.model.workflow.generation.ModuleEvaluation.setModuleValue(studyplanning.model.workflow.Module, int)}{{\bf setModuleValue(Module, int)}} Evaluates the given Module to the specified value.\\
				\end{verse}
			}
			
			\subsection{Constructors}{
				\vskip -2em
				\begin{itemize}
					\item{ 
						\index{ModuleEvaluation!ModuleEvaluation()}
						\hypertarget{studyplanning.model.workflow.generation.ModuleEvaluation()}{{\bf  ModuleEvaluation}\\}
						\begin{lstlisting}[frame=none]
public ModuleEvaluation()
						\end{lstlisting} %end signature
					}%end item
				\end{itemize}
			}
			
			\subsection{Methods}{
				\vskip -2em
				\begin{itemize}
					\item{ 
						\index{ModuleEvaluation!addModuleValue(Module, int)}
						\hypertarget{studyplanning.model.workflow.generation.ModuleEvaluation.addModuleValue(studyplanning.model.workflow.Module, int)}{{\bf  addModuleValue}\\}
						\begin{lstlisting}[frame=none]
public void addModuleValue(Module module, int value)
						\end{lstlisting} %end signature
						\begin{itemize}
							\item{
								{\bf  Description}
								Increases the value of a specified Module by the given parameter.
							}
							\item{
								{\bf  Parameters}
								\begin{itemize}
									\item{\texttt{module} -- The Module to evaluate.}
									\item{\texttt{value} -- The value to add}
								\end{itemize}
							}%end item
						\end{itemize}
					}%end item
					\item{ 
						\index{ModuleEvaluation!getModuleValue(Module)}
						\hypertarget{studyplanning.model.workflow.generation.ModuleEvaluation.getModuleValue(studyplanning.model.workflow.Module)}{{\bf  getModuleValue}\\}
						\begin{lstlisting}[frame=none]
public int getModuleValue(Module module)
						\end{lstlisting} %end signature
						\begin{itemize}
							\item{
								{\bf  Description}
								Returns the value of the given Module.
							}
							\item{
								{\bf  Parameters}
								\begin{itemize}
									\item{\texttt{module} -- The Module to get the value from.}
								\end{itemize}
							}%end item
							\item{
								{\bf  Returns} 
								-- The current value of the Module. 
							}%end item
						\end{itemize}
					}%end item
					\item{ 
						\index{ModuleEvaluation!getValuedModules()}
						\hypertarget{studyplanning.model.workflow.generation.ModuleEvaluation.getValuedModules()}{{\bf  getValuedModules}\\}
						\begin{lstlisting}[frame=none]
public Iterator<Module> getValuedModules()
						\end{lstlisting} %end signature
						\begin{itemize}
							\item{
								{\bf  Description}
								Returns an iterator of the all Modules evaluated here in their valued order (from highest value to lowest).
							}
							\item{
								{\bf  Returns} 
								-- An iterator of all evaluated modules. 
							}%end item
						\end{itemize}
					}%end item
					\item{ 
						\index{ModuleEvaluation!setModuleValue(Module, int)}
						\hypertarget{studyplanning.model.workflow.generation.ModuleEvaluation.setModuleValue(studyplanning.model.workflow.Module, int)}{{\bf  setModuleValue}\\}
						\begin{lstlisting}[frame=none]
public void setModuleValue(Module module, int value)
						\end{lstlisting} %end signature
						\begin{itemize}
							\item{
								{\bf  Description}
								Evaluates the given Module to the specified value.
							}
							\item{
								{\bf  Parameters}
								\begin{itemize}
									\item{\texttt{module} -- The Module to evaluate}
									\item{\texttt{value} -- The value to set.}
								\end{itemize}
							}%end item
						\end{itemize}
					}%end item
				\end{itemize}
			}
		}
		
		\section{\label{studyplanning.model.workflow.generation.Preferences}Class \index{Preferences} Preferences}{
			\hypertarget{studyplanning.model.workflow.generation.Preferences}{}\vskip .1in 
			Class representing all user input. Will be used for generating the Workflow.\vskip .1in 
			
			\subsection{Declaration}{
				\begin{lstlisting}[frame=none]
public class Preferences
				\end{lstlisting}
			}
			
			\subsection{Constructor summary}{
				\begin{verse}
					\hyperlink{studyplanning.model.workflow.generation.Preferences()}{{\bf Preferences()}} \\
				\end{verse}
			}
			
			\subsection{Method summary}{
				\begin{verse}
					\hyperlink{studyplanning.model.workflow.generation.Preferences.currentSemester()}{{\bf currentSemester()}} Returns the semester the user is starting in.\\
					\hyperlink{studyplanning.model.workflow.generation.Preferences.getModulePreferences()}{{\bf getModulePreferences()}} Returns a Collection of \texttt{\small \hyperlink{studyplanning.model.workflow.generation.preference.ModulePreference}{ModulePreference}}{\small 
					\refdefined{studyplanning.model.workflow.generation.preference.ModulePreference}}, which can be weighted for Workflow generation.\\
					\hyperlink{studyplanning.model.workflow.generation.Preferences.getPreferredECTSPerSemester()}{{\bf getPreferredECTSPerSemester()}} Returns the preferred amount of ECTS points per semester.\\
					\hyperlink{studyplanning.model.workflow.generation.Preferences.getPreferredStudyDuration()}{{\bf getPreferredStudyDuration()}} Returns the preferred study duration.\\
				\end{verse}
			}
			
			\subsection{Constructors}{
				\vskip -2em
				\begin{itemize}
					\item{ 
						\index{Preferences!Preferences()}
						 \hypertarget{studyplanning.model.workflow.generation.Preferences()}{{\bf  Preferences}\\}
						\begin{lstlisting}[frame=none]
public Preferences()
						\end{lstlisting} %end signature
						\begin{itemize}
							\item{
								{\bf Description}
								Creates a new empty instance. For a more dynamic dealing with user inputs, this class does not take any parameters in its constructor.
							}
						\end{itemize}
					}%end item
				\end{itemize}
			}
			
			\subsection{Methods}{
				\vskip -2em
				\begin{itemize}
					\item{ 
						\index{Preferences!currentSemester()}
						\hypertarget{studyplanning.model.workflow.generation.Preferences.currentSemester()}{{\bf  currentSemester}\\}
						\begin{lstlisting}[frame=none]
public int currentSemester()
						\end{lstlisting} %end signature
						\begin{itemize}
							\item{
								{\bf  Description}
								Returns the semester the user is starting in. The generation won't add anything to Semesters with a lower index than this number.
							}
							\item{
								{\bf  Returns}
								-- The current semester the user is in. 
							}%end item
						\end{itemize}
					}%end item
					\item{ 
						\index{Preferences!getModulePreferences()}
						\hypertarget{studyplanning.model.workflow.generation.Preferences.getModulePreferences()}{{\bf  getModulePreferences}\\}
						\begin{lstlisting}[frame=none]
public Collection<ModulePreference> getModulePreferences()
						\end{lstlisting} %end signature
						\begin{itemize}
							\item{
								{\bf  Description}
								Returns a Collection of \texttt{\small \hyperlink{studyplanning.model.workflow.generation.preference.ModulePreference}{ModulePreference}}{\small 
								\refdefined{studyplanning.model.workflow.generation.preference.ModulePreference}}, which can be weighted for Workflow generation.
							}
							\item{
								{\bf  Returns}
								 -- A Collection of individual preferences. 
							}%end item
						\end{itemize}
					}%end item
					\item{ 
						\index{Preferences!getPreferredECTSPerSemester()}
						\hypertarget{studyplanning.model.workflow.generation.Preferences.getPreferredECTSPerSemester()}{{\bf  getPreferredECTSPerSemester}\\}
						\begin{lstlisting}[frame=none]
public int getPreferredECTSPerSemester()
						\end{lstlisting} %end signature
						\begin{itemize}
							\item{
								{\bf  Description}
								Returns the preferred amount of ECTS per semester.
							}
							\item{
								{\bf  Returns} 
								-- The preferred amount of ECTS per semester. 
							}%end item
						\end{itemize}
					}%end item
					\item{ 
						\index{Preferences!getPreferredStudyDuration()}
						\hypertarget{studyplanning.model.workflow.generation.Preferences.getPreferredStudyDuration()}{{\bf  getPreferredStudyDuration}\\}
						\begin{lstlisting}[frame=none]
public int getPreferredStudyDuration()
						\end{lstlisting} %end signature
						\begin{itemize}
							\item{
								{\bf  Description}
								Returns the preferred study duration.
							}
							\item{
								{\bf  Returns} 
								-- The preferred study duration. 
							}%end item
						\end{itemize}
					}%end item
				\end{itemize}
			}
		}
	}
	
	\chapter{Package studyplanning.model.workflow.generation.preference}{
		\label{studyplanning.model.workflow.generation.preference}\hypertarget{studyplanning.model.workflow.generation.preference}{}
		\hskip -.05in
		\hbox to \hsize{\textit{ Package Contents\hfil Page}}
		\vskip .13in
		\hbox{{\bf  Classes}}
		\entityintro{ChoiceCategoryPreference}{studyplanning.model.workflow.generation.preference.ChoiceCategoryPreference}{This class is valuing Modules of a category.}
		\entityintro{ChoiceModulePreference}{studyplanning.model.workflow.generation.preference.ChoiceModulePreference}{This class is valuing a given Module.}
		\entityintro{ModulePreference}{studyplanning.model.workflow.generation.preference.ModulePreference}{To allow a class to value the DataSet of the Modules, it must inherit this class.}
		\vskip .1in
		\vskip .1in
		
		\section{\label{studyplanning.model.workflow.generation.preference.ChoiceCategoryPreference}Class \index{ChoiceCategoryPreference} ChoiceCategoryPreference}{
			\hypertarget{studyplanning.model.workflow.generation.preference.ChoiceCategoryPreference}{}\vskip .1in 
			This class is valuing modules of a given category.\vskip .1in 

			\subsection{Declaration}{
				\begin{lstlisting}[frame=none]
public class ChoiceCategoryPreference
 extends studyplanning.model.workflow.generation.preference.ModulePreference
				\end{lstlisting}
			}
			
			\subsection{Constructor summary}{
				\begin{verse}
					\hyperlink{studyplanning.model.workflow.generation.preference.ChoiceCategoryPreference(studyplanning.model.workflow.DataSet, java.lang.String)}{{\bf ChoiceCategoryPreference(DataSet, String)}} \\
				\end{verse}
			}
			
			\subsection{Method summary}{
				\begin{verse}
					\hyperlink{studyplanning.model.workflow.generation.preference.ChoiceCategoryPreference.evaluate(studyplanning.model.workflow.generation.ModuleEvaluation)}{{\bf evaluate(ModuleEvaluation)}} \\
				\end{verse}
			}
			
			\subsection{Constructors}{
				\vskip -2em
				\begin{itemize}
					\item{ 
						\index{ChoiceCategoryPreference!ChoiceCategoryPreference(DataSet, String)}
						\hypertarget{studyplanning.model.workflow.generation.preference.ChoiceCategoryPreference(studyplanning.model.workflow.DataSet, java.lang.String)}{{\bf  ChoiceCategoryPreference}\\}
						\begin{lstlisting}[frame=none]
public ChoiceCategoryPreference(DataSet dataSet, String key)
						\end{lstlisting} %end signature
						\begin{itemize}
							\item{
								{\bf Description}
								Instantiates the object with all modules of a StudySubject and a key. The instance will only evaluate Modules matching the given key.
							}
							\item{
								{\bf Parameters}
								\begin{itemize}
									\item{\texttt{dataSet} -- The container holding all modules of a StudySubject}
									\item{\texttt{key} -- The key to be considered in the evaluation of the DataSet}
								\end{itemize}
								}
						\end{itemize}
					}%end item
				\end{itemize}
			}
			
			\subsection{Methods}{
				\vskip -2em
				\begin{itemize}
					\item{ 
						\index{ChoiceCategoryPreference!evaluate(ModuleEvaluation)}
						\hypertarget{studyplanning.model.workflow.generation.preference.ChoiceCategoryPreference.evaluate(studyplanning.model.workflow.generation.ModuleEvaluation)}{{\bf  evaluate}\\}
						\begin{lstlisting}[frame=none]
public abstract void evaluate(ModuleEvaluation eval)
						\end{lstlisting} %end signature
						\begin{itemize}
							\item{
								{\bf  Description copied from \hyperlink{studyplanning.model.workflow.generation.preference.ModulePreference}{ModulePreference}{\small \refdefined{studyplanning.model.workflow.generation.preference.ModulePreference}} }
								Adds all Modules this preference affects to the given ModuleEvaluation.
							}
							\item{
								{\bf  Parameters}
								\begin{itemize}
									\item{\texttt{eval} -- The ModuleEvaluation to do the evaluation with.}
								\end{itemize}
							}%end item
						\end{itemize}
					}%end item
				\end{itemize}
			}
			
			\subsection{Members inherited from class ModulePreference }{
				\texttt{studyplanning.model.workflow.generation.preference.ModulePreference} {\small 
				\refdefined{studyplanning.model.workflow.generation.preference.ModulePreference}}{
					\small 
					\vskip -2em
					\begin{itemize}
						\item{
							\vskip -1.5ex 
							\texttt{public abstract void {\bf  evaluate}(\texttt{studyplanning.model.workflow.generation.ModuleEvaluation} {\bf  eval})}%end signature
						}%end item
					\end{itemize}
				}
			}
			
		\section{\label{studyplanning.model.workflow.generation.preference.ChoiceModulePreference}Class \index{ChoiceModulePreference} ChoiceModulePreference}{
			\hypertarget{studyplanning.model.workflow.generation.preference.ChoiceModulePreference}{}\vskip .1in 
			This class is valuing a given Module.\vskip .1in 
			
			\subsection{Declaration}{
				\begin{lstlisting}[frame=none]
public class ChoiceModulePreference
 extends studyplanning.model.workflow.generation.preference.ModulePreference
				\end{lstlisting}
			}

			\subsection{Constructor summary}{
				\begin{verse}
					\hyperlink{studyplanning.model.workflow.generation.preference.ChoiceModulePreference(studyplanning.model.workflow.DataSet, studyplanning.model.workflow.Module)}{{\bf ChoiceModulePreference(DataSet, Module)}} \\
				\end{verse}
			}
			
			\subsection{Method summary}{
				\begin{verse}
					\hyperlink{studyplanning.model.workflow.generation.preference.ChoiceModulePreference.evaluate(studyplanning.model.workflow.generation.ModuleEvaluation)}{{\bf evaluate(ModuleEvaluation)}} \\
				\end{verse}
			}
			
			\subsection{Constructors}{
				\vskip -2em
				\begin{itemize}
					\item{ 
						\index{ChoiceModulePreference!ChoiceModulePreference(DataSet, Module)}
						\hypertarget{studyplanning.model.workflow.generation.preference.ChoiceModulePreference(studyplanning.model.workflow.DataSet, studyplanning.model.workflow.Module)}{{\bf  ChoiceModulePreference}\\}
						\begin{lstlisting}[frame=none]
public ChoiceModulePreference(DataSet dataSet, Module module)
						\end{lstlisting} %end signature
						\begin{itemize}
							\item{
								{\bf Description}
								Instantiates an object with all Modules of a StudySubject and a key. The instance will only evaluate Modules matching the given key.
							}
							\item{
								{\bf Parameters}
								\begin{itemize}
									\item{\texttt{dataSet} -- The container holding all Modules of a StudySubject}
									\item{\texttt{module} -- The Module to be considered in the evaluation}
								\end{itemize}
							}
						\end{itemize}
					}%end item
				\end{itemize}
			}
			
			\subsection{Methods}{
				\vskip -2em
				\begin{itemize}
					\item{ 
						\index{ChoiceModulePreference!evaluate(ModuleEvaluation)}
						\hypertarget{studyplanning.model.workflow.generation.preference.ChoiceModulePreference.evaluate(studyplanning.model.workflow.generation.ModuleEvaluation)}{{\bf  evaluate}\\}
						\begin{lstlisting}[frame=none]
public abstract void evaluate(ModuleEvaluation eval)
						\end{lstlisting} %end signature
						\begin{itemize}
							\item{
								{\bf  Description copied from \hyperlink{studyplanning.model.workflow.generation.preference.ModulePreference}{ModulePreference}{\small \refdefined{studyplanning.model.workflow.generation.preference.ModulePreference}} }
								Adds all Modules affected by this preference to the given ModuleEvaluation.
							}
							\item{
								{\bf  Parameters}
								\begin{itemize}
									\item{\texttt{eval} -- The ModuleEvaluation to do the evaluation with.}
								\end{itemize}
							}%end item
						\end{itemize}
					}%end item
				\end{itemize}
			}
			
			\subsection{Members inherited from class ModulePreference }{
				\texttt{studyplanning.model.workflow.generation.preference.ModulePreference} {\small 
				\refdefined{studyplanning.model.workflow.generation.preference.ModulePreference}}{
					\small 
					\vskip -2em
					\begin{itemize}
						\item{
							\vskip -1.5ex 
							\texttt{public abstract void {\bf  evaluate}(\texttt{studyplanning.model.workflow.generation.ModuleEvaluation} {\bf  eval})}%end signature
						}%end item
					\end{itemize}
				}
			}
		}
			
		\section{\label{studyplanning.model.workflow.generation.preference.ModulePreference}Class \index{ModulePreference} ModulePreference}{
			\hypertarget{studyplanning.model.workflow.generation.preference.ModulePreference}{}\vskip .1in 
			To enable a class to value the DataSet of the modules, it must inherit this class.\vskip .1in 
			
			\subsection{Declaration}{
				\begin{lstlisting}[frame=none]
public abstract class ModulePreference
				\end{lstlisting}
			}
			
			\subsection{All known subclasses}{
				{ChoiceModulePreference\small{\refdefined{studyplanning.model.workflow.generation.preference.ChoiceModulePreference}}, ChoiceCategoryPreference\small{\refdefined{studyplanning.model.workflow.generation.preference.ChoiceCategoryPreference}}}
			}
			
			\subsection{Method summary}{
				\begin{verse}
					\hyperlink{studyplanning.model.workflow.generation.preference.ModulePreference.evaluate(studyplanning.model.workflow.generation.ModuleEvaluation)}{{\bf evaluate(ModuleEvaluation)}} Adds all modules affected by this preference to the given ModuleEvaluation. \\
				\end{verse}
			}
			
			\subsection{Methods}{
				\vskip -2em
				\begin{itemize}
					\item{ 
						\index{ModulePreference!evaluate(ModuleEvaluation)}
						\hypertarget{studyplanning.model.workflow.generation.preference.ModulePreference.evaluate(studyplanning.model.workflow.generation.ModuleEvaluation)}{{\bf  evaluate}\\}
						\begin{lstlisting}[frame=none]
public abstract void evaluate(ModuleEvaluation eval)
						\end{lstlisting} %end signature
						\begin{itemize}
							\item{
								{\bf  Description}
								Adds all modules affected by this preference to the given ModuleEvaluation.
							}
							\item{
								{\bf  Parameters}
								\begin{itemize}
									\item{\texttt{eval} -- The ModuleEvaluation to do the evaluation with.}
								\end{itemize}
							}%end item
						\end{itemize}
					}%end item
				\end{itemize}
			}
		}
	}
}

\part{View}{
	
	\begin{figure}[ht]
		\centering
		\includegraphics[width=\textwidth, angle=90]{res/ClassDiagram/viewMarkedCompressed}
		\caption{Model}
	\end{figure}
	
	\chapter{Package studyplanning.view}{
		\label{studyplanning.view}\hypertarget{studyplanning.view}{}
		\hskip -.05in
		\hbox to \hsize{\textit{ Package Contents\hfil Page}}
		\vskip .13in
		\hbox{{\bf  Classes}}
		\entityintro{HTMLBuilder}{studyplanning.view.HTMLBuilder}{Returns HTML coded objects}
		\entityintro{JavaScriptBuilder}{studyplanning.view.JavaScriptBuilder}{Returns JavaScript functionality needed by HTMLBuilder.}
		\entityintro{Languages}{studyplanning.view.Languages}{Class responsible for loading and retrieving \texttt{\small \hyperlink{studyplanning.view.Locale}{Locale}}{\small 
				\refdefined{studyplanning.view.Locale}} objects.}
		\entityintro{Locale}{studyplanning.view.Locale}{Class representing one language.}
		\entityintro{ViewBuilder}{studyplanning.view.ViewBuilder}{Class interacting with Apache Tomcat.}
		\vskip .1in
		\vskip .1in
	
		\section{\label{studyplanning.view.HTMLBuilder}Class \index{HTMLBuilder}HTMLBuilder}{
			\hypertarget{studyplanning.view.HTMLBuilder}{}\vskip .1in 
			Returns HTML coded objects\vskip .1in 

			\subsection{Declaration}{
				\begin{lstlisting}[frame=none]
public class HTMLBuilder
				\end{lstlisting}
			}
		
			\subsection{Constructor summary}{
				\begin{verse}
					\hyperlink{studyplanning.view.HTMLBuilder()}{{\bf HTMLBuilder()}} \\
				\end{verse}
			}
			
			\subsection{Method summary}{
				\begin{verse}
					\hyperlink{studyplanning.view.HTMLBuilder.printBottomSection(java.lang.String)}{{\bf printBottomSection(String)}} Returns bottom section as HTML code string\\
					\hyperlink{studyplanning.view.HTMLBuilder.printModules(studyplanning.model.workflow.DataSet)}{{\bf printModules(DataSet)}} Returns left from section as HTML code string\\
					\hyperlink{studyplanning.view.HTMLBuilder.printPreferences()}{{\bf printPreferences()}} Returns the default preferences as HTML code string \\
					\hyperlink{studyplanning.view.HTMLBuilder.printWorkflow(studyplanning.model.workflow.Workflow)}{{\bf printWorkflow(Workflow)}} Returns right Workflow section as HTML code string\\
				\end{verse}
			}
		
			\subsection{Constructors}{
				\vskip -2em
				\begin{itemize}
					\item{ 
						\index{HTMLBuilder!HTMLBuilder()}
						\hypertarget{studyplanning.view.HTMLBuilder()}{{\bf  HTMLBuilder}\\}
						\begin{lstlisting}[frame=none]
public HTMLBuilder()
						\end{lstlisting} %end signature
					}%end item
				\end{itemize}
			}
		
			\subsection{Methods}{
				\vskip -2em
				\begin{itemize}
					\item{ 
						\index{HTMLBuilder!printBottomSection(String)}
						\hypertarget{studyplanning.view.HTMLBuilder.printBottomSection(java.lang.String)}{{\bf  printBottomSection}\\}
						\begin{lstlisting}[frame=none]
public String printBottomSection(String workflowID)
						\end{lstlisting} %end signature
						\begin{itemize}
							\item{
								{\bf  Description}
								Returns bottom section as HTML code string
							}
							\item{
								{\bf  Parameters}
								\begin{itemize}
									\item{\texttt{workflowID} -- Workflow id to display}
								\end{itemize}
							}%end item
							\item{
								{\bf  Returns} 
								-- HTML result
							}%end item
						\end{itemize}
					}%end item
					\item{ 
						\index{HTMLBuilder!printModules(DataSet)}
						\hypertarget{studyplanning.view.HTMLBuilder.printModules(studyplanning.model.workflow.DataSet)}{{\bf  printModules}\\}
						\begin{lstlisting}[frame=none]
public String printModules(DataSet modules)
						\end{lstlisting} %end signature
						\begin{itemize}
							\item{
								{\bf  Description}
								Returns left from section as HTML code string
							}
							\item{
								{\bf  Parameters}
								\begin{itemize}
									\item{\texttt{modules} -- Modules to display}
								\end{itemize}
							}%end item
							\item{
								{\bf  Returns} -- 
								HTML result 
							}%end item
						\end{itemize}
					}%end item
					\item{ 
						\index{HTMLBuilder!printPreferences()}
						\hypertarget{studyplanning.view.HTMLBuilder.printPreferences()}{{\bf  printPreferences}\\}
						\begin{lstlisting}[frame=none]
public String printPreferences()
						\end{lstlisting} %end signature
						\begin{itemize}
							\item{
								{\bf Description}
								Returns the default preferences as HTML code string
							}
							\item{
								{\bf Returns} -- HTML result
							}
						\end{itemize}
					}%end item
					\item{ 
						\index{HTMLBuilder!printWorkflow(Workflow)}
						\hypertarget{studyplanning.view.HTMLBuilder.printWorkflow(studyplanning.model.workflow.Workflow)}{{\bf  printWorkflow}\\}
						\begin{lstlisting}[frame=none]
public String printWorkflow(Workflow workflow)
						\end{lstlisting} %end signature
						\begin{itemize}
							\item{
								{\bf  Description}
								Returns left from section as HTML code string
							}
							\item{
								{\bf  Parameters}
								\begin{itemize}
									\item{\texttt{workflow} -- Workflow to be displayed}
								\end{itemize}
							}%end item
							\item{
								{\bf  Returns} 
								-- HTML result 
							}%end item
						\end{itemize}
					}%end item
				\end{itemize}
			}
		}
		
		\section{\label{studyplanning.view.JavaScriptBuilder}Class \index{JavaScriptBuilder} JavaScriptBuilder}{
			\hypertarget{studyplanning.view.JavaScriptBuilder}{}\vskip .1in 
			Returns JavaScript functionality needed by HTMLBuilder.\vskip .1in 
			
			\subsection{Declaration}{
				\begin{lstlisting}[frame=none]
public class JavaScriptBuilder
				\end{lstlisting}
			}	
				
			\subsection{Constructor summary}{
				\begin{verse}
					\hyperlink{studyplanning.view.JavaScriptBuilder()}{{\bf JavaScriptBuilder()}} \\
				\end{verse}
			}
				
			\subsection{Method summary}{
				\begin{verse}
					\hyperlink{studyplanning.view.JavaScriptBuilder.printAddListModule()}{{\bf printAddListModule()}} Returns a JavaScript method that calls 'ViewBuilder' that a Module is added.\\
					\hyperlink{studyplanning.view.JavaScriptBuilder.printChangeCurrentWorkflow()}{{\bf printChangeCurrentWorkflow()}} Returns a JavaScript method that calls 'ViewBuilder' to change the current Workflow.\\
					\hyperlink{studyplanning.view.JavaScriptBuilder.printChangeLanguage()}{{\bf printChangeLanguage()}} Returns a JavaScript method that calls 'ViewBuilder' to change current language.\\
					\hyperlink{studyplanning.view.JavaScriptBuilder.printExportWorkflow()}{{\bf printExportWorkflow()}} Returns a JavaScript method that calls 'ViewBuilder' to request an export of the current Workflow. Please add 'onclick = "exportWorkflow()"' to your HTML to activate this generated method.\\
					\hyperlink{studyplanning.view.JavaScriptBuilder.printGenerationCall()}{{\bf printGenerationCall()}} Returns a JavaScript method that calls 'ViewBuilder' to process a new Workflow with given preferences. Please add 'onclick = "generateWorkflow('preferences')"' to your HTML to activate this generated method.\\
					\hyperlink{studyplanning.view.JavaScriptBuilder.printGlobalVariables(java.util.UUID, java.lang.String)}{{\bf printGlobalVariables(UUID, String)}} Returns JavaScript code containing global variables needed by other generated methods.\\
					\hyperlink{studyplanning.view.JavaScriptBuilder.printHeardModule()}{{\bf printHeardModule()}} Returns a JavaScript method that calls 'ViewBuilder' to mark a from as heard.\\
					\hyperlink{studyplanning.view.JavaScriptBuilder.printRemoveModule()}{{\bf printRemoveModule()}} Returns a JavaScript method that calls 'ViewBuilder' to remove a from from the current Workflow.\\
				\end{verse}
			}
			
			\subsection{Constructors}{
				\vskip -2em
				\begin{itemize}
					\item{ 
						\index{JavaScriptBuilder!JavaScriptBuilder()}
						\hypertarget{studyplanning.view.JavaScriptBuilder()}{{\bf  JavaScriptBuilder}\\}
						\begin{lstlisting}[frame=none]
public JavaScriptBuilder()
						\end{lstlisting} %end signature
					}%end item
				\end{itemize}
			}
			
			\subsection{Methods}{
				\vskip -2em
				\begin{itemize}
					\item{ 
						\index{JavaScriptBuilder!printAddListModule()}
						\hypertarget{studyplanning.view.JavaScriptBuilder.printAddListModule()}{{\bf  printAddListModule}\\}
						\begin{lstlisting}[frame=none]
public static String printAddListModule()
						\end{lstlisting} %end signature
						\begin{itemize}
							\item{
								{\bf  Description}
								Returns a JavaScript method that calls 'ViewBuilder' if a Module is added. Please add 'onclick = "addListModule('moduleAcronym')"' to your HTML to activate this generated method. Exp.: onclick = "addListModule("SWT1")"
							}
							\item{
								{\bf  Returns} -- JavaScript event method 
							}%end item
						\end{itemize}
					}%end item
					\item{ 
						\index{JavaScriptBuilder!printChangeCurrentWorkflow()}
						\hypertarget{studyplanning.view.JavaScriptBuilder.printChangeCurrentWorkflow()}{{\bf  printChangeCurrentWorkflow}\\}
						\begin{lstlisting}[frame=none]
public static String printChangeCurrentWorkflow()
						\end{lstlisting} %end signature
						\begin{itemize}
							\item{
								{\bf  Description}
								Returns a JavaScript method that calls 'ViewBuilder' to change the current Workflow. Please add 'onclick = "changeWorkflowTo('workflowID')"' to your HTML to activate this generated method.
							}
							\item{
								{\bf  Returns} -- JavaScript event method 
							}%end item
						\end{itemize}
					}%end item
					\item{ 
						\index{JavaScriptBuilder!printChangeLanguage()}
						\hypertarget{studyplanning.view.JavaScriptBuilder.printChangeLanguage()}{{\bf  printChangeLanguage}\\}
						\begin{lstlisting}[frame=none]
public static String printChangeLanguage()
						\end{lstlisting} %end signature
						\begin{itemize}
							\item{
								{\bf  Description}
								Returns a JavaScript method that calls 'ViewBuilder' to change current language. Please add 'onclick = "changeLanguageTo('acronym')"' to your HTML to activate this generated method. Exp.: onclick = "changeLanguageTo("ger\_ger")"
							}
							\item{
								{\bf  Returns} -- JavaScript event method 
							}%end item
						\end{itemize}
					}%end item
					\item{ 
						\index{JavaScriptBuilder!printExportWorkflow()}
						\hypertarget{studyplanning.view.JavaScriptBuilder.printExportWorkflow()}{{\bf  printExportWorkflow}\\}
						\begin{lstlisting}[frame=none]
public static String printExportWorkflow()
						\end{lstlisting} %end signature
						\begin{itemize}
							\item{
								{\bf  Description}
								Returns a JavaScript method that calls 'ViewBuilder' to request an export of the current Workflow. Please add 'onclick = "exportWorkflow()"' to your HTML to activate this generated method.
							}
							\item{
								{\bf  Returns} JavaScript event method 
							}%end item
						\end{itemize}
					}%end item
					\item{ 
						\index{JavaScriptBuilder!printGenerationCall()}
						\hypertarget{studyplanning.view.JavaScriptBuilder.printGenerationCall()}{{\bf  printGenerationCall}\\}
						\begin{lstlisting}[frame=none]
public static String printGenerationCall()
						\end{lstlisting} %end signature
						\begin{itemize}
							\item{
								{\bf  Description}
								Returns a JavaScript method that calls 'ViewBuilder' to process a new Workflow with given preferences. Please add 'onclick = "generateWorkflow('preferences')"' to your HTML to activate this generated method.
							}
							\item{
								{\bf  Returns} -- JavaScript event method 
							}%end item
						\end{itemize}
					}%end item
					\item{ 
						\index{JavaScriptBuilder!printGlobalVariables(UUID, String)}
						\hypertarget{studyplanning.view.JavaScriptBuilder.printGlobalVariables(java.util.UUID, java.lang.String)}{{\bf  printGlobalVariables}\\}
						\begin{lstlisting}[frame=none]
public static String printGlobalVariables(UUID workflowID, String languageAcronym)
						\end{lstlisting} %end signature
						\begin{itemize}
							\item{
								{\bf  Description}
								Returns JavaScript code containing global variables needed by other generated methods. Always run this first before calling other 'JavaScriptBuilder' methods!
							}
							\item{
								{\bf  Parameters}
								\begin{itemize}
									\item{\texttt{workflowID} -- Users current Workflow}
									\item{\texttt{languageAcronym} -- Users current locale}
								\end{itemize}
							}%end item
							\item{
								{\bf  Returns} -- JavaScript declarations 
							}%end item
						\end{itemize}
					}%end item
					\item{
						\index{JavaScriptBuilder!printHeardModule()}
						\hypertarget{studyplanning.view.JavaScriptBuilder.printHeardModule()}{{\bf  printHeardModule}\\}
						\begin{lstlisting}[frame=none]
public static String printHeardModule()
						\end{lstlisting} %end signature
						\begin{itemize}
							\item{
								{\bf  Description}
								Returns a JavaScript method that calls 'ViewBuilder' to mark a Module as heard. Please add 'onclick = "heardModule('acronym')"' to your HTML to activate this generated method.
							}
							\item{
								{\bf  Returns} 
								-- JavaScript event method 
							}%end item
						\end{itemize}
					}%end item
					\item{ 
						\index{JavaScriptBuilder!printRemoveModule()}
						\hypertarget{studyplanning.view.JavaScriptBuilder.printRemoveModule()}{{\bf  printRemoveModule}\\}
						\begin{lstlisting}[frame=none]
public static String printRemoveModule()
						\end{lstlisting} %end signature
						\begin{itemize}
							\item{
								{\bf  Description}
								Returns a JavaScript method that calls 'ViewBuilder' to remove a Module from the current Workflow. Please add 'onclick = "removeModule('acronym')"' to your HTML to activate this generated method.
							}
							\item{
								{\bf  Returns} 
								-- JavaScript event method 
							}%end item
						\end{itemize}
					}%end item
				\end{itemize}
			}
		}
			
		\section{\label{studyplanning.view.Languages}Class \index{Languages} Languages}{
			\hypertarget{studyplanning.view.Languages}{}\vskip .1in 
			Class responsible for loading and retrieving \texttt{\small \hyperlink{studyplanning.view.Locale}{Locale}}{\small 
				\refdefined{studyplanning.view.Locale}} objects.\vskip .1in 
				
			\subsection{Declaration}{
				\begin{lstlisting}[frame=none]
public class Languages
				\end{lstlisting}
			}
			
			\subsection{Constructor summary}{
				\begin{verse}
					\hyperlink{studyplanning.view.Languages()}{{\bf Languages()}} Creates a new Language instance.\\
				\end{verse}
			}
			
			\subsection{Method summary}{
				\begin{verse}
					\hyperlink{studyplanning.view.Languages.getLocalByAcronym(java.lang.String)}{{\bf getLocalByAcronym(String)}} Returns a localization set with given language acronym 'localAcronym' Returns null on unknown acronym input.\\
					\hyperlink{studyplanning.view.Languages.reload()}{{\bf reload()}} Reloads all locales from the base locale directory \\
				\end{verse}
			}
			
			\subsection{Constructors}{
				\vskip -2em
				\begin{itemize}
					\item{ 
						\index{Languages!Languages()}
						\hypertarget{studyplanning.view.Languages()}{{\bf  Languages}\\}
						\begin{lstlisting}[frame=none]
public Languages()
						\end{lstlisting} %end signature
						\begin{itemize}
							\item{
								{\bf  Description}
								Creates a new Language instance.
							}
						\end{itemize}
					}%end item
				\end{itemize}
			}
			
			\subsection{Methods}{
				\vskip -2em
				\begin{itemize}
					\item{ 
						\index{Languages!getLocalByAcronym(String)}
						\hypertarget{studyplanning.view.Languages.getLocalByAcronym(java.lang.String)}{{\bf  getLocalByAcronym}\\}
						\begin{lstlisting}[frame=none]
public Locale getLocalByAcronym(String localAcronym)
						\end{lstlisting} %end signature
						\begin{itemize}
							\item{
								{\bf  Description}
								Returns a localization set with given language acronym 'localAcronym' Returns null on unknown acronym input.
							}
							\item{
								{\bf  Parameters}
								\begin{itemize}
									\item{\texttt{localAcronym} -- Language acronym}
								\end{itemize}
							}%end item
							\item{
								{\bf  Returns} -- Resulting localization 
							}%end item
						\end{itemize}
					}%end item
					\item{ 
						\index{Languages!reload()}
						\hypertarget{studyplanning.view.Languages.reload()}{{\bf  reload}\\}
						\begin{lstlisting}[frame=none]
public void reload()
						\end{lstlisting} %end signature
						\begin{itemize}
							\item{
								{\bf  Description}
								Reloads all locales from the base locale directory
							}
						\end{itemize}
					}%end item
				\end{itemize}
			}
		}
		
		\section{\label{studyplanning.view.Locale}Class \index{Locale} Locale}{
			\hypertarget{studyplanning.view.Locale}{}\vskip .1in 
			Class representing one language. It's used for translating strings to one specific language.\vskip .1in 
		
			\subsection{Declaration}{
				\begin{lstlisting}[frame=none]
public class Locale
				\end{lstlisting}
			}
			
			\subsection{Method summary}{
				\begin{verse}
					\hyperlink{studyplanning.view.Locale.format(java.lang.String, java.lang.Object[])}{{\bf format(String, Object\lbrack \rbrack )}} This translates and formats the localized version of this key.\\
					\hyperlink{studyplanning.view.Locale.getLanguageAcronym()}{{\bf getLanguageAcronym()}} Returns the language acronym.\\
					\hyperlink{studyplanning.view.Locale.translate(java.lang.String)}{{\bf translate(String)}} This translates a key to its localized part.\\
				\end{verse}
			}
			
			\subsection{Methods}{
				\vskip -2em
				\begin{itemize}
					\item{ 
						\index{Locale!format(String, Object\lbrack \rbrack )}
						\hypertarget{studyplanning.view.Locale.format(java.lang.String, java.lang.Object[])}{{\bf  format}\\}
						\begin{lstlisting}[frame=none]
public String format(String key, Object[] objs)
						\end{lstlisting} %end signature
						\begin{itemize}
							\item{
								{\bf  Description}
								This translates and formats the localized version of this key. \texttt{\small \hyperlink{java.lang.String.format(java.lang.String, java.lang.Object[])}{format(String, Object\lbrack \rbrack )}}{\small 
								\refdefined{java.lang.String.format(java.lang.String, java.lang.Object[])}}
							}
							\item{
								{\bf  Parameters}
								\begin{itemize}
									\item{\texttt{key} -- The key to localize.}
									\item{\texttt{objs} -- The objects to use for formatting.}
								\end{itemize}
							}%end item
							\item{
								{\bf  Returns} 
								-- The formatted String. 
							}%end item
						\end{itemize}
					}%end item
					\item{ 
						\index{Locale!getLanguageAcronym()}
						\hypertarget{studyplanning.view.Locale.getLanguageAcronym()}{{\bf  getLanguageAcronym}\\}
						\begin{lstlisting}[frame=none]
public final String getLanguageAcronym()
						\end{lstlisting} %end signature
						\begin{itemize}
							\item{
								{\bf  Description}
								Returns the language acronym. English would be en\_us, German would be de\_de and so on.
							}
							\item{
								{\bf  Returns}
								-- The language acronym of the language. 
							}%end item
						\end{itemize}
					}%end item
					\item{ 
						\index{Locale!translate(String)}
						\hypertarget{studyplanning.view.Locale.translate(java.lang.String)}{{\bf  translate}\\}
						\begin{lstlisting}[frame=none]
public String translate(String key)
						\end{lstlisting} %end signature
						\begin{itemize}
							\item{
								{\bf  Description}
								This translates a key to its localized part.
							}
							\item{
								{\bf  Parameters}
								\begin{itemize}
									\item{\texttt{key} -- The language key.}
								\end{itemize}
							}%end item
							\item{
								{\bf  Returns} 
								-- The localized value of this key or the key itself, if the mapping does not contain the key. 
							}%end item
						\end{itemize}
					}%end item
				\end{itemize}
			}
		}
		
		\section{\label{studyplanning.view.ViewBuilder}Class \index{ViewBuilder} ViewBuilder}{
			\hypertarget{studyplanning.view.ViewBuilder}{}\vskip .1in 
			
			\subsection{Declaration}{
				\begin{lstlisting}[frame=none]
public class ViewBuilder
  extends HttpServlet
				\end{lstlisting}
			}
			
			\subsection{Constructor summary}{
				\begin{verse}
					\hyperlink{studyplanning.view.ViewBuilder()}{{\bf ViewBuilder()}} Creates a new View\\
				\end{verse}
			}
			
			\subsection{Method summary}{
				\begin{verse}
					\hyperlink{studyplanning.view.ViewBuilder.doGet(HttpServletRequest, HttpServletResponse)}{{\bf doGet(HttpServletRequest, HttpServletResponse)}} \\
					\hyperlink{studyplanning.view.ViewBuilder.doHead(HttpServletRequest, HttpServletResponse)}{{\bf doHead(HttpServletRequest, HttpServletResponse)}} \\
					\hyperlink{studyplanning.view.ViewBuilder.doOptions(HttpServletRequest, HttpServletResponse)}{{\bf doOptions(HttpServletRequest, HttpServletResponse)}} \\
					\hyperlink{studyplanning.view.ViewBuilder.doPost(HttpServletRequest, HttpServletResponse)}{{\bf doPost(HttpServletRequest, HttpServletResponse)}} \\
					\hyperlink{studyplanning.view.ViewBuilder.init(ServletConfig)}{{\bf init(ServletConfig)}} Init method called by apache tomcat, will initializes entire program\\
				\end{verse}
			}
						
			\subsection{Constructors}{
				\vskip -2em
				\begin{itemize}
					\item{ 
						\index{ViewBuilder!ViewBuilder()}
						\hypertarget{studyplanning.view.ViewBuilder()}{{\bf  ViewBuilder}\\}
						\begin{lstlisting}[frame=none]
public ViewBuilder()
						\end{lstlisting} %end signature
						\begin{itemize}
							\item{
								{\bf  Description}
								Creates a new View
							}
						\end{itemize}
					}%end item
				\end{itemize}
			}
			
			\subsection{Methods}{
				\vskip -2em
				\begin{itemize}
					\item{ 
						\index{ViewBuilder!doGet(HttpServletRequest, HttpServletResponse)}
						\hypertarget{studyplanning.view.ViewBuilder.doGet(HttpServletRequest, HttpServletResponse)}{{\bf  doGet}\\}
						\begin{lstlisting}[frame=none]
protected void doGet(HttpServletRequest req, HttpServletResponse resp)
						\end{lstlisting} %end signature
						\begin{itemize}
							\item{
								{\bf  Parameters}
								\begin{itemize}
									\item{\texttt{req} -- Request from website}
									\item{\texttt{resp} -- New website to be displayed}
								\end{itemize}
							}%end item
							\item{{\bf  See also}
								\begin{itemize}
									\item{ HttpServlet.doGet(HttpServletRequest request, HttpServletResponse response)}
								\end{itemize}
							}%end item
						\end{itemize}
					}%end item
					\item{ 
						\index{ViewBuilder!doHead(HttpServletRequest, HttpServletResponse)}
							\hypertarget{studyplanning.view.ViewBuilder.doHead(HttpServletRequest, HttpServletResponse)}{{\bf  doHead}\\}
							\begin{lstlisting}[frame=none]
protected void doHead(HttpServletRequest req, HttpServletResponse resp)
							\end{lstlisting} %end signature
							\begin{itemize}
								\item{
									{\bf  Parameters}
									\begin{itemize}
										\item{\texttt{req} -- Request from website}
										\item{\texttt{resp} -- New website to be displayed}
									\end{itemize}
								}%end item
								\item{{\bf  See also}
									\begin{itemize}
										\item{ HttpServlet.doHead(HttpServletRequest request, HttpServletResponse response)}
									\end{itemize}
								}%end item
							\end{itemize}
						}%end item
						\item{ 
							\index{ViewBuilder!doOptions(HttpServletRequest, HttpServletResponse)}
								\hypertarget{studyplanning.view.ViewBuilder.doOptions(HttpServletRequest, HttpServletResponse)}{{\bf  doOptions}\\}
								\begin{lstlisting}[frame=none]
protected void doOptions(HttpServletRequest req, HttpServletResponse resp)
								\end{lstlisting} %end signature
								\begin{itemize}
									\item{
										{\bf  Parameters}
										\begin{itemize}
											\item{\texttt{req} -- Request from website}
											\item{\texttt{resp} -- New website to be displayed}
										\end{itemize}
									}%end item
									\item{{\bf  See also}
										\begin{itemize}
											\item{ HttpServlet.doOptions(HttpServletRequest request, HttpServletResponse response)}
										\end{itemize}
									}%end item
								\end{itemize}
							}%end item
							\item{ 
								\index{ViewBuilder!doPost(HttpServletRequest, HttpServletResponse)}
								\hypertarget{studyplanning.view.ViewBuilder.doPost(HttpServletRequest, HttpServletResponse)}{{\bf  doPost}\\}
								\begin{lstlisting}[frame=none]
protected void doPost(HttpServletRequest req, HttpServletResponse resp)
								\end{lstlisting} %end signature
								\begin{itemize}
									\item{
										{\bf  Parameters}
										\begin{itemize}
											\item{\texttt{req} -- Request from website}
											\item{\texttt{resp} -- New website to be displayed}
										\end{itemize}
									}%end item
									\item{
										{\bf  See also}
										\begin{itemize}
											\item{HttpServlet.doPost(HttpServletRequest request, HttpServletResponse response)}
										\end{itemize}
									}%end item
								\end{itemize}
							}%end item
							\item{ 
								\index{ViewBuilder!init(ServletConfig)}
								\hypertarget{studyplanning.view.ViewBuilder.init(ServletConfig)}{{\bf  init}\\}
								\begin{lstlisting}[frame=none]
public void init(ServletConfig config) 
  throws ServletException
								\end{lstlisting} %end signature
								\begin{itemize}
									\item{
										{\bf  Description}
										Init method called by Apache Tomcat, will initialize the entire program
									}
									\item{
										{\bf  Parameters}
										\begin{itemize}
											\item{\texttt{config} -- Apache tomcat configuration}
										\end{itemize}
									}%end item
									\item{
										{\bf  Throws}
										\begin{itemize}
											\item{\vskip -.6ex \texttt{ServletException} -- Occurring error}
										\end{itemize}
									}%end item
								\end{itemize}
							}%end item
						\end{itemize}
					}
				}
			}
		}

\part{Other}{
	
	\chapter{Procedures}{
		\section{Initialization}{
			This sequence diagram shows how the system is initialized using \textbf{Tomcat}, which starts the ViewBuilder. After its initialization, the ViewBuilder continues to instantiate the Language object, which is responsible for getting the language specific strings of the website and the Controller-Class. The Controller Class is creating support objects to process commands and to communicate with the Model. One of this objects is the WorkflowOperation class, which during its initialization creates its own support objects.
		
			\begin{figure}[ht]
				\centering
				\includegraphics[width=\textwidth]{res/SequenceDiagrams/InitDiagram}
				\caption{System initialization}
			\end{figure}
		}
		
		\clearpage
		
		\section{Generic Interaction}{
			This sequence diagram shows an generic interaction between the View and the Controller. It is exemplary for other interactions and possible outcomes.
			
			\begin{figure}[ht]
				\centering
				\includegraphics[width=\textwidth]{res/SequenceDiagrams/BeispielAblauf}
				\caption{Example interaction}
			\end{figure}
		}
		
		\clearpage 
		
		\section{Generation}{
			Generating requires parsing the workflow id, the study subject and the preferences. After loading the workflow, the Model uses the given informations to generate a workflow. The generation algorithm assigns each module a value. This value is calculated in accordance with the preferences and the module's own properties (e.g. compulsory module or desired module). The workflow is then assembled out of high-value modules, while making sure that it complies with the rules. 
			\begin{figure}[ht]
				\centering
				\includegraphics[width=\textwidth]{res/SequenceDiagrams/Generation}
				\caption{Generation}
			\end{figure}
		}
		
		\clearpage
		
		\section{Verification}{
			The Controller begins the verification process, by parsing the message generated by the View. It then loads the workflow and the workflow's study subject and hands both to the Model. The Model checks if every constraint is satisfied, and creates a Collection of Mistakes, leaving it empty if the workflow is valid. The Collection is handed back to the Controller, which then uses it to generate a Response for the View. In the end the Response is sent to the View, which uses it to build the new website.
			
			\begin{figure}[ht]
				\centering
				\includegraphics[width=\textwidth]{res/SequenceDiagrams/Verification}
				\caption{Verification}
			\end{figure}
		}	
	}

	\chapter{Database}{
		\section{Given Database}{
			\begin{figure}[ht]
				\centering
				\includegraphics[width=\textwidth]{res/ERM/Vorgabe}
				\caption{Given Database}
			\end{figure}
			\begin{itemize}
				\item \textbf{modulhandbuch} - This table saves all modules with their specific data. \\
					\begin{tabular}{|c|c|}
						\hline 
						\textbf{Column} & \textbf{Content} \\
						\hline 
						ID & Primary key for the modules. Used for referencing \\ 
						Modul & Name of the module \\ 
						Vertiefung & The specialization of the module \\ 
						ECTS & The ECTS this module rewards \\ 
						WS\_SS & The semester this module is happening in \\ 
						Information & Anything important like compulsory or core \\ 
						Breich & The section of the module \\ 
						\hline 
					\end{tabular}
				\item \textbf{abhaengigkeit} - This table shows the relation between two modules and their constraint. \\ \\
					\begin{tabular}{|c|c|}
						\hline 
						\textbf{Column} & \textbf{Content} \\
						\hline 
						ID & Primary key for the constraints \\ 
						Modul1 & The first module in this constraint \\ 
						Modul2 & The second module in this constraint \\ 
						Typ & The ID of the constraint type \\ 
						\hline 
					\end{tabular} 
				\item \textbf{regel} - This table saves all constraints and their meaning/name. \\
					\begin{tabular}{|c|c|}
						\hline 
						\textbf{Column} & \textbf{Content} \\
						\hline
						Typ & Primary key of the constraint. Used for referencing \\ 
						Beschreibung & The meaning/name of this constraint \\ 
						\hline 
					\end{tabular} \\ \\
					The content of this table will be implemented as Enums (Constraint) in the program, making this table obsolete for the implementation.
			\end{itemize}
		}
	
		\section{Extended Database}{
			\begin{figure}[ht]
				\centering
				\includegraphics[width=\textwidth]{res/ERM/Erweiterung}
				\caption{Extended Database}
			\end{figure}
			To fulfill all criteria, we need to add user data to the database. These are the tables we plan on adding.
			\begin{itemize}
				\item \textbf{von Nutzer} - This table is a relation between users and workflows. \\ \\
					\begin{tabular}{|c|c|}
						\hline 
						\textbf{Column} & \textbf{Content} \\ 
						\hline 
						StudnetID & The ID of the user and a primary key \\ 
						WorkflowID & The ID of the workflow is a primary key \\ 
						\hline 
					\end{tabular} \\ \\
					No user informations are stored in the database, but there can be multiple user entries for each new workflowID. \\
				\item \textbf{workflow} - This table stores all workflows and their specific informations. \\ \\
					\begin{tabular}{|c|c|}
						\hline 
						\textbf{Column} & \textbf{Content} \\
						\hline 
						ID & Primary key and identifier for references \\ 
						Name & The user generated name of the workflow \\ 
						\hline 
					\end{tabular} \\
				\item \textbf{besteht aus} - This table is a relation between the workflows and the modules in it. It shows in which semester a module takes place. \\ \\
					\begin{tabular}{|c|c|}
						\hline 
						\textbf{Column} & \textbf{Content} \\
						\hline 
						WorkflowID & The ID of the workflow, is a primary key \\ 
						ModuleID & The ID of the module, is a primary key \\ 
						SemesterNumber & The semester this module is part of \\ 
						\hline 
					\end{tabular} 
			\end{itemize}
		}
		
		\section{Combined Database}{
			\begin{figure}[ht]
				\centering
				\includegraphics[width=\textwidth]{res/ERM/Gesamt}
				\caption{Combined Database}
			\end{figure}
		}
	}
}		

\printindex

\begin{appendices}
			\multido{\i=1+1}{17}{
				\begin{figure}[p]
					\centering
					\includegraphics[angle=90, width=\textwidth]{res/ClassDiagram/AllSliced/allMarked-\i}
				\end{figure}
			}
\end{appendices}
\end{document}