\documentclass{article}

\usepackage[margin=2.5cm]{geometry}

\usepackage[utf8]{inputenc}
\usepackage[T1]{fontenc}
\usepackage[german]{babel}

\usepackage{hyperref}
\hypersetup{
pdftitle={Pflichtenheft},
bookmarks = true
}
\usepackage[toc]{glossaries}

\usepackage{graphicx}

\usepackage[shortlabels]{enumitem}
\usepackage{parskip}


\makeglossaries



\title{RaGE Pflichtenheft}
\author{Jonas Kasper, Bernard Hohmann, Thomas Fischer, Christian Jung, Jonas Linßen}

\begin{document}
	\maketitle
	
	\newglossaryentry{Graph}{
		name = ungerichteter einfacher Graph,
		description = {Tupel $G = (V,E)$ mit Knotenmenge $V$ und Kantenmenge $E \subseteq \{X \subset V \mid \#X = 2\}$}
	}
	
	\newpage
	
	\tableofcontents
	
	\newpage
	
	
	\section{Zielbestimmung}
	Das Produkt ermöglicht dem Lehrstuhl IPD Böhm die automatische Generierung ungerichteter, einfacher Graphen sowie einfacher Hypergraphen und die automatisierte Evaluation verschiedener Heuristiken zur Lösung von bisher ungelösten Problemen der Informatik.
	
	\subsection{Musskriterien}
	\begin{enumerate}[(M1)]
		\item{Es können einfache, ungerichtete Graphen und einfache Hypergraphen nach durch den Nutzer spezifizierten Kriterien zufällig generiert werden.}
		\item{Auf die generierten Graphen können durch den Nutzer auszuwählende Heuristiken angewandt werden.}
		\item{Die Graphen und Ergebnisse der Heuristiken werden in einer graphischen Oberfläche angezeigt.}
		\item{Die Graphen können durch den Nutzer modifiziert werden, d.h. es können Knoten und Kanten hinzugefügt und gelöscht werden. Insbesondere ist es möglich Graphen von Hand zu zeichnen.}
		\item{Für die folgenden offenen Probleme sind Heuristiken implementiert:
			\begin{enumerate}[i)]
				\item{Total Coloring Conjecture}
				\item{Erdös Faber Lovasz Conjecture}
			\end{enumerate}
		}
	\end{enumerate}
	
	\subsection{Kannkriterien}
	\begin{enumerate}[(K1)]
		\item{Die folgenden Eigenschaften von Graphen können erkannt werden:
			\begin{enumerate}[i)]
				\item{Zusammenhang}
				\item{Baum}
				\item{Bipartition}
				\item{Planarität}
				\item{Intervallgraph / Chordalität}
			\end{enumerate}
			Die zufällige Generierung der Graphen lässt sich insbesondere auf eine Auswahl dieser Eigenschaften einschränken bzw. einige der genannten Eigenschaften können dabei ausgeschlossen werden.
		}
		\item{Weitere Heuristiken können als Plugins in das Programm eingebunden werden. Dies ermöglicht die Anwendung auf andere ungelöste Probleme der Informatik als die, für die bereits Heuristiken implementiert sind.}
		\item{Die folgenden Sprachen werden unterstützt
			\begin{enumerate}[i)]
				\item{Deutsch}
				\item{Englisch}
			\end{enumerate}
		}
	\end{enumerate}
	
	\subsection{Abgrenzungskriterien}
	\begin{enumerate}[(A1)]
		\item{Die Heuristiken sind nicht an sich parallelisierbar.}
		\item{Es können keine gerichteten (Multi-)Graphen und keine allgemeinen Hypergraphen generiert werden.}
		\item{Bei der Generierung erfolgt keine Erkennung von Duplikaten. Die Erkennung von Graphenisomorphie ist nicht möglich.}
		\item{Das Programm kann weder für das Total Coloring Conjecture, noch für das Erdös Faber Lovasz Conjecture einen Beweis finden.}
	\end{enumerate}
	
	
	
	
	\section{Produkteinsatz}
	Das Produkt dient der Auswertung von Heuristiken für ungelöste Probleme der Graphentheorie auf möglichst vielen generierten Graphen.
	
	\subsection{Anwendungsbereiche}
	Das Programm ist für die Forschung in der Mathematik und Informatik, genauer der Graphentheorie, vorgesehen.
	
	\subsection{Zielgruppen}
	Das Produkt wird für die Nutzung im Lehrstuhl IPD Böhm am KIT entwickelt. Eine Anwendung in anderen Forschungseinrichtungen, Universitäten und Hochschulen sowie durch Privatpersonen mit Interesse an offenen Problemen der Graphentheorie ist möglich und erwünscht.
	
	\subsection{Betriebsbedingungen}
	Das Programm läuft auf Privat- und Betriebsrechnern (z.B. in einer Büroumgebung).
	
	
	
	\section{Produktumgebung}
	
	\subsection{Software}
	Die folgende Software ist für die Lauffähigkeit des Programms hinreichend:
	\begin{itemize}
		\item{Windows 10}
		\item{Java 8 Runtime Environment}
	\end{itemize}
	Durch die Plattformunabhängigkeit der JRE ist auch eine Anwendung auf anderen Betriebssystemen möglich.
	
	\subsection{Hardware}
	Das Programm läuft auf Rechnern mit durchschnittlicher Rechenleistung. Es benötigt mindestens:
	\begin{itemize}
		\item{4GB Arbeitsspeicher}
		\item{4GB Festplattenspeicher}
		\item{Maus und ggf. Tastatur}
	\end{itemize}
	
	
	
	\newpage
	\section{Funktionale Anforderungen}
	Die durch * markierten Einträge sind Kannkriterien und nur aus Vollständigkeitsgründen im Folgenden aufgeführt.
	
	\subsection*{/F10/ Zufällige Generierung von Graphen} \label{f10} \addcontentsline{toc}{subsection}{\nameref{f10}}
	
	\subsubsection*{/F11/ Voreinstellungen} \label{f11} \addcontentsline{toc}{subsubsection}{\nameref{f11}}
	
	Die folgenden Voreinstellungen können getroffen werden:
	\begin{enumerate}[i)]
		\item{Es kann ausgewählt werden, ob einfache ungerichtete Graphen (SGs) oder einfache Hypergraphen (SH) generiert werden sollen}
		\item{Die Anzahl der zu generierenden (Hyper-) Graphen kann eingestellt werden. Es werden zwischen 1 und 100 000 Graphen unterstützt.}
	\end{enumerate}
	
	\subsubsection*{/F12/ Einstellungen für Graphen} \label{f12} \addcontentsline{toc}{subsubsection}{\nameref{f12}}
	Hat der Nutzer die Generierung von SGs gewählt, so kann er die folgenden Einstellungen vornehmen:
	\begin{enumerate}[i)]
		\item{Die Anzahl von Knoten $n$ der zu generierenden Graphen kann festgelegt werden. Es gilt $1 \leq n \leq 1000$.}
		\item{Der minimale und maximale Knotengrad $\delta_{min}$ und $\Delta_{max}$ der zu generierenden Graphen kann festgelegt werden. Für einen beliebigen generierten Graphen $G$ gilt dann $$0 \leq \delta_{min} \leq \delta(G) \leq \Delta(G) \leq \Delta_{max} \leq n-1.$$}
		\item[iii)*]{Die folgenden Grapheigenschaften können (falls gewünscht) bei der Generierung ausgeschlossen oder forciert werden:
			\begin{enumerate}[-]
				\item{Zusammenhang}
				\item{Baum}
				\item{Bipartition}
				\item{Planarität}
				\item{Intervallgraph / Chordalität}
			\end{enumerate}
		}
	\end{enumerate}
	
	\subsubsection*{/F13/ Einstellungen für Hypergraphen} \label{f13} \addcontentsline{toc}{subsubsection}{\nameref{f13}}
	Hat der Nutzer die Generierung von SHs gewählt, so kann er die folgenden Einstellungen vornehmen:
	\begin{enumerate}[i)]
		\item{Die Anzahl von Knoten $n$ der zu generierenden Graphen kann festgelegt werden. Es gilt $1 \leq n \leq 1000$.}
		\item{Der minimale und maximale Knotengrad $\delta_{min}$ und $\Delta_{max}$ der zu generierenden Graphen kann festgelegt werden. Für einen beliebigen generierten Graphen $G$ gilt dann $$0 \leq \delta_{min} \leq \delta(G) \leq \Delta(G) \leq \Delta_{max} \leq n-1.$$}
		%\item{Es kann (falls gewünscht) ein $r$ festgelegt werden, sodass die generierten SHs $r$-uniform sind. Dabei gilt $$2 \leq r \leq n.$$}
		\item[iii)*]{Es kann (falls gewünscht) die minimale und maximale Kardinalität $k_{min}$ und $k_{max}$ der Hyperkanten festgelegt werden. Für einen beliebigen generierten Hypergraph $G$ und eine Hyperkante $e \in E(G)$ gilt dann $$2 \leq k_{min} \leq \# e \leq k_{max} \leq n.$$ Insbesondere lässt sich die Generierung über $k_{min} = k_{max} =: k$ auch auf $k$-uniforme Hypergraphen beschränken.}
	\end{enumerate}
	
	\newpage
	
	\subsection*{/F20/ Anwendung von Heuristiken} \label{f20} \addcontentsline{toc}{subsection}{\nameref{f20}}
	
	\subsubsection*{/F21/ Liste von Heuristiken} \label{f21} \addcontentsline{toc}{subsubsection}{\nameref{f21}}
	Es kann eine Liste von Heuristiken bestimmt werden, die auf jeden der generierten Graphen angewandt werden. Diese obliegt den folgenden Einschränkungen:
	\begin{enumerate}[i)]
		\item{Die möglichen Heuristiken sind abhängig von dem generierten Graphentyp.}
		\item{Dieselbe Heuristik kann mit unterschiedlichen individuellen Einstellungen mehrfach in der Liste vorkommen.}
		\item{Dieselbe Heuristik kann nicht mehrfach mit den gleichen Einstellungen in der Liste vorkommen.}
		\item{Eine Änderung der Liste bewirkt eine Neuberechnung aller neuen oder veränderten Heuristiken auf allen generierten Graphen, nicht jedoch die Neuberechnung von bereits angewandten und unverändert gebliebenen Heuristiken.}
	\end{enumerate}
	
	\subsubsection*{/F22/* Ergänzung neuer Heuristiken} \label{f22} \addcontentsline{toc}{subsubsection}{\nameref{f22}}
	Es können neue Heuristiken als Plugins hinzugefügt werden. Die Erkennung der Plugins geschieht zum Programmstart und ist während der Laufzeit nicht mehr möglich.
	
	
	\newpage
	
	\subsection*{/F30/ Modifikation von Graphen} \label{f30} \addcontentsline{toc}{subsection}{\nameref{f30}}
	\subsubsection*{/F31/ Modifikation der einfachen Graphen} \label{f31} \addcontentsline{toc}{subsubsection}{\nameref{f31}}
	Das Programm bietet die Möglichkeit die Graphen in der folgenden Art und Weise zu modifizieren:
	\begin{enumerate}[i)]
		\item{Es können Knoten hinzugefügt werden.}
		\item{Kanten zwischen zwei bestehenden und noch nicht adjazenten Knoten können hinzugefügt werden.}
		\item{Jeder Knoten $v$ kann gelöscht werden. Alle zu $v$ inzidenten Kanten werden automatisch ebenfalls gelöscht.}
		\item{Der Nutzer kann Kanten löschen.}
		\item{Knoten können kontrahiert werden. Auftretende Mehrfachkanten werden automatisch zu einfachen Kanten reduziert. Auftretende Schleifen werden gelöscht.}
		\item{Jeder Knoten $v$ kann dupliziert werden, d.h. es wird ein neuer Knoten $v$ mit derselben Nachbarschaft wie $v$ hinzugefügt.}
	\end{enumerate}
	
	Eine beliebige Auswahl von $n$ Knoten kann auf die folgenden Graphen ergänzt werden:
	\begin{enumerate}[i)]
		\item{\textbf{Leerer Graph} $E_n$: Alle Kanten zwischen den ausgewählten Knoten werden gelöscht.}
		\item{\textbf{Pfad} $P_n$: In Reihenfolge der Auswahl werden zwischen aufeinanderfolgenden und nichtadjazenten Knoten Kanten hinzugefügt. Der erste und der letzte Knoten werden nicht explizit durch eine Kante verbunden. Wenn sie jedoch bereits adjazent sind, wird die bestehende Kante nicht gelöscht.}
		\item{\textbf{Kreis} $C_n$: Die Knoten werden wie beim Pfad in der Reihenfolge ihrer Auswahl verbunden mit dem Unterschied, dass auch der erste und letzte Knoten explizit verbunden werden.}
		\item{\textbf{Clique} $K_n$: Je zwei Knoten aus der Auswahl, die nicht bereits adjazent sind, werden durch eine neue Kante verbunden.}
	\end{enumerate}
	
	\subsubsection*{/F32/ Modifikation der Hypergraphen} \label{f32} \addcontentsline{toc}{subsubsection}{\nameref{f32}}
	TODO Hypergraph modifikation
	
	\subsubsection*{/F33/ Manuelle Erstellung von Graphen und Hypergraphen} \label{f33} \addcontentsline{toc}{subsubsection}{\nameref{f33}}
	Die in /F31/ und /F32/ erwähnten Modifikationen implizieren die Möglichkeit Graphen von Hand im Programm einzutragen.
	
	
	\subsection*{/F40/ Speichern und Laden} \label{f40} \addcontentsline{toc}{subsection}{\nameref{f40}}
	
	
	\subsection*{/F50/ Ausgabe über eine graphischer Benutzeroberfläche} \label{f50} \addcontentsline{toc}{subsection}{\nameref{f50}}
	Die graphische Benutzeroberfläche gliedert sich im wesentlichen in die folgenden Bereiche:
	
	\subsubsection*{/F51/ Graphgenerierung} \label{f51} \addcontentsline{toc}{subsubsection}{\nameref{f51}}
	In diesem Fenster können alle Einstellungen aus /F10/ für die zu generierenden Graphen vorgenommen werden. Nach Bestätigung durch den Nutzer werden die Graphen unter Berücksichtigung dieser Einstellungen generiert.
	
	\subsubsection*{/F52/ Preview} \label{f52} \addcontentsline{toc}{subsubsection}{\nameref{f52}}
	
	\subsubsection*{/F53/ Detailansicht} \label{f53} \addcontentsline{toc}{subsubsection}{\nameref{f53}}
	
	\subsubsection*{/F54/ Grapheditor} \label{f54} \addcontentsline{toc}{subsubsection}{\nameref{f54}}
	
	
	\subsection*{/F60/ Heuristiken für das Total Coloring Conjecture} \label{f60} \addcontentsline{toc}{subsection}{\nameref{f60}}
	\subsection*{/F70/ Heuristiken für das Erdös Faber Lovasz Conjecture} \label{f70} \addcontentsline{toc}{subsection}{\nameref{f70}}
	
	
	
	\section{Nichtfunktionale Anforderungen}
	
	
	
	\section{Produktdaten}
	
	
	\section{Graphische Benutzeroberfläche}
	
	
	\section{Systemmodelle}
	\subsection{Szenarien}
	\subsection{Anwendungsfälle}
	\subsection{Durchführbarkeitsanalyse}
	
	\section{Glossar}
	\glsaddall
	\printglossaries
\end{document}